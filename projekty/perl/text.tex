\documentclass[12pt,a4paper]{book}
\usepackage[polish]{babel}
\usepackage[T1]{fontenc}
\usepackage[utf8x]{inputenc}
\usepackage{graphicx} 
\usepackage{caption}
\usepackage{float}
\usepackage{amsmath}
\usepackage{fancyhdr}
\usepackage{hyperref}
\author{sdf
}
\title{dsfdsf
}
\begin{document}\maketitle
\chapter{dsfsdf
}
              Prolog
\\Ami i Dylan Coutcherowie mieli dwójkę dzieci- jedno właśnie przyszło na świat. Tygodniowa Hope wielkimi błękitnymi oczami obserwowała otaczającą ją rzeczywistość. Jeszcze nie wiedziała, że sala szpitalna to tylko jedna milionowa całej Ziemii. Widziała kolory i twarze, i nie sposób domyślić się co też sobie wtedy myślała. Mama czule trzymała ją w ramionach, przepełniona miłością i radością. Tata z jej  starszą siostrą Cinną siedzieli po dwóch stronach łóżka. 
- Czemu daliście jej takie dziwne imię?- spytała dziewczynka, zazdrośnie łypiąc oczami na Hope.
- Kochanie, to śliczne imię. Mi się bardzo podoba- tłumaczyła mama.
- Nie znam nikogo o takim imieniu- odpowiedziałą na to Cinna. 
Ami wiedziała, że jej córeczka jest zazdrosna o swoją nową siostrę, to było normalne. Cinna od teraz nie będzie najważniejsza i jedyna, będzie ktoś jeszcze, kto niejako zastąpi ją. Po mimo tego Ami przykro się zrobiło na słowa córki i oczy się jej zaszkliły. Wyciągnęła w stronę męża dłoń, a ten chwycił ją i ucałował. Posłała mu blady uśmiech. Nie musiała mu nic mówić, rozumieli się bez słów. Wiedział jak zmęczona jest ich obecną sytuacją. Nie chodziło o urodzenie dziecka, to dodawało jej energii, a nie zabierało. Energię wysysała z niej myśl o ich planach. Mieli zamiar za parę miesięcy założyć tajną organizację opozycyjną do obecnej władzy. Nie robili tego legalnie, bo się nie dało. Wszystko było tajemnicą, tylko zaufane osoby wiedziały o ich projekcie. Po co małżeństwo z dwójką dzieci narażało swoje życie, bo to  co chcieli zrobić groziło karą śmierci? Były ku temu powody i było ich z pewnością wiele. 
Dylan  uśmiechnął się do Ami najszerzej jak potrafił, by dodać jej otuchy. A ona od razu poczuła się lepiej - to w tym uśmiechu się zakochała. Wyrażał jednocześnie miłość i pewność, że razem pokonają wszelkie przeszkody. 
Wziął od  niej Hope, by mogła zasnąć. Trzymał córkę na rękach, a ona zaciekawiona się w niego wpatrywała. Rozrzewniło go to. Przypomniał sobie jak trzymał tak kiedyś Cinnę. To uczucie, które towarzyszyło mu wtedy i teraz było trudne do opisania. Duma z życia, któremu się dało początek i bezgraniczna miłość do niego. Poza tym, gdy patrzył w oczy dziecka, miał wrażenie, że jest jak otwarta księga, że nic się nie da przed nim ukryć. I to było piękne. Ta prostota dziecka.
Hope usnęła, więc ułożył ją w łóżeczku. Cinna zasnęła na łóżku obok Ami. Stwierdził, że zrobiłby to samo co one. Oczy kleiły mu się ze zmęczenia, choć na dobrą sprawę nie robił dziś nic męczącego.  
Wyszedł na korytarz, gdzie siedzieli jego rodzice- Catrina i Greg.
- Jak się czuje Ami? -  spytała Catrina.
- Obecnie zasnęła. A poza tym jest zmęczona i zdenerwowana, z oczywistych względów. 
- Trzeba było to przełożyć na następny rok, nikomu nie sprawiło by to różnicy - skomentował Greg.
- Tato, nie będziemy tego ponownie rozstrząsać. Podjęliśmy już decyzję.
- Wiem, wiem, przepraszam – położył mu rękę na ramieniu. - Wiesz, że jesteśmy z tobą. Po prostu to... trudne dla nas wszystkich. 
- Idź się położyć- poprosiła Catrina. - Zajmiemy się tu wszystkim jak się obudzą. 
Nie posłuchał jej, tylko usiadł na krześle obok nich. Nie próbowali protestować. Ciężko oparł przedramiona na kolanach i przymknął oczy. Wiedział, że rozsądnym było przekimać kilka godzin, ale za bardzo się bał, że do szpitala wparują wojskowi, chcąc ich aresztować. Nie mógł być wtedy daleko swoich skarbów-  Ami, Cinny i Hope. Co prawda nie miał się o co martwić, nikt nic nie wiedział o ich planach, tylko zaufani ludzie. Jak ich przyjaciel Oscar Johnson- sławny polityk, minister spraw zagranicznych. To dzięki niemu dowiedziali się o ciemnej stronie Rządu, choć już wcześniej mieli pewne przypuszczenia, ale dzięki niemu mieli dowody. To dlatego zdecydowali się na stworzenie organizacji i narażnie swojego życia dla poprawienia bytu społeczeństwa. To była trudna decyzja, byli świadomi, że narażają też swoje dzieci na niebezpieczeństwo. Oboje mieli przez to wielkie wyrzuty sumienia i nie dało się o nich zapomnieć. Tak naprawdę już sprawili, że życie ich dzieci nie będzie łatwe i kolorowe. Podjęli za nie decyzje. Czy one będą potrafiły to zrozumieć, gdy będą większe? Czy jakieś tłumaczenie da efekty? Niewiadomo. Ami i Dylan postanowili już, że nic nie powiedzą dzieciom o organizacji, dopóki nie będzie to konieczne. A w przyszłości nie zmuszą ich, by poszli w ich ślady.
Ostatecznie przysnął na siedząco. 


                     

                                     
                       Rozdział 1

Oscar poprawił poły marynarki i zapukał do drzwi prezydenta. Gdy dostał telefon od sekretarki z prośbą o spotkanie z Brownim był mocno zaskoczony. Prezydent rzadko zapraszał do siebie swoich współpracowników. Wolał pisać emaile albo pisma. Gdy prosił kogoś do biura musiała być to sprawa wysokiej rangi. 
Prezydent podniósł się i uścisnął mu dłoń. Potem oboje zasiedli po swoich stronach biurka. Browni ułożył ręce w kształt piramidki i przypatrywał się swojemu gościowi. Widział zdenerwowanie czające się pod jego skórą, które próbował zatuszować opanowaniem i neutralną miną. Ale Browni był wprawnym obserwatorem i nic nie potrafiło się przed nim ukryć. Spodziewał się takiej rekacji po Johnsonie i z zadowoleniem uniósł teraz kącik ust. 
- Zastanawiasz się zapewne po co Cię wezwałem. Nie będę owijał w bawełne – nachylił się nad stołem- zadziwiłeś mnie. Nie wiedziałem, że jesteś na tyle sprytny by mieć aż dwie maski, a jednak. Wiem o tej całej konspiracji, Ruchu Oporu, czy jak to zwiecie. 
Oscar zbladł. Nie tego się spodziewał, idąc tu. Miał dostać pochwałę albo bardzo ważny projekt do wykonania, a nie dowiedzieć się, że jego drugie życie zostało właśnie zdekonspirowane. 
Przełknął ślinę i próbował sprawiać pozory idioty, który nie ma pojęcia o czym do niego mówią. 
- Nie mam pojęcia o niczym takim. 
- Nie kłam. Mam wiarygodne źródła, zapewniam. Zresztą wiem o tym nie od dziś. Nie do wiary, że działacie już sześć lat, bo to o sześć lat za długo. Powiedz mi jak miewają się twoi przyjaciele, bo nim są nieprawdaż? Ami i Dylan Coutcherowie. Piękni ludzie, tacy dobrzy i mają urocze dzieci Cinna, Hope i Pou. 
Z każdym słowem Oscar czuł się coraz gorzej jakby właśnie ktoś mu mówił, że ma raka. 
- Nie mam pojęcia...- wydukał.
- Łżesz. Ale przejdźmy do sedna. Twoi kumple nieźle sobie radzą, aż za bardzo. Znaleźli sojuszników za granicą, ale nie doszło jeszcze do oficjalnych umów. Mają się odbyć za tydzień 23 kwietnia. Popraw mnie, jeśli się mylę. 
Oscar milczał.
- Tak myślałem. A więc, jak widzisz sprawy zaszły za daleko. Nie możemy dopuścić żeby grupa idiotów zburzyła w naszym kraju harmonię i ład. Nie chcemy kolejnych rozlewów krwi. Nie muszę chyba przypominać pamiętego 6 sierpnia 2024?
- Nie - Ocar spuścił głowę. 
Był powalony przez rywala o wiele od niego silniejszego. Nie miał z nim żadnych szans. 
- Dobrze, że to sobie wyjaśniliśmy- ciągnął Browni. -Twoje zadanie jest proste. Umówisz spotkanie z Ami i Dylanem, a podczas ich podróży stanie się wypadek. I po kłopocie. 
Oscar z niedowierzaniem pokręcił głową. Chciało mu się płakać i uciec daleko od tego wszystkiego. Jak mógł do tego dopuścić? Jak się zdradził? Co miał teraz zrobić?
- Nie zrobię tego - powiedział, kumulując w sobie resztki pewności siebie. 
- Słucham? 
- Nie zrobię tego - powtórzył z większą mocą. 
- A mnie się wydaje, że nie masz tu nic do gadania. Widzisz, nie chciałem sięgać po tak drastyczne rozwiązanie, ale nie dajesz mi wyboru. Jeśli nie wyślesz im tej wiadomości, zabiję twoją żonę i syna. 
Oscar poczuł się tak jakby ktoś wydrążył w nim dziurę sztyletem.  
- Nie zrobisz tego. 
- Owszem zrobię, jeśli się nie podporządkujesz. 
- Ale, ale... – próbował coś powiedzieć, ale tylko załamał ręce i zaczął szlochać. 
- Wybór jest prosty według mnie. Przecież nie poświęcisz życia swojej żony, swojego syna dla przyjaciół. Pomyśl kto dał Ci więcej szczęścia. Rodzina, która Cię kocha czy przyjaciele, którzy narażają tą rodzinę na niebezpieczeństwo. 
- To przez Ciebie to jest niebezpieczne! Tylko przez Ciebie! Gdyby była tu prawdziwa demokracja a nie tylko jej ułuda, nie trzeba było się chować ze swoimi poglądami. 
- Nie zapominaj, że ty też w tym siedzisz, hę? Ty też tworzysz tą ułudę. 
- Nieprawda. Nigdy jej nie tworzyłem. Robiłem wszystko co mogłem, żeby było lepiej dla ludzi. Dla tych ludzi, którymi gardzisz. 
- Jesteś zabawny. Myślisz, że masz tu na coś wpływ. Że nie kontroluję każdej ustawy, którą tacy jak ty chcą przepchnąć w sejmie. To wprost niesamowite jak głupi jesteś. I jak słaby. Nie wstyd Ci tak się mazgaić. Co z Ciebie za facet. 
- Lepszy od Ciebie. 
Browni obserował ten spektakl z rosnącym rozbawieniem. Oscar był jeszcze słabszym zawodnikiem niż zakładał, tym lepiej dla niego. Mniej roboty. 
- Miło było porozmawiać, ale twój czas się już skończył. Napiszesz list i go wyślesz, zrozumiano? – przybrał surowy ton.
Oscar już odpuścił, przestał walczyć. Nie było już nic co mógł zrobić, nie było już nic na co miał wpływ. Chyba że...
- Pod jednym warunkiem: nie krzywdź dzieci. Zostaw je w spokoju. Proszę
- Uważasz, że masz prawo stawiać mi warunki. Nonsens. To ja tu rządzię. Albo to zrobisz albo stracisz rodzinę. A teraz juz idź. Hopkins pójdzie z tobą, by przypilnować co tam napiszesz w liście i doprowadzi sprawę do końca. Miłego dnia. 
Oscar opuszczał gabinet jako inny człowiek. Zrezygnowany, zrozpaczony i złamany. 
Za to Browni zadowolony rozsiadł się w fotelu. Jak zwykle nie mógł się nadziwić jak beznadziejnych i słabych psychicznie miał podwładnych. Każdego mógł złamać jak gałązkę i z lubością to robił. Ludzie nie pojmowali najprostszych prawd. Każda więź czyni nas słabaszymi w oczach wroga, bo można jej użyć do szantażu. Nie trzeba wynajdywać na kogoś brudów, starczy zagrozić jego rodzinie, a wtedy będzie potulny jak baranek. Trzeba też by grożący był rzeczywiście zdolny do czynów, którymi grozi. I koniec. 
„Świat w ogóle jest prosty- pomyślał Browni- tylko ludzie lubią go sobie komplikować”.

                                       ***
Oscar wrócił do domu. Do jego nozdrzy dopłynęłe smakowite zapachy z kuchni. Zastał w niej żonę –Melani, krzątającą się radoście w okół pięknie nakrytego stołu. Dziewięcioletni Brave pomagał mamie, ale na widok ojca, rzucił się go przywitać. Ocar przytulił syna z tkliwością. „Prawie Cię straciłem”- pomyślał. Pocałował żonę i zasiedli do obiadu. 
- Jakiś zmarnowany jesteś. Ciężko w pracy? – spytała Melani. 
- Tak. Szykujemy duży projekt i jeszcze wiele trzeba w nim poprawić.
Kłamał jak z nut. Projekt już dawno był wykonany. Nim dotarł do domu, postanowił sobie, że nic nie powie żonie o dzisiejszej rozmowie z Brownim. Bał się jak zaraguje, że będzie histeryzowała. Poza tym nie chciał wkładać na niej ciężaru winy, bo na pewno czułaby się winna, musiał go nieść sam. 
Przy obiedzie słuchał relacji syna ze szkoły i napawał się dumą z jego osiągnięć. W tej chwili, siedząc z rodziną przy stole, czuł że dobrze zrobił, że nie mógł postąpić inaczej. Rodzina była dla niego wszystkim, nie mógł jej poświęcić. 
Ale już tego samego wieczoru, kładąc się do łóżka czuł wyrzuty sumienia. Podpisał na swoich przyjaciół wyrok. Zdradził ich. Nigdy sobie tego nie wybaczy. Nigdy. 

                  Rozdział 2

To był deszczowy dzień, jakby świat już przygotował się na opłakiwanie dzisiejszych wydarzeń. 
Dla Hope to był jeden z tych dni, których bardzo nie lubiła- dzień wyjazdu rodziców w „ważnych sprawach”. Nie lubiła, gdy byli daleko. Zawsze bała się, że coś im się stanie i nigdy więcej ich nie zobaczy. Tata powtarzał jej za każdym razem, że za nim się obejrzy, już będą w domu i nie ma powodu do lęku. Po mimo tego przed każdym planowanym wyjazdem prosiła, żeby jej nie zostawiali. Liczyła, że kiedyś jej posłuchają. Ale to nie był jeszcze ten dzień. 
Siedziała przy oknie z nosem przyciśniętym do szyby. Lubiła deszczowy krajobraz i te zabawne kropelki sunące po tafli szkła. 
Po domu kręcili się rodzice i dziadkowie dopakowując walizki i głośno wymieniając się uwagami. Napięcie wysiało w powietrzu jak gęsta burzowa chmura. 
Hope odlepiła się od szyby i pobiegła do pokoju, który dzieliła z bratem- dwuletnim Pou. Mały siedział na podłodze i bawił się klockami, którymi kiedyś bawiły się Hope i Cinna. Hope wyciągnęła rysunek z szuflady. Chciała go dać rodzicom na szczęście jakby to był talizman. Na każdą podróż był nowy talizman. Babcia spytała ją kiedyś czy jeden nie wystarczy na zawsze, na co kręciła główką poirytowana. 
- Babciu, on ma moc tylko na jeden wyjazd i ani chwili więcej. 
- A co to za moc?
- Moja miłość - uśmiechnęła się dziewczynka. 
Z talizmanem w jednej ręce i bratem u boku skierowała się do rodziców, którzy już zakładali płaszcze. Mama na ich widok szeroko się uśmiechnęła, choć Hope w jej oczach dostrzegała smutek. Ami wzięła na rece synka i wyściskała go, potem uklękła i pożegnała się z najstarszą córką Cinną, a na koniec z Hope. W tej samej kolejności żegnał dzieci Dylan. Kiedy już oboje byli zajęci Hope, ona wręczyła im talizman. 
- Jest wspaniały, kochanie - powiedziała mama. 
- To cała nasza rodzina. To wy i ja – wskazywała postacie na rysunku, a to Cinna i Pou i dziadkowie, a to piesek, o którym marzę. 
- Kochanie, jesteś za mała na pieska. 
- Mam sześć lat - oznajmiła dumnie. 
- Oczywiście, to nie lada wiek. Porozmawiamy o piesku jak wrócimy, dobrze? I skończymy czytać „Piotrusia Pana”...
- I upieczemy tort - weszła jej w słowo Hope. 
- Co tylko zapragniesz. 
Rodzice, jeszcze raz wszystkich powyściskali. Dziadkowie jeszcze raz życzyli im udanej podróży i spytali, czy wszystko mają. A potem po całym zamieszaniu nie było śladu. Ami i Dylan odjechali czarnym samochodem w sinął dal. A deszcz padał dalej. 
                                              
                                                   ***
Tego samego dnia Hope nałożyła kalosze, kurtkę przeciwdeszczową i wybiegła z bratem na podwórko. Potem dołączyła do nich także Cinna. Hope wskakiwała w każdą kałużę, rozchlapując wodę na wszystkie strony. Oczami wyobraźni widziała jak będzie wskakiwać w kałuże razem ze swoim pieskiem, o ile rodzice się w końcu na niego zgodzą. Od roku sumiennie przypominała im, że chciała mieć pieska, a oni powtarzali, że jest jeszcze za mała. Ale i tak się nie poddawała. Kiedyś się zgodzą- pocieszała się w duchu. 
Zastanawiała się, kiedy rodzice będą u celu i zadźwonią, by ich o tym poinformować. Zawsze mówiła wtedy, że talizman zadziałał i była z siebie dumna. Oczywiście, wiedziała, że tak naprawdę to nie talizman zadziałał, a Opatrzność Boża, ale stwierdziła, że Bóg się chyba nie obrazi, jak przypisze sobie część tej zasługi. Jej rodzice byli wierzący, tak jak dziadkowie. Wspólnie modlili się wieczorami. W intencjach Hope zawsze prosiła o psa i by rodzice nie musieli już nigdy wyjeżdżać z miasta. Na razie na jedno i drugie jeszcze czekała. Tata mawiał, że czekanie to połowa przyjemności. 
                               
                                       ***
 Gdy Hope kładła się spać rodzice jeszcze nie zadźwonili. Trochę się denerwowała, ale dziadkowie uspakajali ją, że może coś zatrzymało ich w drodze. Po mimo tego zasypając usłyszała z kuchni jak babcia płacze. A potem zasnęła. 
                                     ***
Rano nie było telefonu. Wieczorem nie było telefonu. Babcia nie wychodziła ze swojej sypialni. Dziadek stał się markotny. Pou nie rozumiał co się dzieje, Cinna obgryzała paznokcie i popłakiwała, a Hope czekała przy oknie. Tak jak wczoraj siedziała przyklejona do szyby  ,a deszcz padał. Strużyki deszczu płynęły po szybie, a łzy płynęłu po jej policzkach. Ale dalej miała nadzieję. Co sobie dziadkowie myślą, że talizman nie zadziałał, że Bóg nie czuwał?- myślała. Rodzice na pewno zadźwonią, a jak nie to i tak wrócą do domu. Otarła łzy i wyszła pobawić się na podwórko. Czy ja nie wiem czegoś, co oni wiedzą? Przecież, mógł im się zepsuć telefon. Mógł, prawda? Mógł, ale do tej pory znaleźli by sposób, żeby się skontaktować – podpowiedział jej głos z wnętrza głowy. Zaszlochała i wróciła do domu, gdzie rzuciła się na łóżko i płakała dopóki nie zasnęła. 

Rozdział 3

Trzy dni po wyjeździe rodziców pod dom zajechał czarny wysłużony samochód. Wyszło z nich dwóch młodych mężczyzn. Jeden był barczysty i niski, a drugi szczupły i wysoki. Zapukali do drzwi rodziny Coutcherów. 
Otworzył im dziadek. Rozpoznał ich i szybko wpuścił do środka. 
- Albert, Bob, dobrze Was widzieć. 
- Pana również, ale niestety nie mamy dobrych wieści – odezwał się Bob. 
Greg posmutniał i głośno przełknął ślinę. 
- Tego się obawiałem. Usiądźcie, pewnie jesteście zmęczeni po podróży. Zrobię Wam herbaty i zawołam Catrinę. 
Nastawił czajnik i zapukał do pokoju, gdzie mieszkał razem z żoną. Nie usłyszał proszę, więc cicho otworzył drzwi. Tak jak myślał Catrina spała na łóżku. Ciężko znosiła całą tą sytuację. Dużo spała i płakała. 
Usiadł na krawędzi łóżka i pogłaskał ją po ramieniu. Obudziała się i spojrzała na niego zaczerwienionymi oczyma. 
- Znowu zasnęłam, przepraszam- powiedziała. 
- Nie masz za co przepraszać. Obudziłem Cię, bo przyjechali Bob i Alberto. Nie mają dobrych wieści. 
Na jej twarzy pojawił się grymas bólu i na chwilę przymknęła oczy jakby chciała wrócić do snu, który jej przerwał. Może był weselszy niż rzeczywistość. 
- Mogę porozmawiać z nimi sam jeśli wolisz. 
Pokręciła głową. 
- Też muszę przy tym być – postanowiła. 
Pomógł jej się podnieść i trzymając się za ręce poszli do salonu. Opadli na sofy i wysłuchali smutnych wiadomości. 
- Zginęli w wypadku samochodowych jadąc na spotaknie z Oscarem- relacjonował Alberto. 
- Co?- spytała zaskoczona babcia. 
- Nie wiedzieliście o tym spotkaniu?- również był zaskoczony.  Oscar przysłał wiadomość, że musi się pilnie z nimi spotkać jeszcze przed podpisaniem umowy. Skontaktowali się z nami, żeby nas o tym poinformować. Mieliśmy im przecież towarzyszyć w podróży do Lupusu. Plan był taki, że oni jadą na spotkanie z Oscarem, gdzie my do nich dojeżdżamy i dalej podróżujemy razem. Czekaliśmy w umówionym miejscu, ale ich nie było. Oscara też. Przejechaliśmy trasę, którą mieli pokonać i znaleźliśmy ich samochód dachem do góry. Już nie żyli, gdy tam byliśmy. Bardzo nam przykro. 
Catrina się rozpłakała. Greg pozwolił sobie na łzy, choć zwykle zachowywał twarz, nie ważne w jak trudnych okolicznościach. Jego syn Dylan odziedziczył to po nim. 
Zaległa cisza. 
Nikt nie wiedział, że całej rozmowie przysłuchuje się Hope. Gdy tylko usłyszała jak ktoś wchodzi do domu, ukryła się za schodami i przysłuchiwała rozmowie. Nie wiele rozumiała z tej rozmowy, poza tym, że dorośli mają tajemnice, o których nic nie mówią dzieciom. 
- Gdzie zabraliście ciała?- spytał dziadek.
- Są w bagażniku. 
- Co my powiemy dzieciom?- załamała ręce babcia. 
- Prawdę, kochanie. Prawdę. 
- Ktoś zdradził- zauważył Bob. 
- Podejrzewamy Oscara. 
- Nie- pokręciła głową babcia. Znali się od dziecka i przyjaźnili. Nie zrobił by czegoś takiego. 
- Obawiam się, że zrobił. Wszystko wskazuje na niego. Mogli go zaszantażować, zagrozić. 
- Nie uwierzę dopóki nie powie mi tego w twarz- wstała gwałtownie. Był dla mnie jak dziecko. Napiszę do niego z prośbą o spotkanie. 
- To niebezpieczne. 
- Ja nie widzę innego rozwiązania- odparła zdecydowanie. 
Wróciła do swojego pokoju. A Greg dał gościom herbaty. 
- Porozmawiam z nią jeszcze, ale boję się, że będzie nieugięta. Traktowała Oscara jak własnego syna. Mieszkaliśmy dom w dom, a rodzice Oscara nie należeli do kochających swoje dzieci. Jego ojciec był gburowaty i dużo pił. Dochodziło do aktów przemocy. Czasami Oscar nocował u nas. Ciężko uwierzyć, że nas zdradził. 
- Rozumiemy- odparli chłopcy. 
- Zadźwonię po grabarza, trzeba zorganizować pogrzeb. 
                           
                                            *** 
Hope widziała tylko jedna czarną plamę przez zapłakane oczy. Czarne postacie, które zebrały się na pogrzeb jej rodziców. Znajomi, koledzy z pracy, sąsiedzi. Widziała dwie białe trumny. Wiedziała, że rodzice woleli by być w jednej trumnie jak najbliżej siebie, nawet po śmierci. Hope nie słuchała składanych kondolencji, przemówień, wyłączyła się na otaczający ją świat. Zastanawiała się, co rodzice czuli przed śmiercią. Czy umierając pomyśleli o niej? Czy byli smutni, że odchodzą z tego świata i zostawiają ją? Czy się bali? A może cieszyli, że spotkają się z Bogiem w niebie? Czy płakali? Czy ich bolało? Czy bolały ich sekrety, które zataili przed swoimi dziećmi? 
Żałowała, że pozwoliła im wyjechać. Może powinna mocniej naciskać. Przecież miała już kiedyś koszmary, że rodzice umierają. Mama przybiegła wtedy do niej i obiecywała, że to tylko złe sny, że nigdy jej nie zostawią. Dopiero kiedy będą bardzo, bardzo starzy, ale nie teraz kiedy była małą dziewczynką. Nie dotrzymali obietnicy. Tej i jeszcze wielu innych. Nigdy już nie spyta ich czy może mieć pieska. Nigdy nie usłyszy ich czułego głosu, gdy czytają jej bajkę czy tulą do snu. Nigdy jej już nie przytulą, nie pogładzą po głowie, nie poczuje pod palcami dotyku ich gładkiej skóry. Nie zaśmieją się już z jej żartu, nie odmówią wspólnie modlitwy. Nigdy już nie zobaczy jak się obejmują i jaka bije od nich miłość. Nigdy już żadnym zmysłem nie poczuje ich obecności. zostaną tylko w pamięci, wśród licznych wspomnień. Ale teraz oprócz nich będę też pytania. Kim tak naprawdę byli? Czego im nie mówili? 
Przyjrzała się ludziom zebranym na uroczystości pogrzebowej. Wszyscy w czarnych kostiumach. Kobiety ocierały łzy z policzków. Czy tak naprawdę wiedziały za kim płaczą? Za kim tęsknią? Przecież oni tak naprawdę nie znali jej rodziców! Po co tu przyszli. Nie powinni tu być. Wcale im nie przykro. Za chwile zapomną o tym, że już ich nigdy nie zobaczą. Nie wymienią paru słów. 
Oni nie wiedzieli, że mama lubi zjadać korzuchy z mleka, że tata nie umie wiązać krawatu. Nie znali ich ulubionych słów i cytatów. Znali tylko ich imiona i wygląd. Nie mogli wiedzieć jak cudownymi ludźmi byli. Nie wiedzieli jak bardzo ich kochała. Jaką pustkę powodowała ich strata. Jak nagle świat stał się mroczniejszy, ludzie brzydsi, jak zgasła jej szczęśliwa przyszłość. Oni tego nie rozumieją. 
Gdy ludzie się rozeszli i zostali tylko najbliżsi, Hope rzuciła samotną czerwoną różę na trumny. Stała przed ciałami swoich rodziców opakowanymi w pudełka. Ich dusze były już w niebie, czyli tu już ich nie było. Były tylko zmiany, które spowodowała ich śmierć. Hope się zmieniła. Świat nie był już ten sam. Wszystko stało się inne. Na zawsze.



                                 Rozdział 4
Dowiedział się z liściku:
                     
                    Dobra robota.
                                    Browni

Osłupiały wpatrywał się w kartkę, a ręce drżały mu coraz mocniej. Powtarzał w myślach ” To się nie wydarzyło. To nieprawda.”,ale dobrze wiedział, że się okłamuje. Zabił swoich przyjaciół. Zdradził i tym samym zabił, żeby chronić rodzinę. Ale zabił. 
Od tej chwili nie mógł nic przełknąć . Czuł się jak robot, jak ktoś pozbawiony uczuć. Jak potwór. Jedna część jego mózgu przekonywała go, że postąpił racjonalnie,a druga, że jak najgorszy człowiek. Zdawał sobie sprawę, że był między Scyllą a Harybdą, ale czy naprawdę? Czy nie mógł wymyślić innego rozwiązania. Na przykład, wysłać ten list do Ami i Dylana. A potem jeszcze drugi, o którym Browni by nie wiedział, by ich ostrzec. Ale jeśli oni nie przyjechaliby na spotkanie, to Browni zabiłby jego rodzinę, czy domyśliłby się, że działał za jego placami. Skąd miał to wszystko wiedzieć?! Jak miał dokonać wyboru między rodziną i przyjaciółmi. To niemożliwe. A jednak to zrobił. I do końca życia będzie żył z tą decyzją. A co z ich dziećmi- Cinną, Hope i Pou, czy Browni je oszczędził? Miał nadzieję, że tak, ale to tylko zachcianki głupca. 
Poszedł do łazienki, gdzie płakał dopóki nie wylał wszystkich łez, które miał. Potem zmusił się do pracy i wciagnął się na tyle, by przez chwilie zapomnieć o swoich zmartwieniach. 
Gdy wybiła szesnasta spakował teczkę i ruszył do wyjścia. Bał się powrotu do domu. Tego, że żona wszystko wyczyta mu z twarzy, że dowie się że oni nie żyją przez niego. Potem by na niego nakrzyczała, zwyzywała, napiętnowała. Już raz tak zrobiła, gdy był za mało ostrożny w załatwianiu pewnej sprawy dla Ruchu. Była tak przejęta całą organizacją, tak wierzyła w jej cel, że traktowała je wręcz jak własne dziecko. Dlatego, gdy zrobił coś nie tak w zwiazku z nią, przestawała być zwyczajową, dobrą Melani, a włączał jej się tryb chorobliwie opiekuńczej matki. Wtedy nie trakowała go jak męża, tylko jak intruza, zagrożenie. Może właśnie dlatego zbywała jego prośby, by postarali się o drugie dziecko, bo przelewała całe swoje uczucia w organizacje. To ich tak strasznie różniło, a w innych kwestiach byli zgodni. Dlatego teraz zamiast do domu, pojechał do pubu. Zamówił drinka i smętnie go sączył. Obsługiwała go ładna blondynka w wieku około dwudziestu kilku lat. Uśmiechała się do niego miło, wręcz z troską. 
- Ciężki dzień- zagadnęła.
- Nie pytaj- machnął ręką. Fatalny. 
- Kobieta pana rzuciła? 
Zaśmiał się. 
- Nie, ale z pewnością to zrobi. 
- To ona straci, nie pan- mrugnęła do niego filuternie. 
Spojrzał na nią zdziwiony. Czy ona właśnie z nim flirtowała? Przecież był od niej z piętnaści lat starszy, a ona właśnie zasugerowała, że jest przystojny. Polepszyło mu to humor. Ciekawie się jej przyjrzał. Była naprawdę ładna. Miała śliczne niebieskie oczy, amorkową twarz, spory biust i zgrabne nogi. „I co z tego pacanie, ty masz żonę”- skarciuł się w duchu. 
Barmanka podchwyciła jego wzrok i rozpromieniła się. Nachyliła się nad blatem, aż poczuł słodki zapach jej perfum. 
- Nie obchodzi mnie ile masz lat, jesteś zabójczo przystojny i mnie podniecasz. 
Jej bezpośredniość tak go uderzyła, że nie wiedział co odpowiedzieć. Więc głupio się w nią wpatrywał.
- Jakby co wiesz, gdzie mnie szukać- puściła mu oko i poszła obsłużyć innego klienta. 
Siedział tak przez chwile, a do jego głowy przychodziły różne myśli. Kiedy ostatnio kochał się z żoną? Jak się dowie, o tym co się stało, to z pewnością nie wpuści go do łóżka. A ta dziewczyna jest taka śliczna i chętna. Zawsze był wiernym mężem, nie zdradzał i nawet nigdy o tym nie myślał. Nie oglądał kolorowych pisemek z półnagimi kobietami, nie interesowały go inne kobiety poza Melani. Ale teraz jego sytuacja była beznadziejna, a wola słaba. Zanim się na coś zdecydował, opuścił lokal i wrócił do domu. 
Zastał Melani w salonie na kanapie. Piła wino z kieliszka- miała taki zwyczaj, gdy się czymś denerwowała. 
- Gdzie byłeś?- spytała bez wstępów. 
- Zatrzymali mnie w pracy- skłamał. 
- Podejdź tu. 
Zrobił jak kazała. 
- Czuję od Ciebie alkohol.
- Wracając wstąpiłem na drinka. 
- Czyli obiad z rodziną się nie liczy?
- Oczywiście, że się liczy.
Usiadł obok niej i objął ją ramieniem, ale ona gwałtownie wstała. 
- Gdyby się liczył, to byś na nim był. 
- Zawsze jestem na obiedzie, a gdy raz nie jestem robisz mi wymówki?- spytał z niedowierzaniem.
- Przepraszam, denerwuję się. Dziś Ami i Dylan mieli podpisać umowę. 
Oscara zmroziło. 
- Dlatego sam rozumiesz, jestem zestresowana. Chciałabym by wszystko poszło dobrze. To dla nas wielka szansa, żeby rozwinąć skrzydła. 
- Wiem, też bym chciał, żeby wypaliło- przytulił ją. 
- Mam nadzieję, że dadzą szybko znać jak dojadą. 
- Na pewno- uspokoił ją. 
- Obiad jest jeszcze ciepły, idź zjeść. 
Ale zamiast jeść tylko dziabdział w talerzu. Był cały spięty. Jedno nieuważne słowo i wszystko się wyda. Każde kolejne kłamstwo kładło się za nim cieniem. Zmusił się by zjeść choć trochę, a potem poszedł się położyć. Wstał dopiero nazajutrz. 

                                   ***
Minął tydzień. Melani była cały czas poddenerwowana, a więc drażliwa i niemiła. Dlatego dbał o to,  by przyjeżdżać na obiad punktualnie i być dla niej jak najmilszym. Konsekwentnie odpychał od siebie myśli o ładnej blondynce z baru. Nie miał na to głowy. Zresztą na nic nie miał. Właśnie dziś Browni zarządził, że zawiesza go służbie. Gdy spytał się dlaczego, odpowiedział mu Hopkins.
- Stary, to chyba oczywiste. Nadszedł czas by przyjrzeć się twojej osobie wnikliwie. Jesteś nam potrzebny, by obalić Ruch Oporu. To dopiero początek. 
Oscar nie wierzył właśnym uszom. Miał znowu zdradzić? Miał znowu przechodzić w domu piekło? Już wolał się powiesić, choć wiedział, że nia ma dość siły, by to zrobić. 
- A dom? Przeszukacie mi dom?
Hopkins spojrzał na niego jak na idotę i kiwnął głową. 
Załamał ręce. Prawda się wyda. Żona go znienawidzi i na pewno nie przystąpi do współpracy z Rządem. Nie zdradzi swojego dziecka. 
- Kiedy, kiedy przyjdziecie?- spytał.
- Jutro. Radzę uprzedzić o wszystkim żonę, nie chcemy mieć potem z nią problemów.

Wracał do domu jak zbity pies, choć to źle powiedziane. Zbity to zostanie w domu. Może nie fizycznie, ale z pewnością psychicznie. Zastanawiał się czy siebie nie oszczędzić i przyjść pijanym, może lepiej by wtedy to zniósł. Albo mógłby wstąpić do tej miłej blondynki. Ale po co się odprężać, jak za chwilię idzie się na skazanie. Nie, załatwi to na trzeźwo, a potem się upije, a później kto wie?
Przywitał się z żoną i synem, przywołując na twarz sztuczny uśmiech. Niewiele odzywał się w czasie obiadu. Gdy Brave opuścił kuchnie, Melani zaczęła.
- Nie odzywali się. Odchodzę od zmysłów. Czy oni nie mają krztyny przyzwoitości, by poinformować mnie czy wszystko jest w porządku? 
- Kochanie, właśnie o tym chciałem z tobą porozmawiać. 
- Wiesz coś? Masz wiadomość?- pytała podekscytowana. 
Przez chwile wahał się z odpowiedzią. Ostatecznie przyjął strategię, że będzie mówił bez ogródek. 
- Ami i Dylan nie żyją. 
- Co? Ale?
Jej reakcja była natychmiastowa. Z podeksytowania przeszła w rozpacz. Po policzkach pociekły jej łzy. 
- To niemożliwe. 
- To bardzo możliwe i pewne. Rząd nigdy się nie myli. Wiem od tygodnia. 
- Co ma do tego Rząd? Wiesz od tygodnia i nic mi nie mówisz! Potwór! Wiesz jak się martwiłam! A ty mnie okłamałeś!
Spojrzał na Melani. Na jej piękną buzię, teraz pełną gniewu. Drżące ręce, nienaganny ubiór. Żona idelana. 
Chciał zapamiętać moment, w którym ją traci. Chociaż może stracił ją dawno temu. I czy kiedyś ją kochał? Myślał, że się tego nauczył. Postarał się przelać wszystkie uczucia, które żywił do Ami na Melani. I początkowo mu się udawało. Był szczęśliwy. Aż dostrzegł rażące różnice między tymi dwiema kobietami. Może to był koniec?
Ponownie spojrzał na Melani. 
- Rząd to zaplanował, a ja mu pomogłem. Wysłałem list do Ami i Dylana z prośbą o spotkanie przed obradami 23 kwietnia. W czasie dojazdu na spotkanie mieli upozorowany wypadek. Zginęli. Chcę, żebyś wiedziała, że nie miałem wyboru. Powiedzieli, że albo to zrobię albo zabiją Ciebie i naszego syna. 
Patrzyła na niego z mieszanką odrazy i gniewu. 
- Trzeba była zabić mnie! Oni mieli większą wartość! Oni mogli zmienić świat! Jesteś obrzydliwym zdrajcą! Nigdy nie wierzyłeś w naszą misję!
- Wierzyłem! Właśnie dlatego prowadziłem podwójne życie. I po cholerę mi to było, co? Żebyś mnie teraz obwiniała! To opanowałaś do perfekcji. Nasze życie jest jednym wielkim kłamstwem. Czy mnie w ogóle kiedyś kochałaś, czy dalej durzyłaś się w Dylanie?- spytał. 
Chciał by ta kłótnia wyjawiła mu odpowiedź na wszystkie pytania, które kiedykolwiek chciał jej zadać. 
Nie odpowiedziała.
- Szczęściarz z niego. Mieć taką żonę jak Ami. Byłby idiotą jak ja, żeby poślubić taką żmiję.
Uderzyła go w policzek i jeszcze raz, i jeszcze raz. Aż złapał ją za nadgarstki. 
- Zdradzałeś mnie z nią?- spytała.
Zaśmiał się.
- Oczywiście, że nie. Ami to prawdziwa żona i nie zdradza męża. Ja też Cię nidy nie zdradziłem aż do teraz. Od teraz nie ma nas, zresztą chyba od dawna nie ma. 
Puścił jej nadgarstki i skierował się do wyjścia.
- Nigdy Ci nie wybaczę! Smaż się w piekle, ty zdrajco! I wiesz czemu nie dałam Ci kolejnego dziecka, bo nie chciałam by miało takiego ojca niedorajdę. Żaden z Ciebie facet! 
- Żadna z Ciebie kobieta!- odkrzyknął. A i jutro pójdą przeszukać nasze rzeczy, więc bądź grzeczna. 
Jadąc do baru płakał. Ostatnie słowa, go zabolały. Dokładnie tak samo mówił do niego ojciec „niedorajda”, „imbecyl”, „ mamisynek”. Malani o tym wiedziała, dlatego wykorzystała to przeciwko niemu. Otarł łzy i wszedł do baru. Ładna blondynka od razu się do niego uśmiechnęła. 
- Hej! Co podać?
- Chce się upić.
- Czyli już Cię kobieta rzuciła?
Kiwnął głową.
- Dam Ci całą butelkę, a potem- dodała szeptem -zapraszam do schowka.
Jakiś czas później, gdy był już nieźle wstawiony poprowadziła go do ciasnego pomieszczenia na tyłach lokalu.
- Co teraz?- spytał. 
- Co tylko chcesz.
- Ja nigdy ...
- Tak byłeś wierny jak pies. Ale to już koniec. Ta wredna suka już Cię nie pilnuje. Zabaw się. Chcę widzieć jak się bawisz. 
Pocałował ją. A potem wszystko działo się szybko. Zdjął z niej ubrania, całował jej piękne ciało, a ona jego. A potem byli jednością. Wolno, szybko. Wydawane przez nich dźwięki były dla niego muzyką. Poczuł się jak pierwotny mężczyzna. Dziki, pełen żądź, bez żadnych ograniczeń. 
Gdy skończyli byli cali spoceni. 
- Jesteś wspaniały.
- Nigdy nie robiłem tego w taki sposób.
- Jesteś dziki. Mój kocurek- musnęła go w ucho. Musze wracać do pracy. Ale jutro możemy powtórzyć naszą dziką przygodę. 
- A nie możemy? – zawachał się.
Zaśmiała się i ściągnęła z siebie ledwo co założone ubrania. I kochali się znowu i znowu. I wtedy wydawało mu się, że to najpiękniejsza noc w jego życiu. Lepsza niż noc poślubna. 



                         ***
Wrócił do domu około piątej i legł na podłogę w salonie.  Obudziły go hałasy. Ktoś trącił go butem.
- No co my tu mamy. 
To był głos Hopkinsa. 
Oscar zaczął się podnosić i poczuł potworny ból głowy, a  potem zwymiotował. I znowu zasnął. Ponownie obudził się na kanapie. Przetarł oczy. Jacyś ludzie kręcili się po domu. 
- To się upiłeś, pierwszy raz w życiu. Masz to aspiryna, pomoże.
Posłusznie wziął tabletkę i popił ją wodą. 
- Która godzina?
- Wpół do komina, dla takich smarków nie ma zegarków- odezwał się jeden z ludzi, grzebiących mu po szafkach.
- Zamknij się Alfred. Wybacz, ma bachora w domu i nauczył się wszystkich możliwych rymowanek na świecie. Jezu, oby mnie nigdy nic takiego nie spotkało. Starczy, że mam siostrzenica. Szkoda gadać. Jakbyś się doprowadził do ładu, to mógłbyś nam pomóc i szybciej byśmy skończyli. 
- Gdzie Melani i mój syn?
- Ona zabarykadowała się w sypialni, a syn w domu. Widzę, że z Ciebie mały rozrabiaka- wskazał na jego szyję.
- Co tam jest?
- Malinki. 
- O - zmieszał się Oscar. 
- Nowe życie pełną parą. Jak chcesz to zaznajomię Cię z każdym klubem, w którym mają ładne i chętne panie. A jak wolisz płacić, to znam też różne agencje. 
- Dzięki, ale ja...- nie wiedział, co odpowiedzieć na taką propozycję. 
- Łapie, za szybko. Jakby co, wiesz gdzie mnie szukać - klepnął go przyjacielsko po ramieniu. A teraz idź do łazienki. Jezu, cuchnie od Ciebie jak od bezdomnego. 
Ogarnął się w łazience i pomógł chłopakom przetrząsać swój dom. Ku jego zdziwnieniu miło upłynął mu czas wśród męskich rozmów i żarcików. W sumie od kiedy rozpoczął swoje podwójne życie, odsunął się towarzystko od życia. A teraz miał kolegów. I w zasadzie znajdowali się po tej samej stronie barykady, choć Oscar z przymusu. Skoro zaczął nowe życie to nowi koledzy też się przydadzą. Dlatego przystał na ich propozycję wspólnego wypadu na striptiz wieczorem. Miał poczucie, że odbija się do dna, chociaż nie wiedzial co w tym wypadku jest dnem. 




Rozdział 5
Świat nie może być taki sam, gdy nie ma na nim ludzi, których najmocniej kochamy. Hope to czuła- myjąc zęby, jedząc śniadanie, zasypiając. Najprostsze czynności uświadamiały jej stratę rodziców. Brak ich ciepła, głosu, bicia serca. Dwa puste miejsca przy stole. Co dalej? – pytała siebie. Jak ma wyglądać moje życie? Za rok pójdzie do szkoły, zacznie się uczyć. Ale nie będzie mogła pochwalić się swoimi osiągnięciami przed rodzicami, bo ich nie ma. Więc po co tam w ogóle iść, po co się starać, skoro oni tego nie zobaczą? Czytać i pisać już umiała, tata ją nauczył. Była taka szcześliwa, gdy rodzice z dumą patrzyli jak składa wyrazy i je czyta. Jak z koślawych literek powstają słowa. To wspomnienie jest wciąż tak świeże i bolesne. 
Dwóch młodych mężczyzn, którzy przyjechali poinformować jej dziadków o śmierci rodziców dalej przebywa w ich domu. Dziadkowie mówią, że to przyjaciele Ami i Dylana, że dzięki nim są bezpieczni. Ale przed czym musimy się chronić?- zastanawia się Hope. Patrzy teraz na dziadków inaczej niż kiedyś. Wcześniej widziała w nich przyjaciół, a teraz widzi ludzi, którzy czegoś jej nie mówią. Pewnego wieczoru nie wytrzymała i zaczęła na nich krzyczeć podczas kolacji. 
- Kiedy powiecie nam prawdę?! Kiedy dowiemy się przed czym oni nas ochraniają - wskazała na Alberto i Boba. Słyszałam waszą rozmowę, gdy oni przyjechali. Kim jest Oscar? To nie był zwykły wypadek? Chcę odpowiedzi! Rodzice nas okłamywali i wy też!
Greg i Catrina nie byli przygotowani na taki wybuch pretensji i żalu. 
- Kochanie- zaczęła babcia. Wszystko Wam wytłumaczymy w swoim czasie. Poznacie odpowiedzi, ale teraz jest za wcześnie. Jesteście za młodzi. 
- Nie byłam za młoda, żeby stracić ojca i matkę, to nie jestem za młoda, żeby usłyszeć dlaczego ich straciłam- rzuciła ostro i wybiegła z pokoju. 
Catrina wtuliła się w pierś męża. Tego było dla niej za dużo. Poza wyniszczającym uczuciem straty i smutku, musiała jeszcze znosić wyrzuty dzieci. Choć to Cinna była najstarsza z całej trójki, to Hope zadawała najwięcej pytań, choć była jeszcze za mała na poznanie prawdy. Takiego zdania była Catrina. Greg widział w swojej wnuczce ogromną siłę, dzięki której zdoła znieść wszelkie przeciwności losu i każde słowa. Dlatego uważał, że zasługuje na wyjaśnienia. Ostatecznie zebrali dzieci w salonie i wszystko im opowiedzieli. O tajnej organizacji, którą założyli rodzice. O niebezpieczeństwie, które za tym szło. O Oscarze, który najprawdopodobniej ich zdradził. I o tym, że są teraz na celowniku Rządu i każde z nich jest w niebezpieczeństwie. Z tego powodu nazajutrz mieli opuścić miasto i wyjechać do Castalony, a Alberto i Bob mieli zamieszkać z nimi na stałe. Hope była zadowolona, że jej prośby zostały wysłuchane i otrzymała odpowiedzi. Wróciła do pokoju. Głowa pękała jej od natłoku nowych widomości. Nim położyła się do łóżka, wzięła do ręki fotografię leżącą na stoliku nocnym. Przedstawiała ją i rodziców. Na ich twarzach gościły szerokie uśmiechy. Zdjęcie wykonano zaledwie parę miesięcy temu w jej szóste urodziny. Teraz ta data wydała się taka odległa. Próbowała dostrzec w twarzach Ami i Dylana coś, co zdradzało by ich drugie życie. Ale nic takiego nie było. Ich uśmiechy były prawdziwe, ich miłość szczera. To byli oni, Ci sami których kochała i za którymi tęskniła, a jednak trochę inni. Rozumiała, że nie mogli jej wcześniej powiedzieć o swojej działalności, nie zrozumiałaby tego. Ale czy kiedykolwiek zamierzali to zrobić? Cinna miała już jedenaście lat, a nic nie wiedziała. To znaczy, że nie zamierzali im powiedzieć, dopóki nie będzie to konieczne. Dopóki nie będą im potrzebni. Była na nich wściekła, dlatego rzuciła zdjęcie na ziemię. Rozległ się trzask łamanego szkła. Nie poczuła się przez to lepiej. Przez nich nie wiedziała już jak powinna się czuć. Zabrali jej stabilność, szczęście. Oszukiwali, narażali na niebezpieczeństwo, a równocześnie kochali i tak mocno za nimi tęskniła. Rozdarta między miłością a nienawiścią,  zasnęła. Była tylko jednego pewna, że jej serce potłukło się na tysiąc kawałków i krwawiło. 

                                        ***
Dom w Castalonie był ładny i po paru tygodniach wszyscy się w nim zadomowili. Hope poznała dzieci zza płotu i miała się z kim bawić. Po roku poszła do pierwszej klasy podstawówki, zaczęła obcować z rówieśnikami. Starała się nie myśleć o rodzicach, Ruchu Oporu i swoich uczuciach względem nich. Ale i tak to robiła. Poznając nowych ludzi, czuła się zagubiona i wycofana. Nosiła w sobie ciężar i wiedziała, że nikt go nie zrozumie. Starała się zachowywać jak jej otoczenie i nie myśleć za dojrzale, choć ku temu zmierzał jej umysł. 
Im starsza była, tym bardziej rozumiała, czym zajmowali się Ami i Dylan. Co nie znaczy, że im wybaczyła. Nie potrafiła tego zrobić. Choć czasem czuła się nawet z nich dumna, ale częściej była po prostu zła i smutna. Ciężko żyje się mając w sobie ambiwalentne uczucia, które wciąż się ścierają. Ale tak właśnie wyglądało jej nowe życie. Nieustatna walka ze swoimi uczuciami i koszmarami, które ją nawiedzały. 

 


.
Rozdział 6
Oskar przyszedł do swojego biura i zabrał się za czytanie tony papierków, które musiał podpisać albo odrzucić. Kiedyś minister spraw zagranicznycznych, teraz pracownik kancelarii prezydenckiej. Taką cenę poniósł za rebelię, w której uczestniczył. Starał się nie wracać myślami do tamtych dni, gdy miał inne poglądy i żył inaczej. Liczyło się tylko tu i teraz. Wyznawał tą zasadę cztery lata i wiedział, że będzie się jej trzymać do końca życia. Po chwilii wsiąkł w czytane papiery. Po dwóch godzinach monotonnej pracy do drzwi jego biura zapukał Hopkins. Usiadł na stole, jedząc jabłko. 
- Nie nudzi Ci się ta praca?- spytał, głośno ciamkając.
- Przyzwyczaiłem się, a ty lubisz być chłopcem na posyłki?- odbił piłeczkę. 
- Chociaż nie siedzę za biurkiem, tylko się ruszam. Gdybyś o siebie nie dbał pewnie miałbyś teraz wielki brzuchal. 
- Mam dużo pracy. O co chodzi?
- Mam Cię zaprowadzić na komisariat. 
Oskara przeszły dreszcze. Ostatnio był tam cztery lata temu i okoliczności nie należały do najprzyjemniejszych. 
- W celu?- doptywał się. 
- Masz odwiedzić starych znajomych.
- Powiesz mi dokładnie o kogo chodzi czy będziemy bawić się w    „Zgadnij kto to” ?
Szeroki uśmiech na twarzy Hopkinsa wystarczył za odpowiedź.
Podjechali samochodem pod komisariat, prowadząc luźną rozmowę o życiach innych ludzi i kobietach, które ostatnio gościły w ich łóżkach. Oskar nie nazwałby Hopkinsa przyjacielem, ale prawdą było, że od kiedy jego życie bezpowrotnie się zmieniło, Hopkins był jego ważnym elementem. Zapoznał go z wieloma ludźmi, pomógł zaaklimatyzować się w nowym środowisku, nauczył jak zachowywać się wśród innych. Przez to wszystko Oskar czuł, że jego życie jest kompletne i niczego mu nie brakuje. 
Weszli do szarego monumentalnego budynku zbudowanego w modernistycznym stylu. Hopkins podszedł do policjanta i przez chwile z nim rozmawiał, wskazując przy tym na Oskara. Policjant poprowadził ich na drugie piętro, dalej długim korytarzem z licznymi drzwiami po bokach, aż zatrzymał się przy jednych z nich. 
- Powiesz mi w końcu z kim się spotkam?- spytał lekko poddenerwowany Oskar. 
- Oczywiście, z Catriną i Gregiem Coutcherami. Pewnie się za sobą stęskniliście. 
Oskarowi zrzedła mina. Ręce zaczęły mu się pocić. Myślał, że kwestie dotyczące Ruchu Oporu ma już dawno za sobą. Po zdradzie trochę jeszcze pomagał Rządowi przeciwko organizacji, ale na szczęście trwało to krótko. A po tym nic. Ani słychu, ani widu. Jakby nigdy nie brał w tym udziału. Cieszył się z tego obrotu spraw. Dzięki temu jego nowe życie nie miało rys z przeszłości. Aż do teraz. Drzwi się przed nim otworzyły i wszedł do środka. 
Na środku małego kwadratowego pokoju o szarych ścianach stał stół, do którego przykuci za nadgarstki, siedzieli zatrzymani. Podnieśli na niego wzrok i patrzyli w osłupieniu jak siada na przeciwko nich. Drzwi się zatrzasnęły, a on starał się nie okazywać jak zdenerwowany jest. Wiedział, że jedna ze ścian jest lustrem weneckim i ktoś za nim stoi i obserwuje każdy jego ruch.  W górnym roku pokoju dodatkowo dostrzegł kamerę. Rozpiął guziki marynarki i zagarnął poły do tyłu. Założył nogę na nogę i wyciągnął z kieszeni paczkę papierosów oraz zapalniczkę. Musiał zachowywać się nonszalancko, bez okazywania żadnego przywiązania do ludzi, którzy byli dla niego kiedyś jak rodzina. Odpalił papierosa i wyciągnął w ich stronę paczkę.
- Chcecie?- spytał.
Oboje pokręcili głową. Wygladali starzej niż zapamiętał. O kilka więcej zmarszczek na twarzy, więcej siwych włosów i zmęczenie ukryte w oczach. No coż, każdy płaci jakąś cenę za swoje decyzje. 
- Od kiedy palisz?- spytała Catrina, wbijając w niego nieprzychylny wzrok. 
- Od czterech lat- odpowiedział zgodnie z prawdą. 
Pokręciła głową z dezaprobatą. 
- Twoi przyjaciele zginęli przez Ciebie, a ty zacząłeś palić?- spytała z niedowierzaniem jakby te kwestie nie mogły istnieć razem. 
- Będziesz mi robić wyrzuty za coś takiego? Wiem, że nie lubisz dymu
papierosowego, ale nie spodziewałem się, że właśnie od tego zaczniemy rozmowę. 
- Dlaczego Cię przysłali? Co masz za zadanie z nas wyciągnąć?- spytał Greg, ignorując poprzedni tor rozmowy. 
- Nic. To zwykła pogawędka. 
- Nie mam ochoty na pogawędki ze zdrajcami. 
- Wreszcie to z siebie wydusiłeś. „Zdrajca”. Tobie łatwiej przyszło się z tym pogodzić niż twojej żonie. Nie przepadałeś za mną od początku. 
-  Masz rację. Uważałem, że jesteś za mało ostrożny, zbyt naiwny, by pełnić funkcję wtyki w szeregach wroga - przyznał Greg. 
- Zawiodłam się na tobie- popatrzyła na niego ze smutkiem i złością Catrina. - Ufałam Ci, wspierałam. Przychodziłeś do mojego domu, przyjaźniłeś się z moim synem-  z każdym kolejnym słowem wylewało się z niej coraz więcej goryczy i złości. A potem doprowadziłeś do jego śmierci i mojej synowej. Jak mogłeś? Jak Ci nie wstyd?
Podniosła się i od razu opadła z powrotem na krzesło. 
Odżywało w nim poczucie winy, ale nie mógł sobie pozwolić by to zobaczyli. Dlatego wziął do ust papierosa i leniwie wydmuchał ustami dym jakby był kompletnie wyluzowany. 
- Trzeba było ufać mężowi, nie mnie. 
- Powiedz dlaczego? Dlaczego złamałeś obietnicę? Pamiętasz ją? „Przywódców swoich nie zdradzę, oddam za nich życie jeśli zajdzie taka potrzeba”. 
- Chciałaś mojej śmierci?- spytał z ironią.
- Co? Nie. Ale zoobowiązałeś się do czegoś. 
- Kiedy na szali jest życie moje i mojej rodziny, wszystko inne nie ma znaczenia .
- Przyrzekłeś – naciskała. 
- Wolałabyś, żeby oni żyli, bo byli lepszymi ludźmi?
- Przecież wiesz, że nie o to chodzi - uniosła się. 
- Oczywiście. Cel uświęca środki. 
- To nie ma sensu kochanie – zwrócił się Greg do Catriny, nim ta zdążyła odpowiedzieć coś na słowa Oskara.  -Ma tak wielkie wyrzuty sumienia, że chce nas też wpędzić w irracjonalne poczucie winy. Jak możesz ze sobą żyć? Jak Melani może?
- To ja podjąłem decyzję, nie ona. Wtedy pewnie jak byłbym martwy, a nie Ami i Dylan.
- Na to się pisałeś. Nikt Cię nie zmuszał. 
Kiwnął głową, bo nie miał na to stwierdzenie lepszej odpowiedzi. 
- Z nią też przyjdzie nam rozmawiać?
- Nie, nie żyje. Popełniła samobójstwo. 
Powiedział to głosem wypranym z emocji. Na twarzy Catriny i Grega malował się szok. Zamrugała kilka razy i schowała twarz w dłoniach przykutych do stołu. 
- Kiedy?- uniosła zapłakanął twarz ku niemu i spytała słabym głosem.
- Trzy i pół roku temu. 
- Jak?
- Strzeliła sobie w usta. 
Catrina pokręciła głową z niedowierzaniem, niezdolna nic na to odpowiedzieć.
- Przykro nam - powiedziała za nią Greg.
- Proszę przestań - poruszył się nerwowo na krześle, nie lubił o tym rozmawiać. 
- Co z twoim synem? 
- Posyłam go w tym roku do wojska.
- Jest za młody. 
- Ze względu na wiadome okoliczności zgodzili się go przyjąć wcześniej. 
-  Naprawdę chcesz to zrobić? Stanie się mordercą bez skorupułów – włączyła się do rozmowy Catrina.
- To mój syn, nie twój. Potrzebuje dyscypliny i tam się jej nauczy. 
- Kolejny posłuszny robot waszej armii- wypluła te słowa z odrazą. 
- Masz coś jeszcze do powiedzenia w kwestiach wychowawczych mojego syna?
Pokręciła głową. 
- To świetnie, bo czas kończyć spotkanie. 
Spojrzał na zegarek. Minęło dziesięć minut odkąd tu przyszedł.  Podniósł się i skierował w stronę drzwi. 
- Wiem, że jest w tobie cząstka, która dalej wierzy w naszą sprawę, która ma marzenie lepszego świata i do tej cząstki mówię. Proszę, zadbaj by zostawili naszych wnuków w spokoju. Niech dadzą im spokój, dopóki nie zaczną czynnie działać w Ruchu, o ile zaczną. Zasługują na normalne dzieciństwo. 
- Ono nigdy nie będzie normalne, przecież wiesz - odpowiedział, nie odwracając się do niej. - Tak postanowili ich rodzice i ja nie mogę z tym nic zrobić- wypowiadał to ze smutkiem i dodał - Jestem teraz waszym wrogiem i zawsze nim będę. Nie proście mniej więcej o pomoc, bo Wam jej nie udzielę. 
Wyszedł z pomieszczenia i ruszył korytarzem. Spotkał po drodze Hopkinsa jak zwykle roześmianego.
- I jak rozmowa?
- Po co to było, co?- rzucił wściekle. 
- Spokojnie, nie wściekaj się na mnie. Ja tylko wykonuję polecenia z góry. 
-Nie mam zamiaru więcej z nimi rozmawiać. Sprawdzacie mnie czy myślicie, że przy mnie wyjawią jakieś tajemnice? Uspokoje Was, oni mnie nienawidzą, a ja jestem po Waszej  stronie.
- Ja Ci wierzę stary, naprawdę, ale oni może nie. Mówią, że gdy ktoś zdradzi zawsze pozostaje zdrajcą.
 
                                       ***
- Kiedy ich przymknęli?- spytał Oscar Hopkinsa, gdy byli na basenie w jazzuzi. 
- Myślałem, że nie chcesz o tym gadać?- spytał, unosząc brwi.
- Proszę, powiedz mi. 
- Okej, ale nie rozpowszechniaj tych informacji, jasne?
- Oczywiście. 
Niby komu miałbym o tym powiedzieć - pomyślał w duchu Oskar.
- Wyśledzili ich. Nie znam szczczegółów. Zagarnęli ich spod domu, znaczy tylko dziadków, dziaciaki z dwoma facetami odjechali.
- Jakimi facetami?
- Pewnie z tej organizacji, ale nie dopytywałem. 
- Zbiegli?
- Tak, jednego z facetów postrzelili, reszta odjechała samochodem. 
- Ścigali ich?
- Nie wiem. Nie posiadam za wiele informacji, pewnie wiedzieli, że będziesz mnie wypytywać. 
Oskar westchnął. Chciałby mieć pewność, że dzieciakom nic się nie stało. Chciałby, by Browni dał im na razie święty spokój. Nie mógł w tym pomóc, nie miał żadnej władzy. Mógł tylko mieć nadzieję. 
Rozejrzał się po sali z dwoma basenami, jazzuzi i sauną. Zatrzymał wzrok na atrakcyjnej blondynce w dwuczęściowym czerwonym stroju. Nie zauważyła, że się w nią wpatruje. 
- Znasz ją?- zwrócił się do Hopkinsa, który do głowy był zanurzony w wodzie. 
- Kogo?
- Tamtą laskę? Blondyna w czerownym stroju. 
- Taa. To Amanda, nowa sekretarka prezydenta.
- Serio? Nie wiedziałem, że zwolnił poprzednią. 
- Też byłem zaskoczony. 
- Rozmawiałeś z nią? 
- Nie miałem jeszcze okazji. A co, masz co do niej plany?- uniósł kącik ust w figlarnym uśmiechu. 
- Może - odpowiedział niepewnie. 
- Pracujesz w kancelarii, może będziesz miał jakąś sposobność do niej zagadać, a może chodzi do tych samych pubów co my? Popytam się o nią wśród znajomych. 
- Dzięki. 
- Zawsze do usług - wyszczerzył się w uśmiechu.

                                ***
- Gdzie mnie prowadzisz?- spytała Oscar Hopkinsa, gdy szli korytarzami budynku Federalnego Biura Bezpieczeństwa. 
- Dobrze wiesz gdzie, tylko trudno Ci to przed sobą przyznać- spojrzał na niego Hopkins i Oskar dostrzegł w jego oczach cień współczucia, ale to szybko znikło. 
- Nie - pokręcił stanowczo głową Oskar i zatrzymał się w miejscu. – Nigdzie nie pójdę. Nie do nich. Jestem im niepotrzebny. 
- Twierdzą inaczej. Chyba nie będziesz się kłócić z FBB?- skarcił go. 
- Błagam Cię, nie mogę tam pójść. To okrutne dla nich i dla mnie. Nie każ mi tego robić. 
- Słuchaj mnie uważnie – podszedł kilka kroków i stanął pare centymetrów przed Oskarem, przyjmując poważną i wrogą minę.      – Nie masz tu nic do gadania. Jeśli nie wykonasz moich poleceń, będę zmuszony zgłosić to górze i to nie skończy się dla Ciebie dobrze. Mamy materiały, by posadzić Cię do więzienia na dłuugie lata. Wiesz o czym mówię? Były polityk w więzieniu, to na pewno nie spodoba się twoim współwięźniom. Strażnicy nie będą interweniować, kiedy koledzy będą Ci wchodzić w dupę. Załapałeś?
Oskar przełkął ślinę i pokiwał głową. 
- Świetnie. Czas nagli.
Szli dalej, a przed Oskarem malowała się wyraźnie scena, którą przedstawił mu Hopkins. Wzdrygnął się i próbował o tym nie myśleć. Nic takiego się nie wydarzy, bo on Oskar Johnson od tej pory będzie posłuszny. Zrobi wszystko co mu każą, bez jęczenia i sprzeciwów. Poniesie konsekwencje tego co zrobił cztery lata wczęśniej, i tego co działo się jeszcze przed tym. Zapracował sobie na taki żywot. Wmawiał to sobie, aż stanął przed drzwiami, za którymi miał się spotkać z Gregiem i Catriną. Zaczął się pocić, ręce lekko mu drżały. Nie chciał tu być. Miał zamiar ponownie protestować, po mimo wszystkiego co sobie przed chwilą postanowił, ale jedno upominające spojrzenie wystarczyło, by porzucił swój zamiar. Nim tam wszedł, sprawdził czy paczka papierosów tkwi w swoim stałym miejscu- prawej kieszeni marynarki. Była. Był gotowy, choć to jak być gotowym na skok ze spadochronu. Sprzęt jest sprawdzony, ale nie wiadomo jak będzie , gdy już się skoczy. 
Wszedł.
Przez następny tydzień codziennie odwiedział salę w Federalnym Biurze Bezpieczeństwa. Spowodowało to : po pierwsze przywołanie mor z przeszłości, które nawiedzały go od teraz w nocy. Po drugie, wzrost dostarczonego do płuc tytoniu. Po trzecie – rosnącą niechęć do Hopkinsa. To wszystko złożyło się na jeden z najgorszych od trzech lat tygodni jego życia. Ponadto, syn chciał się z nim spotkać. Nie miał na to kompletnie siły. Dawno nie występował w roli ojca i jakoś nie miał ochoty ponownie się w nią wcielać. 
Umówili się w knajpce niedaleko miejsca, gdzie mieszkał Oscar. Gdy dotarł do lokalu, syn już tam czekał. Na powitanie Brave go przytulił. Oskar stał osłupiały nie wiedząc co powinnien zrobić z rękami. Ostatecznie poklepał syna po plecach i odsunął go na długość ręki, by uważnie mu się przyjrzeć. Wyrósł odkąd widział go po raz ostatni, czyli trzy i pół roku temu. Był wyższy, muskulatura zaczynała przypominać męską. Niegdyś dziecięca twarzyczka, teraz rysy odziedziczone po matce. Te same oczy o głebokiej granatowej barwie. Z całą pewnością, mógłby stwierdzić, że tylko brwi świadczyły za tym, że jest jego ojcem. Poza tym nie byli w ogóle podobni. Brave założył elegancką koszulę i spodnie w kant, na krześle obok wisiała marynarka. Nagroda pocieszenia, pomyślał z rozrzewnieniem Oskar. Strój w tym samym stylu. 
Nie chodziło o to, że nie chciał widzieć w Brave’ie Melani. Nie było sposobu by to wykasować i nie było potrzeby. Nie czuł się nieswojo albo dziwnie patrząc na kogoś tak ją przypominającego. Nie z tym miał problem. Żałował, że Brave nie ma z nim więcej cech wspólnych. Miał na myśl głównie wygląd, charakter był w tym momenecie kwestią drugorzędną. A będąc zupełnie szczerym, Oskar nie chciał by Brave wdał się w tatusia. Uważał się za człowieka pokonanego, złamanego i słabego. Nikt nie życzył czegoś takiego własnemu  dziecku. Ale aparycja? Miło usłyszeć tekst  w stylu „ Ale jesteście podobni, widać, że to twój syn”. Czy ktoś powiedziałby coś takiego patrząc na nich dwóch? Szczerze w to wątpił. Nie mógł uwierzyć, że to było dla niego tak ważne. Unikał kontaktu z synem przez trzy lata, a gdy wreszcie go widzi, jego największym zmartwieniem jest brak podobieństwa między nimi? Śmieszne czy tak boleśnie prawdziwe? Może oba. 
Usiedli, mierząc się spojrzeniami dobrą chwilę, zanim Oskar rzucił z uznaniem:
- Wyglądasz inaczej.
- Tak. Człowiek się zmienia przez ponad trzy lata - zauważył Brave z przekąsem. 
Oskar przymknął oczy i potarł ręką czoło. Nastawił się mentalnie na to, że nie będzie łatwo. Ale chyba nic nie jest w stanie przygotować na takie rozmowy.
- Naprawdę chcesz by tak wyglądało to spotkanie? Myślałem, że miło spędzimy czas- próbował ratować sytuacje , była jeszcze szansa na lżejszy i przyjemniejszy przebieg tego spotkania.
- Tato, nie widziałeś mnie od trzech lat i uważasz, że to w porządku? Brave zmarszczył szerokie brwi z niedowierzaniem w oczach.
- Wiem, to kupa czasu. Ale byłem zajęty. 
Łganie. 
- W święta? Byłem jedynym, który zostawał w tym okresie w internacie. To uwłaczające i smutne. Nie sądzisz?- jego ton był ostry.
- Dostałeś prezenty, rozmawialiśmy przez telefon - próbował się bronić, choć wiedział że to marne argumenty. 
- To nie to samo, co spędzić święta z ojcem. Nie zastąpisz siebie pocztówkami i najdroższymi rzeczami. 
- Wiem. 
Pokiwał głową z powagą. 
- Nie wiesz!- krzyknął, mając łzy w oczach. -To nie ty czułeś się jak sierota przez te lata!
- Ciszej, bo ludzie zacznął się gapić. 
- Mam to gdzieś - założył ręce na piersi.
- Ale ja nie. 
- Czy zdecydowali się państwo na coś ?- jak spod ziemi wyrosła przed nimi kelnerka. 
Oskar posłał jej dziękujące/dziękczynne spojrzenie, uśmiechnęła się w odpowiedzi. 
- Co chcesz synu?- zwrócił się do Brave’a, który dalej miał grymas na twarzy. 
- Nie wiem.
- Co pani proponuje?
- Pomidorową z kluseczkami, a na drugie kurczak z ziemniakami i surówką. 
- Poproszę razy dwa. Można wziąć na wynos?
- Jasne. 
Kelnerka zanotowała coś szybko w notesiku i poinformowała, że zamówienie powinno być za pół godziny. Przez ten czas ojciec z synem ze sobą nie rozmawiali. Brave spoglądał przez okno, a Oskar na klientelę lokalu. Swoimi strojami nie wpasowywali się w klimat knajpki. Wszyscy byli ubraniu luźno, codziennie. Oskar był tu pierwszy raz, zwykle jadał obiady bliżej miejsca pracy. Ale nie tylko dlatego odstawali. Ludzie byli tu uśmiechnięci, beztroscy, oni nie.
Odebrali zamówienie i wyszli. 
- Gdzie mnie zabierasz?- spytał Brave. 
- Do mojego mieszkania.
- Chcesz mi zaimponować?- prychnął.
- Nie. Nie chcę, żebyś się na mnie darł w miejscu publicznym.
Brave kiwnął głową i nic nie odpowiedział. Wjechali windą na przedostatnie piętro wieżowca. Oskar przekręcił zamek i otworzył drzwi do swojego świata. Brave był pod wrażeniem. Przestronne mieszkanie z pięknym minimalistycznym wnętrzem. Królowały biel i czerń. Wielki salon z kanapą i płaskim telewizorem, otwierał się na niewielką kuchnię z wyspą. W głębi sypialnia z rozsuwanymi szafami i łazienka o mlecznobiałych kafelkach. 
- Wow. Wypas - skomentował Brave. - To przede mną ukrywałeś? Drogi apartamentowiec?- odwrócił się do ojca. 
- Niczego nie ukrywałem. Po prostu...- wzruszył ramionami, nie kończąc zdania. 
- Nie chciałeś mnie tu, hę? Skaza w twoim idelanym życiu – dodał z goryczą. 
- To nie prawda.
- W takim razie wytłumacz mi, dlaczego mnie nie chcesz?!- prawie krzyknął. 
Oskar zamrugał parę razy, wypuścił i zassał powietrze ustami, głowiąc się nad dobrą odpowiedzią na to pytanie. Ale takiej nie było. A Brave czekał. Z zaciśniętymi pięściami, z żalem i wściekłością w oczach.
Jak wytłumaczyć coś, czego samemu się w pełni nie rozumie? Zawsze chciał być ojcem. Starał się spełniać ten przywilej i obowiązek jak najlepiej. Wszystko szło świetnie do momentu zdrady, a potem już nic nie było takie samo.  
- Ja - zaczął niepewnie Oskar, ale nie wiedział jak kontynuować swoją wypowiedź. 
- Wyduś to z siebie - zażądał Brave. – Dawaj!
- Nie wiem – powiedział z rezygnacja w głosie Oskar. 
Brave pokiwał głową jakby się tego spodziewał. 
- Przypominam Ci ją?
- Kogo? - zmarszczył brwi Oskar. 
- Moją matkę – odpowiedział, a jego słowa ociekały pogardą. 
- Nie – stanowczo zaprzeczył. - Dlaczego tak sądzisz?
- Może widzisz ją we mnie i nie możesz tego znieść. 
- Nie, to nie ma z nią związku. 
- Oczywiście, że ma. Zmieniłeś się po jej samobójstwie, a w zasadzie jeszcze przed. Przestałeś bywać w domu, nie rozmawailiście ze sobą, nie zajmowałeś się mną. A gdy odeszła wysłałeś mnie do internatu. Więc nie mów mi, że nie chodzi o nią. 
- Masz rację, zaniedbałem Cię i przepraszam. 
- Nie chcę przeprosin, tylko wyjaśnienia. 
- To skomplikowane.
- Nieważne, chcę wiedzieć. Zasługuję na to. 
- Zasługujesz na wszystko co najlepsze, a to takie nie jest. 
- Nie ochronisz mnie przed smutnymi i złymi rzeczami, tato. One po prostu się wydarzają. 
- Wiem. 
- Powiedz, proszę. Co robię źle?
- Nic, naprawdę. Jesteś dobrym chłopcem. 
- Nie mydl mi oczu. 
Oskar zacisnął mocno powieki. Nie mógł wymigać się od odpowiedzi. Ale mógł ją zmodyfikować tak by powiedzieć jak najwięcej prawdy, nie zdradzjąc całej reszty. I to zrobił. Nabrał tchu.
- Nie zapobiegłem śmierci twojej mamy i czuję się przez to winny. Nawaliłem jako mąż i ojciec. Dlatego nie mogę, nie potrafię przy tobie być.
- Tato, to nie twoja wina. Sama się zabiła. 
Ostatnie zdanie zadudniło w głowie Oskara. 
- Przed tym zdarzeniem, nie byliśmy z twoją mamą w dobrych stosunkach. Nie komunikowaliśmy się ze sobą, unikaliśmy się. To był dla nas trudny czas. Jej śmierć bardzo mnie zaskoczyła i przeraziła. Nie wiedziałem, dlaczego to zrobiła. Ale wiem, że mogłem temu jakoś zaradzić. To ja sprawiłem, że się od siebie oddaliliśmy i w porę tego nie naprawiłem. Więc, tak to jest moja wina. 
- Nie możesz myśleć w ten sposób. To nie ty pociągnąłeś za spust. 
Oskar opuścił głowę i zacisnął mocno usta. Wspomnienia wracały z podwójną mocą i wyrazistością. 

-Nie mogę z tym żyć. Nie mogę żyć dłużej ze sobą. Czuję się tak bardzo winna. Jak ty możesz ze sobą żyć, co? Hektolitry alkoholu i rżnięcie lasek Ci pomaga?- zakpiła Melani. 
- Nie musisz być taka ordynarna i proszę odłóż ten pistolet. 
Wyciągnął w jej stronę rękę w uspakająjącym geście.
- Co Cię to obchodzi? Mogę być jaka chcę. Ty samolubny, okrutny, morderco. 
- Może gdybyś mnie nie nienawidziła, łatwiej by Ci się żyło. Ulżyj sobie. Weźmy rozwód, weź Brave’a ze sobą. Zgodzę się z tobą, że jestem fatalnym ojcem. Zrób cokolwiek, ale odłóż tę broń. 
- Nie mogę. To poczucie winy zżera mnie od środka i ten wstyd. Gdybyś tylko wiedział, co to za uczucia, też chciałbyś ze sobą skończyć. 
- Masz syna, który Cię potrzebuje - przekonywał ją. 
- Nie umiem być już matką. Ja już nie wiem jak kochać – odjęła pistolet od skroni, musiał go tylko wyszarpnąć z jej dłoni. Zabrałeś mi to. To ty stawiasz mnie w tej sytuacji, nie chciałam tego. Wszystko zniszczyłeś - łkała.
Machała pistoletem, w ramach gestykulacji rękoma. Przestał słuchać co mówi, skupił się tylko na pistolecie. Musi go zabrać. 
Wyrwał się do przodu i przewrócił ją na łóżko. Zaczęli się szamotać. Melani mocno zaciskała palce na broni. Skierowała ją w stronę twarzy, wsadziła do ust. Miała takie silne ręce. Mocował się z nią. Warczała na niego. Wyzywała go. Ignorował to. Musiał zabrać jej ten przeklęty pistolet. Oddaj go, błagał bez słów, ale go nie słuchała. 
Nagłe puf, zakończyło ruch w pokoju. 
Nie.

- Tato- Brave zamachał mu przed oczami dłońmi i szturchnął w ramię.
Oskar pokręcił głową i wrócił do teraźniejszości. 
- Tak? – spytał. – Coś mówiłeś?
- Zawiesiłeś się, wiesz?
- Przepraszam. 
- Mówiłem, że to nie ty pociagnąłeś za spust. 
- Wiem, ale czuję się jakbym to zrobił. 
Odwrócił się od syna, nie będąc w stanie dłużej patrzeć w te ufne oczy. Pełne nadziei, oczekiwań.
- Nie możesz sprawić, żeby ona zrujnowała Ci życie. Oboje nie możemy na to pozwolić. Chcę z powrotem mojego ojca. 
- Ja nie potrafię, przepraszam. 
- Oczywiście, że potrafisz. To nie takie trudne. Pomogę Ci. Pozwól mi Ciebie odzyskać. 
Dotknął delikatnie jego ramienia, na co Oskar się odwrócił. 
- Razem nam się uda. Proszę. 
Jak ta mała istota, do której życia się przyczynił, mogła obdarowywać go tak bezbrzeżną miłością? Jak to możliwe. Po tym jak nawalił. Jak go zaniedbał. Gdyby znał całą prawdę... Ale nigdy jej nie pozna. Oskar na to nie pozwoli. I nie starci przez to syna. Zatrzyma go. Będzie szczęśliwy. Będą spędzać razem święta, wakacje, ferie, może niektóre weekendy. Naprawią swoją więź ojca z synem, syna z ojcem. To może zadziałać. To może się stać. 
Oskar pokiwał głową. 
- Tak, tak zrobimy. Zgadzam się. 
Brave przywarł do niego mocno. Oplótł tłuw ojca tak jakby bał się, że ponownie go straci. Oskar odwzajemnił uścisk, kiwając wciąż głową jak w jakimś tiku/transie. Każde kiwnięcie było kolejnym tak, było jak mantra, którą musiał przyswoić i zapamiętać. 
Dawno nie czuł się tak kochany, tak potrzebny. Jego życie wracało na dobre tory. Odbuduje ze zgliszczy swój świat, dla siebie, dla Brave. Umocni go, by już nigdy się nie rozpadł. To musi się udać. Nareszcie miał nadzieję. Nawet nie zauważył, że ma mokre od łez policzki. Otarł je szybko rękawem, ze śmiechem na ustach. 
Brave odsunął się i spojrzał w górę na ojca. 
- Mogę u Ciebie dziś przenocować? Proszę, proszę, proszę. 
Brave miętosił i ciągał  jego koszulę, jak gdy miał pięć lat i czegoś chciał. Oskar uśmiechnął się do tego wspomnienia. 
- Jasne. Możesz zostać na cały weekend.
- Naprawdę?- oczy zaświeciły mu się ze szczęścia. 
- Uhym- powikiwał głową. -Cały weekend. 
Brave zaczął skakać i biegać po całych mieszkaniu. Oskar dawno nie widział piękniejszego widoku i chciał go widzieć od teraz codziennie. 
 
W sobotę po jego pracy byli na basenie, zamówili pizzę do domu i wspólnie oglądali mecz. Brave opowiedział mu o szkole, ulubionych przedmiotach i nauczycielach, swojej przyjaciółce Lilu. 
- Podoba Ci się?- spytał Oskar. 
- Lilu? No weź, to tylko przyjaciółka- obruszył się Brave. 
- Każdy tak mówi.
- Ale ja mówię serio - zapewnił Brave, ale coś w jego oczach kazało sądzić, że nie jest to do końca prawda. 
Oskar postanowił nie drążyć. Zamiast tego zaczął wypytywać, czy może jakaś inna dziewczyna w klasie mu się podoba. Brave pokręcił przecząco głową.
- Nie potrzebna mi teraz dziewczyna, mam dopiero trzynaście lat. 
- Ja w twoim wieku bywałem zakochany.
- Naprawdę?- zdziwił się Brave. 
- Tak, miała na imię Alicja. Była niesamowicie bystra i zawsze odejmowało mi mowę w jej towarzystwie. Byłem skazany na porażkę, a jednak zaprosiła mnie na dyskotekę szkolną. 
- Ona Ciebie?
- Dziwne co? Była bardzo odważna i pewna siebie. 
- Zgodziłeś się?
- Oczywiście, miałem przepuścić taką okazję? W życiu. Chociaż było mi trochę wstyd, że to nie ja ją zaprosiłem, tylko ona mnie. 
- Rozumie się. Czyli to po tobie brak mi odwagi – podsumował Brave.
Oskar się zaśmiał.
- Dzięki, synu.
- Naprawdę. To Lilu jest tą odważniejszą. Zawsze wszystko załatwia, nie boi się stawiać nauczycielom i innym uczniom. 
- Pożyteczna towarzyszka. 
- Nawet nie wiesz jak bardzo. Jak mnie ktoś zaczepia, to idzie do tej osoby i przemawia jej do rozumu. Potem zwykle mam już spokój.
- Zaczepiają Cię?- zaniepokoił się Oskar. 
- Zdarza się -  zbył temat Brave. 
- Jeśli to byłoby coś poważnego, powiedziałbyś mi?- spytał kontrolnie. 
- Jasne- powiedział lekkim tonem Brave i wstał z kanapy i się przeciągnął. – Pójdę się umyć. 
Oskar pokiwał głową na znak, że przyjął to do wiadomości i włączył telewizor. Leciał akurat jeden z jego ulubionych filmów akcji Szklana pułapka z Brucem Willisem w roli głównej. Zapadł się głębiej w kanapę i skupił na filmie.
Z krótkiej drzemki wybudził go dzwonek przychodzącego połączenia. Szybko chwycił za telefon i odebrał. 
- Słucham?- spytał, równocześnie ściszając głośność telewizora.
- Tu Hopkins. Musisz przyjechać do Federalnego Biura Bezpieczeństwa, teraz.
- Stary, jest dwudziesta druga. Mam syna w domu, nie mogę po prostu pojechać. 
- Ja Cię nie pytam, ja oznajmiam, że masz to zrobić. Masz tu być najdalej za pół godziny, jasne? Inaczej oboje będziemy mieli kłopoty. 
Hopkins rozłączył się, nie dając czasu na odpowiedź. 
Oskar odłożył telefon i wpatrywał się tępo w ekran telewizora. Znowu nie miał wyboru. Był marionetką, którą poruszał Browni i pewnie się przy tym zaśmiewał. Jak ma wcielić  w życie swój idelany plan bycia dobrym ojcem, kiedy na każdym kroku będą przypominać mu o przeszłości. Jak ma patrzeć na syna bez poczucia winy, za to co zrobił. Jak ma okłamywać go, że wszystko jest w porządku. Już się przekonał, że nie może na sobie polegać. Że jest słaby, podatny na wpływy. Więc jak jego marzenia miały się spełnić. 
Ciężko wstał z kanapy, odsuwając te dołujące myśli na bok. Zapukał do drzwi łazienki i zakomunikował Brave’owi, że musi wyjść z powodu ważnej sprawy w pracy. Opuścił mieszkanie, przestawiając się na tryb, który nazwał: obojętność. Z obojętnością przywitał się z jak zwykle uśmiechniętym Hopkinsem, z obojętnością usiadł na przeciwko Catriny i Grega, z obojętnością prowadził z nimi rozmowę, z obojętnością odpalił papierosa i wypuszczał dym pod sufit. Gdy wyszedł z pokoju przesłuchań i chciał z obojętnością pożegnać się z Hopkinsem i udać się do domu, ten go zatrzymał.
- To nie koniec - powiedział z nieczęstą u niego powagą. 
- Pogadałem z nimi jak chciałeś - nie rozumiał Oskar.
- Tak, ale to ma ich tylko trochę sponiewierać, dołożyć zmartwień. Teraz musisz zrobić coś co ma realne znaczenie.
- To znaczy co?
- Musisz zażądać podania dokładnego adresu ośrodka, w którym działa Ruch Oporu.
- Nie powiedzą mi, nie ma szans- zaoponował.
- Zagrozisz, że zrobimy krzywdę dzieciom. 
Oskar spojrzał na Hopkinsa z przestrachem.
- Naprawdę to zrobicie?
- Nie, ale musimy ich nastraszyć, inaczej nie puszczą pary z ust. 
- Chwila- zreflektował się Oskar- nie możecie im tym zagrozić. Dobrze wiedzą, że nie możecie skrzywdzić dzieci, bo posiadają istotne dla Was informacje. 
- Poniekąd masz rację, ale możemu skrzywdzić je dostatecznie, by nadal mogłby wyjawić nam swoje tajemnice – odpowiedział z szaleńczym uśmieszkiem Hopkins. 
- Zwariowałeś! Zamierzacie je torturować?- nie mógł uwierzyć Oskar.
- Zrobimy co będzie konieczne- zapewnił Hopkins.
- Bestie. 
- Nie większe niż ty, jak sądzisz?- zaśmiał się pogardliwie.
- Nie spodziewałem się tego po tobie – błądził oczami pełnymi trwogi po twarzy kolegi, poszukując człowieka, którego niegdyś nazywał przyjacielem, nikogo takiego nie znalazł. 
- Ja po tobie też się pewnych rzeczy nie spodziewałem. Na przykład takiej głupoty i nawiności. Myślisz, że dlaczego jestem twoim najbliższym kolegą od czterech lat, hę? Bo mi na tobie zależy? A może mi kazano? Gdybyś był mądrzejszy, dawno znałbyś odpowiedź na to pytanie. Poczekaj tu, naradzę się z resztą czym ich zastraszymy. 
Hopkins zostawił skołowanego Oskara na korytarzu- wiedział, że się stamtąd nie ruszy, i wszedł do pokoju przylegającego do pokoju przesłuchać. W środku stały komputery i urządzenia nagrywające wszystkie rozmowy przesłuchiwanych. Przy niedużym prostokątnym stole siedzieli dwaj funkcjonariusze państwowi- Agata i Chistian, ubrani w służbowe uniformy i notujący coś w notatnikach, a Agata także na tablecie. Podnieśli wzrok na Hopkinsa, gdy tylko stanął przed nimi.
- Zrobi to?- spytała Agata.
- Tak, jest przecież po naszej stronie. 
- Ale kiedyś był po ich – przypomniała.
- Jest lojalny wobec nas, możemy to przyjąć za pewnik i ruszyć dalej - spytał lekko zniecierpliwiony, chociaż podejrzliwość jego kolegów była uzasadniona. 
Agata i Christian, jak również inni zajmujący sie sprawą Ruchu, nie wiedzieli o czynach, które popełnił Oskar, a które on i jego przełożeni wykorzystywali jako szantaż. Wciąż obawiali się, że Johnson tylko udaje, że / iż im pomaga w zwalczaniu organizacji, a tak naprawdę dla nich szpieguje. Tylko nieliczni znali prawdę. 
- Okej- przypieczętował wcześniejsze ustalenia Christian. – A więc najlepszym sposobem na nich nie będzie straszenie a oferta. 
- Jaka oferta?- spytał Hopkins, któremu już nie podobał się obrót tej rozmowy.
- Damy spokój dzieciakom przez parę lat. Uwolnimy ich, nie będziemy szpiegować, w żaden sposób nie będziemy wtrącać się do ich życia. Tego właśnie chcieli –odpowiedziała mu Agata.
- To naprawdę świetna propozycja, powinni na to przystać- dodał Christian
- Nie uwierzą, że to rzeczywiście zrobicie- zanegował genialność pomysłu Hopkins.
- Nie mają wyboru. Jeśli jest szansa, że damy im wnukom święty spokój, zgodzą się. Szczególnie, że wizja dzieciaków w więzieniu jest nie do zniesienia, dla ich przepełnionych miłością serc - mówiła z udawanym współczuciem Agata. 
- Na raz dowiedzą się, że dzieci są w więzieniu i że je uwolnimy. To podejrzane. Wcześniej myśleli, że zbiegły. Sam nie wiem, wolałbym straszaka a nie ofertę. 
- Tu nie chodzi o sposób, ale o cel. Mamy szansę nareszcie ruszyć dalej i przymknąć całą organizację - tłumaczyła z werwą Agata, chociaż i tak wiedziała, że nie przekona Hopkinsa.
- I zająć się kolejnymi sprawami- pociągnął dalej ich piękną wizję Christian.
Hopkins chwile dumał, choć już postanowił, że wykorzystają ich pomysł, choć w dalszym ciągu wolał swój. 
- Okej, do dzieła. Powiem Oskarowi co i jak, i za chwile zobaczymy rezultaty. 





Część 2

Rozdział 1

Hope waliła w worek treningowy raz po raz uwalniając codziennie pokłady agresji i flustracji. Znała sztuki walki, potrafiła się bronić, a to wszystko za sprawą treningów z Benem, na które uczęszczała od trzech lat. Grace - kobieta, która ich przygarnęła, gdy tego potrzebowali z Pou, dała im nową tożsamość i stała się ich prawną opiekunką - wskazała im Bena na trenera, a po niedługim czasie okazało się, że to jej mąż, z którym rozstała się siedem lat wcześniej. Małżonkowie do siebie wrócili, Ben zamieszkał razem z nimi, od teraz ( wtedy) był również ich prawnym opiekunem, co niekoniecznie cieszyło Hope. Ben był dobrym trenerem, dość wymagającym, ale nie chciała by robił jej za ojca. Niestety, nie miała w tej kwestii za wiele do powiedzenia. Były plusy tej sytuacji i Hope starała się je dostrzegać na każdym kroku, by życzliwiej spoglądać na Bena. Dzięki niemu Grace była szczęśliwsza, a Hope czuła się bepieczniejsza. No i, od tamtej pory mogła częściej korzystać z hali do ćwiczeń. 
Od sześciu lat Hope była Sky Cloudy, a jej brat Pou - Johnnym Cloudy. Otrzymała nowe życie, po wszystkim co się wydarzyło w przeszłości i była za to wdzięczna. Jednak i tak wspomnienia ją nie opuszczały i nawiedzały ją czasem w snach w formie koszmarów. Jej egzystencja była przepełniona strachem o siebie i brata, który udało jej się wytłumić na tyle by móc normalnie funkcjonować.
Przestała walić w worek. Bolały ją ręce, ciężko połykała powietrze, pot spływaj jej po czole i karku zlepiając kosmyki włosów. Tak łatwo można było zatracić się w tej czynność. Była wykończona. Ile to trwało? Miała wrażenie, że z pół godziny. 
Sięgnęła po butelkę wody leżącą przy ścianie i wyżłopała na raz połowę jej zawartości. Usiadła i oparła się o blachę, z której zbudowana była hala. Nie był to najlepszy materiał, nagrzewał się latem, zimną nie trzymał ciepła. Przewróciła się na bok i przytuliła policzek do zimnej posadzki. Co za ulga. 
Na raz usłyszała walenie do drzwi. Błyskawicznie podniosła się do pionu i nasłuchiwała. Dobijanie się do drzwi nie ustępowało. Serce przyśpieszyło bieg, przelewała się przez nią fala strachu. Jeśli to Ci sami ludzie, którzy przyszli ostatnio do Bena i zamiast niego znaleźli ją, a potem pobili i związali. Gdy Ben ją znalazł godzinę później nie miała nawet siły na niego krzyczeć, płakała tylko z bólu i szczęścia, że jej nie zgwałcili. To było trzy miesiące temu, od tamtego czasu wymieniono zamek od drzwi na bezpieczniejszy, Ben dał jej broń i nauczył jak z niej korzystać. Ale czy to rzeczywiście by pomogło, gdyby tamci mężczyźni wrócili? Czterech na jedną? Mogłaby każdego postrzelić i uciec, ale czy byłaby do tego zdolna? Czy nie sparaliżowałby ją strach? Wolałaby nigdy nie przekonać się jakie są odpowiedzi na te pytania.
Potruchtała do przyległego do hali niewielkiego murowanego domu, gdzie Ben kiedyś mieszkał i wyciągnęła z jednej z kuchennych szafek pistolet. Model Walther, reszty szczegółów nie zapamiętała. Wróciła do hali i powoli zbliżała się do drzwi, w które ktoś nadal walił. Trzymała broń pewnie oburącz, z palcem nad cynglem. Starała się uspokoić galopujące serce, powoli wciągając i wypuszczając powietrze. Gdy dotarła do drzwi, wyjęła z kieszeni klucz i przekręciła go w zamku. Płynnym ruchem otworzyła drzwi i schowała się za nimi, a potem szybko zza nich wychynęła celując w potencjalnego agresora. Okazał się nim przystojny młody chłopak, parę lat od niej starszy, z mocno zarysowaną szczęką i zapadającymi się policzkami, o śniadej cerze i zaskoczonym spojrzeniu ciemnobrązowych oczu. Wyglądał przyjaźnie, ale to o niczym jeszcze nie świadczyło. Lepiej być podejrzliwym, niż dać się zaskoczyć. 
- Kim jesteś?- spytała przybysza chłodnym tonem.
- Jestem Stone i pracuję z twoim ojcem. A ty pewnie jesteś jego córką, Sky, tak?- nie wydawał się przestraszony widokiem wycelowanej w niego lufy, co trochę ją wkurzyło, wyglądała aż tak niegroźnie czy po prostu często bywał w tego typu sytuacjach? Nie ważne. 
- Adoptowaną córką – poprawiła go, bo nie lubiła nieścisłości w tym temacie.
- Wspominał, że jesteś precyzyjna w tej kwestii. 
- Często o mnie plotkujecie?- rzuciła z obojętnością, choć była tego ciekawa. 
- Nie, coś nie coś o tobie wiem. Ale tego się nie spodziewałem.
- Wycelowanej w Ciebie lufy?- podsunęła.
- Właśnie. Ładnie trzymasz pistolet, Ben musiał Cię dobrze wyszkolić. Każdego tak witasz?
- Jesteś pierwszy - powiedziała grobowym głosem.
Zasępił się. 
- Kiepsko jak na początek znajomości- zauważył. 
Przewróciła oczami. 
- Żadnej znajomości nie zaczynamy. Spływaj, a jak to jest tak bardzo pilne to poczekaj na zewnątrz do przyjścia Bena, ja Cię nie wpuszczę.
Po tych słowach zatrzasnęła drzwi i z ulgą się o nie oparła. Wypuściła głośno powietrze. Chyba nie jest źle jak na pierwszą konfrontację z człowiekiem, kiedy się w niego celuje. Odłożyła pistolet na miejsce. Wróciła do ćwiczeń, wypędzając z głowy myśli, że atrakcyjny facet czeka zapewne pod drzwiami. Porozciągała się, powtórzyła sekwencję ciosów, uników i innych ruchów potrzebnych w walce. Pobiegała po obwodzie hali. To wszystko razem musiało zająć minimum godzinę, jak nie więcej. Po tym czasie drzwi się otworzyły i wszedł przez nie Ben razem z tym chopakiem spod drzwi. Zignorowała ich i biegała dalej. W końcu poczuła się tak wycieńczona, że położyła się na ziemi i zamknęła oczy. Krew głośno pulsowała jej uszach. 
Usłyszała czyjeś kroki i otworzyła oczy. Automatycznie podniosła się do pozycji półleżącej. 
- Pomóc Ci wstać?- zaoferował się Stone, stając przed nią. 
Spojrzała na niego nieprzychylnie i podniosła się sama. 
- Dzięki, ale umiem o siebie zadbać.
Minęła go i poszła do części mieszkalnej, żeby zabrać swoje rzeczy. Zastała tam siedzącego przy stole Bena. 
- Hej - podniósł na nią wzrok. – Dałaś sobie niezły wycisk - zlustrował ją wzrokiem.
- Aż tak to widać?- spytała zawiedziona.
Pokiwał głową. 
- Możesz skorzystać z prysznica. 
Miała na to wielką ochotę, ale pokręciła przecząco głową.
- Nie kiedy jest tu obcy facet. 
- To mój kumpel. 
- Młodych masz kumpli.
- Pracujemy razem. 
- Też tak twierdził.
- Dobrze, że go nie wpuściłaś. Ostrożność jest wskazana. 
- Co ty nie powiesz – zadrwiła z mieszanką goryczy. 
- Przepraszam, okej. Ile jeszcze razy mam to powiedzieć. Naprawdę jest mi z tego powodu przykro. To nie powinno się wydarzyć – kolejny raz to powtarzał, już nie mogła zliczyć który. 
- Nieważne - zbyła go i zdjęła z krzesła bluzę, po czym wciagnęła ją na siebie. 
Było jej gorąco, ale chociaż ukryje pod tym przylepiony do ciała podkoszulek i mokre od potu włosy. Pożegnała Bena i miała już opuścić halę, ale nadziała się na Stona. 
- Idziesz już? - spytał.
- Jak widać - wzruszyła ramionami.
- Podprowadzić Cię?
- Nie jesteś zajęty?
- Właściwie to tak.
- No właśnie. To cześć. 
Wyszła, a on jeszcze długo za nią patrzył, nawet po tym gdy drzwi się już zamknęły. 
Stone wrócił do Bena, który dalej czegoś szukał.
- Wytłumaczysz mi dlaczego twoja adoptowana córka celowała do mnie z broni?- zagadnął.
Ben posłał mu ciężkie spojrzenie i westchnął.
- Naprawdę nie lubię do tego wracać. 
- Okej, byłem po prostu ciekaw. Znalazłeś to zamówienie?- spytał, zmieniając temat.
- Nie. I za cholerę nie wiem, gdzie je posiałem. Szukałem już w domu, teraz tu. Zniknęło.
- Kiepsko. 
- Taa. Nie potrzebuję kolejnych nieporozumień z mafią.
Stone zmarszył brwi.
- Były jakieś przetem?
Ben pokiwał głową.
- Tak. Zarzucili mi, że sprzedaję broń Trovowskiemu.
- Nosens.
- Wiem, ale paru kolesiów Tonego tak myślało i przyszli się ze mną policzyć. Akurat mnie nie było, a Sky trenowała. Szybko rozprawili się z zamkiem i weszli. Zastali ją i pobili. Wiedzieli, że należy do mojej rodziny, więc chcieli dać mi nauczkę sponiewierając ją. Gdy wróciłem leżała związana, z nosa ciekła jej obficie krew, miała masę siniaków i zadrapań. Strasznie płakała, aż sam myślałem, że się rozryczę. Na szczęście...- zaczerpnął tchu, ale nie kotynuował.
- ... nie zgwałcili jej - dokończył za niego Stone poważnym tonem.
Ben pokiwał głową. 
- Dalej ma mi za złe za to co się stało i nie dziwię się jej. Nigdy nie mieliśmy dobrych relacji, a po tym tylko pogorszyłem jej zdanie o sobie. 
- Wybaczy Ci prędzej czy później - dodał mu otuchy.
- Pewnie masz rację. Ale nawet jeśli ona mi wybaczy, to ja nie wybaczę sobie. Pomyśl, a co jeśli...
- Ale się nie stało i tego się trzymaj - wszedł mu w słowo Stone. 
Ben zamilkł i przez chwile się nie odzywał.
- Co było potem?- spytał Stone, przerywając ciszę. 
- Poszedłem w pełni uzbrojony do Tonego Sajczento i zrobiłem mu awanturę. Nic nie wiedział o tej sytuacji. Jego chłopcy nic mu nie powiedzieli. Przeprosił za nieporozumienie i wydał mi osoby, które to zrobiły. Dotkliwie ich pobiłem i pogodziliśmy się z Tonym. Ale pewnie część jego ludzi ma mi za złe, że sprałem ich kumpli. Więc muszę teraz być bardzo ostrożny. Tony nie jest głupi i wie, że nie chcę mieć kłopotów z mafią, ale jeśli wszyscy jego podwładni będa przeciwko mnie, to możliwe, że nie będzie miał wyboru i ich poprze. A wtedy, nie wiem co może się stać. 
- Wow, nie spodziewałem się tego. Zamierzałeś mi kiedyś o tym powiedzieć? Siedzimy w tym razem, pamiętasz? Jeśli ty miałbyś kłopoty, to ja też – za/wyrzucił mu Stone. 
- Przepraszam. Możemy zakończyć temat i ruszyć dalej - poprosił.
- Pewnie - odpowiedział bez wahania Stone. – Pomoge Ci tego szukać. 
Po przekopaniu całego mieszkania, w końcu znaleźli zagubiony dokument. 
- Eureka!- powiedział z uśmiechem Ben.- Dzięki za pomoc.
- Nie ma za co. 
- Uczcijmy to - zażądał Ben i wyciągnął z lodówki dwa piwa. 
- Każda okazała jest dobra, żeby się napić – zaśmiał się Stone.
Sączyli powoli alkohol, rozkoszując się truimfem. 
- Powiedz mi coś więcej o Sky - poprosił Stone. 
Ben popatrzył na niego uważnie.
- Wpadła Ci w oko?
- Wydaje się interesująca. 
- Wymijająca odpowiedź, jakże by inaczej. To nie jest dziewczyna dla Ciebie, nie tylko dlatego że tego nie aprobuję i że jesteś problemowym człowiekiem, ale też dlatego że ona jest zagadką. 
- To znaczy?
- Grace nigdy nie wyjaśniła mi jak to się stało, że ich przygarnęła. Ta historia jest owiana tajemnicą, której chyba nigdy nie dane będzie mi poznać. Mogę się tylko domyślać, że w życiu Sky i jej brata wydarzyło się coś tragicznego, o czym nie chcą rozmawiać i co chcą zostawić za sobą. Ja nie pytam, Grace nie pyta, więc tak naprawdę oboje nie wiemy kim oni tak naprawdę są. 
- Ale opiekujesz się nią jak własnym dzieckiem.
- Oczywiście, jestem jej prawnym opiekunem. Tworzymy rodzinę, Grace zawsze o tym marzyła. Jest taka szczęśliwa. Poza tym gdyby nie te dzieciaki pewnie nigdy nie zszedłbym się z Grace. Wiele im zawdzięczam, ale równie wiele o nich nie wiem. 
Stone pokiwał głową i się zamyślił, na co odezwał się Ben.  
- Stone, wiem o czym myślisz. Że jesteś w stanie jej pomóc uporać się z przeszłością jakakolwiek ona by nie była, ale przekonałeś się, że to nie jest takie proste. Nie muszę Ci chyba przypominać Idy - spojrzał na niego wymownie. 
- Nie porównuj Sky do Idy, wiesz że to zupełnie inny przypadek- żachnął się.
- Ciągnie Cię do tajemniczych, skomplikowanych dziewczyn, z jedną taką mieszkasz. Spencer, tak? Na szczęście w niej się nie zakochałeś. 
- To też złe porównanie. 
- Oczywiście, dla Ciebie każde będzie złe, bo każda z nich była inna. Masz nieposkromioną chęć pomagania ludziom, których uważasz za podobnych do siebie, a co z tobą? Kiedy uporasz się z własnymi problemami. 
- Świetnie sobie z nimi radzę - rzucił ostro.
- To kiedy zdejmiesz rękawiczki?- spojrzał wymownie na jego dłonie, okutane w rękawiczki bez palców.
Stone spojrzał na swoje ręce z żalem i złością, zacisnął mocno powieki, by odgodnić napływające wspomnienia. Pokręcił przecząco głową i spojrzał z bólem na Bena.
- Nigdy. I ty postąpiłbyś tak samo. Nie chciałbyś, żeby ktoś coś takiego widział. 
- Spróbowałbym z tym walczyć – nachylił się do Stone jak dobrotliwy ojciec i kontynuował. - Żeby nie przejęło nade mną kontoli. Możesz zrobić tatuaże na rękach w okół tego znaku, zmienić go, żeby nikt go nie rozpoznał. 
- Ten znak jest zbyt rozpoznawalny, dobrze o tym wiesz. 
- A co jak rękawiczka kiedyś spadnie Ci z ręki? Myślałeś o tym?
Stone się podniósł, podszedł do blatu kuchennego i się o niego oparł. Nie lubił prowadzić rozmów na ten temat, szczególnie dlatego, że Ben miał racje. Ale to nie było takie proste. 
- Według Ciebie powinnienem zrzucić rękawiczki, pozbyć się bandaża z piersi i żyć jakby tamto nigdy się nie zdarzyło. Nie płoszyć się jak ktoś spojrzy na moje ręce, tylko wypinać pierś, bo to jest już za mną. Pokazać, że te znaki, które ktoś kiedyś na mnie nałożył, mnie nie definiują. Udawać, że nie widzę szyderczych uśmiechów niektórych albo litościwych czy współczujących. Może ty umiesz coś takiego zrobić i znieść, ja nie. 
Mówiąc to, nie patrzył mu w oczy, nie potrafił. Nie kiedy łzy zbierały mu się w oczach, choć i tak nie zacząłby płakać. Nie płakał od bardzo dawna i nie zamierzał tego zmieniać. Był cały rozedrgany, wyjął lekko trzęsącą się dłonią paczkę papierosów i wziął jednego.
- Masz zapalniczkę?- spytał Bena.
Ten pokiwał głową i odpalił mu papierosa. Stone poczuł znajomy smak w ustach, zaciągnął się i wypuścił do góry dym. Zaczynał się uspokajać. Nikotyna działała na niego kojąco. Poczuł się bardziej zrelaksowany niż jeszcze chwile temu. 
Ben też zapalił. Palili w ciszy. Gdy oboje dopalali papierosy, Ben wyciagnął popielniczkę i zgnietli pety. Smużka dymu wiła się przez jakiś czas. 
- Zrobisz jak będziesz chciał, ale wiedz, że jeśli będziesz próbował się umówić z moją córką, to nie będę z tego powodu zadowolony – powiedział z lekko wyczuwalną groźbą Ben. 
Stone wybuchł śmiechem i poczuł, że całe wcześniejsze spięcie wyparowało. Nie musiał mówić Benowi, kiedy nie drążyć tego konkretnego tematu, sam wiedział kiedy przestać. I tak też zrobił. Wrócił na bezpieczne wody, które były wszystkim co bezpośrednio nie dotyczyło jego przeszłości. 
- Nie uważasz, że lepszy ktoś znany niż jakiś przybłęda niewiadomo skąd? – zagadnął z figlarnym uśmiechem, bo kompletnie nie przejęły go pouczania Bena. 
- Myślę, że na razie nikt nie jest odpowiedni. 
- Zaborczy ojciec?- podsunął Stone.
- Coś takiego. Nigdy nie sądziłem, że będę jednym z takich ojców.
- Serio? Widać na pierwszy rzut oka- przyznał ze znawstwem Stone. 
- Pogrążasz się, ale to jeszcze nic. Nie ja będę stanowił największy problem przy twoich zalotach, tylko ona. Nie jest zbyt towarzyska. 
- Zobaczymy. 
Stone pożegnał się z Benem i wrócił do domu. Mieszkał razem z trójką przyjaciół w suterenie w jednej z kamienic. Każdy z nich miał nieduży pokój dla siebie, wspólną niewielką łazienkę z prysznicem oraz kuchnio-salon. Całości przydałby się remont, ale obecnie nie było ich na to stać. 
Gdy wszedł do domu, wszyscy siedzieli już przy obiadokolacji, którą starali się jadać wspólnie. 
- Hej, spóźniłeś się - przywitała się Spencer. 
- Przepraszam, zagadałem się z Benem. 
- Dalej nie kumam, jak możesz kumplować się z tak starym gościem – odezwał się Brook. 
- Hej - powiedziała Allison, gdy usiadł między nią a Spencer przy stole. 
Uśmiechnął się do niej w odpowiedzi i nałożył sobie jedzienie na talerz. Allison była najmniej gadatliwa z całej czwórki oraz najmłodsza. Miała piętnaście lat i wciąż wyglądała jak dziecko, dlatego tak też ją traktowali. Miała okrągłą pucołowatą buzię, wielkie jasnobrązowe oczy i krótkie brązowe włosy z grzywką. Była niewysoka i drobna. Stone czuł się za nią najbardziej odpowiedzialny z całej trójki – znalazł ją na ulicy i przygarnął, bo zrobiło mu się jej szkoda, a przy okazji potrzebował jeszcze jednego lokatora do mieszkania. Znalazł jej pracę i nigdy nie powiedział reszcie, gdzie ją poznał. Nigdy też nie wypytywał czemu znalazła się na ulicy. 
Brooka - wysokiego, patykowatego chłopaka z czupryną blond włosów, poznał w warsztacie samochodowym, gdzie ten pracował. Szybko się polubili i poszli razem na piwo. Stone zaproponował mu wspólne mieszkanie po w miarę przystępnej cenie. Brook się zgodził, bo żył wcześniej w małej klitce nad warsztatem i pragnął odmiany. Tego samego wieczoru poznali Spencer, która wparowała do baru, zamówiła drinka, wypiła go duszkiem i oznajmiła, że nie ma czym zapłacić. Spojrzała wtedy błagalnie na Stone i Brooka, którzy zlitowali się nad nią i opłacili jej napój. Potem do późna siedzieli we trójkę i rozmawiali. Spencer była bezpośrednia, brawurowa i bez ogłady. Potrafiła zaczepić nieznajomego i poprosić, by postawił jej drinka. Meżczyźni chętnie to robili, bo była bardzo atrakcyjna i przy tym pewna siebie, zapewne liczyli też na coś więcej, ale ona odprawiała ich ruchem ręki. Dali się nabrać, a ona ich wykorzystała. 
Była tylko niewiele niższa od Brooka, smagła i zgrabna – wyglądała jak modelka. Była w tym samym wieku co Brook, czyli o rok młodsza od Stone. 
W tym momencie wstała od stołu i włożyła brudny talerz do zlewu. Odwróciła się i uważnym wzrokiem zlustrowała całą czwórkę. 
- Wychodzę, więc radźcie sobie beze mnie – wyrecytowała rutynową formułkę przed wyjściem do pracy. 
Zabrała z pokoju torebkę i lekką dżinsową kurtkę, po czym wyszła z domu. Po dziesięciu minutach dotarła do klubu, w którym robiła za barmankę. Weszła drzwiami dla pracowników i natknęła się na Nadię. 
- Hej- przywitała się. 
- Hej, dobrze że jesteś. Natalie zachorowała, jesteśmy dziś tylko we dwie. Będziemy miały pełne ręce roboty. 
Spencer pufnęła z niezadowolenia, na co jej koleżanka się roześmiała. 
- Jak to usłyszałam chciałam zareagować tak samo, ale się opanowałam.  Nie jestem tobą. 
- Wiem, tylko ja posiadam ten przywilej – podsumowała z wyższością Spencer.
Zostawiła swoje rzeczy w małym pokoju z biurkiem i kanapą, i poszła za Nadią do baru. Były ubrane tak samo – w białe bluzki na ramiączkach z pokaźnym dekoldem i krótkie dżinsowe spódniczki. Z racji, że nikt nie widział ich stóp wkładały dla wygody tampki. 
Klub otwierano o dwudziestej. Choć był poczatek tygodnia zawsze była masa chętnych do picia i tańczenia. Klub oferował także pokoje na pięterku. Spencer zdarzało się tam zawitać, ale niewiele pamiętała poza niezbyt czystą pościelą i popsutym zamkiem. 
Najpierw zjawili się stali bywalcy, którzy dobrze znali barmanki i często wchodzili z nimi w luźne pogawędki. Mężczyźni pracujący w fabrykach, w usługach, klienci różnej maści styrani życiem. Spencer poznała tu wiele interesujących ludzi. Niektórzy przytaczali jej cały swój życiorys. Inni zwierzali się z aktualnych problemów, starała się im doradzić albo wesprzeć w paru słowach. W takich momentach zapominała o sobie i stawała się osobą empatyczną. Poza miłymi aspektami pracy jako barmanka, doświadczyła też wielu niemiłych sytuacji. Gburowaci albo obleśni mężczyźni, którzy potrafili tak pochylić się nad blatem, by wsadzić ręce do stanika. Obelgi, uciążliwe zaczepki. Ale i tak w ogólnym rozrachunku lubiła swoją pracę oraz wynagordzenie, które za nią dostawała. 
W klubie poza zwykłymi elementami, przebywali też Ci wyjątkowi z mafii. Spencer czasami odróżniała ich od reszty, ale często nie. Zbyt stapiali się z otoczeniem. Starała się trzymać z dala od mafi, nie tylko przez namowy Stone, ale też z własnych obaw. Krążyły różne plotki do czego zdolni są gangusi, jeśli czegoś chcą. Spencer nie chciała być w kręgach ich zainteresowań. 
Jak spod ziemii pojawił się przed nią Freddi. Dwudziestoparoletni facet o urodzie małego chłopca, nerwowych tikach i wścibskim usposobieniu. Spencer była wdzięczna, że oddziela ich blat baru. 
- Hej, Spensuś - przywitał się, dziwacznie zdrabniając jej imię, czego nie znosiła. 
- Hej, Freddi – przywitała się z wymuszonym uśmiechem. 
- Poproszę gin z tonikiem.
- Się robi- zasalutowała i poszła przygotować trunek. 
Miała ochotę robić go jak najdłużej, ale zadowolenie klientów było ważnym elementem jej pracy i płacy. Liczyła na napiwki, a od Freddiego dostanie go na pewno, o ile go nie wkurzy. 
- Proszę – podała mu zamówienie i otrzymała zapłatę z napiwkiem, tak jak oczekiwała. 
- Masz ochotę się spotkać po pracy?
- Freddi zadajesz mi to pytanie od paru miesięcy i wciąż Ci odmawiam, rusz głową i wyciagnij z tego wniosek - rzuciła ostro i nareszcie poczuła się w swoim żywiole. 
Wolała traktować ludzi obcesowo niż miło, a mogła sobie na to pozwolić po napiwku.
- Ale tamta noc...- zaczął Freddi patrząc na nią z uczuciem. 
- To była pomyłka. Ile razy mam to powtarzać. 
Podeszła do innego klienta, ale kątem oka widziala, że Freddi nie zmienił pozycji. Obsłużyła parę osób i do niego wróciła.
- Spływaj stąd - warknęła na niego. 
- Mówił Ci ktoś, że wyglądasz gorzej, gdy się złościsz - nerwowo się zaśmiał i poruszył głową w tiku. 
- Pierdol się - wysyczała mu do ucha, na co spojrzał na nią osłupiały.
- Nie słyszałeś pani- zwrócił się do Freddiego obcy jej mężczyzna, przy którym jej napastor wyglądał jak kurdupel.
Freddi spojrzał lękliwie na faceta i czmychnął w tłum. 
Spencer odetchnęła z ulgą. Spojrzała przyjaźnie na swojego wybawiciela, który swoją drogą był niczego sobie. Wyglądał jak ktoś nadziany i miał czarujący uśmiech. 
- Dzięki - odezwała się do nieznajomego. 
- Nie ma za co. Nie lubię jak ktoś się naprzyksza kobiecie, to takie nieeleganckie. 
-Nie każdy ma dobre maniery, dlatego miło poznać dla odmiany dżentelmena. 
- Jestem Harry.
- Spencer.
Uścisnęli sobie dłonie i rownocześnie Harry posłał jej powłóczyste spojrzenie, które Spencer dobrze znała. 
- Wybacz, muszę wracać do pracy – uśmiechnęła się przepraszająco. 
- O której kończysz?- zawołał za nią.
- O czwartej. 
- Poczekam. 
Spencer w to wątpiła, ale się pomyliła. Gdy klub zamknięto, czekał na nią pod drzwiami dla pracowników z butelką wina. 
Pożegnała się z Nadią, która przesłała jej porozumiewawcze spojrzenie i podeszła do Harrego.
- Poczekałeś - rzuciła zamiast powitania.
- Niektóre kobiety są tego warte – odparł aksamitnym tonem.
- Co to?- wskazała na napój.
- Chateau de Chantegrive rocznik 1997.
- Wow. Stare, dobre wino i drogie - zauważyła.
- Takie lubię najbardziej. 
- Czym się zajmujesz?
- Inwestuję. 
- W co?
- W budynki. 
- Nieźle. 
Szli w bliżej nieokreślonym kierunku i Spencer już wiedziała jak zakończy się to spotkanie. 
- Mam w mieszkaniu jeszcze inne dobre wina, chciałabyś dokonać degustacji? – spytał niewinnie.
Uśmiechnęła się. Żadko, kto wysilił się na finezyjną formę, spytania czy ma ochotę na seks. Miła odmiana.
- Z przyjemnością - zgodziła się. 


Hope obudziła się, ciężko dysząc. Była cała spocona, policzki miała mokre od łez. Spojrzała na Pou ułożonego przy niej, oddychał równomiernie. Opadła na poduszki i zaczerpnęła tchu. 
To tylko zły sen, tylko zły sen. 
Od trzech miesięcy nie było nocy, by nie obudziła się w takim stanie. Za pierwszym razem krzyczała i obudziła tym Pou, Grace i Bena. Nie mogła się uspokoić i Grace musiała długo tulić ją zanim ponownie zasnęła.
Śniło jej się, że czterech mężczyzn przychodzi i ją bije do nieprzytomności. Cucą ją i przyciskają do ściany, a wtedy jeden z nich ściąga jej spodnie i słyszy jak rozpina swoje. Słyszy wszystko dokładnie. Rozpinany zamek, jego przyśpieszony oddech. Czuje jego obleśny oddech i odplatające ją ręce, które dotykają jej piersi. Gdy ma się wydarzyć najgorsze, budzi się. Cały czas w tym samym momencie. Nie wiedziała, czy jej mózg nie jest w stanie dopowiedzieć reszty, czy ma jakiś instykt, który każe jej się obudzić w odpowiednim momencie. Nie znała praw snu i chyba mało kto je poznał. 
Zwlekła się z łóżka i jak najciszej wyszła z pokoju i trafiła do łazienki. Spojrzała w lustro na swoje umęczone odbicie. Potargane włosy, rozbiegany wzrok, smutna mina, wypieki po płaczu, podkrążone oczy. Rano znów będzie musiała skorzystać z kosmetyków Grace, by jakoś zatuszować oznaki koszmarów. Wmawiała ostatnio swojej opiekunce, że śpi dobrze, nie męczą ją złe sny, ale kłamała. Grace patrzyła na nią smutno, więc chyba jej nie uwierzyła. 
Pochlapała twarz wodą, wytarła ją ręcznikiem i wróciła do pokoju. Ułożyła się wygodnie i przytuliła do brata. Spali w jednym łóżku od sześciu lat i żadnemu to nie przeszkadzało. Hope chciała mieć Pou jak najbliżej siebie. Był dla niej najważniejszą istotą na ziemi i zrobiła by dla niego wszystko. Troszczyli się o siebie nawzajem, bo mieli tylko siebie. Tylko oni przetrwali. 
Zasnęła i tym razem śniło jej się coś przyjemnego.

Spencer odbudziła się i przeciągnęła jak kot. Szybko zorientowała się, że nie jest w swoim łóżku, a co ważniejsze w swoim mieszkaniu. Wróciły do niej ostatnie wydarzenia. Poszła za Harrym do jego mieszkania. Było olśniewające i duże, i na pewno drogie. Tak jak obiecywał miał kolekcję starych win. Piła niewielką ilość każdego. Były przepyszne. Alkohol szybko na nią zadziałał i rzuciła się na Harrego w przypływie namiętności, a potem wylądowali w łóżku. 
Ściągnęła z nagiego ciała kołdrę i podniosła swoje ubrania z podłogi. Szybko wciągnęła je na siebie i zajrzała pod łóżko. Torebka leżała tam z pełną zawartością. Zawsze tak robiła, wrzucała najcenniejsze rzeczy pod jakiś mebel. Gdyby ktoś chciał, z łatwością by je ukradł, ale chociaż nie leżały na widoku i nie kusiły. Wzięła jeszcze kurtkę z ramy łóżka i miała zamiar po cichu wymknąć się z mieszkania. Jednak Harry ją ubiegł i przyszedł do sypialni z tacą pełną jedzenia. Nici z planu.
- Dzień dobry - przywitał ją z szerokim uśmiechem Harry i położył tacę na łóżku.
- To dla mnie?- wskazała na siebie zdziwiona. 
- Tak. 
- Miło z twojej strony, ale nie musiałeś. Tylko się ze sobą przespaliśmy. 
- I dlatego nie mogę zrobić Ci śniadania? Na pewno jesteś głodna. 
Westchnęła, zawsze znajdzie się oporny. Dlatego lepiej było wyjść niepostrzeżenie.
- Serio, dziękuję. Muszę już iść. 
Opuściła pokój i skierowała się do drzwi. Wyprzedził ją i zagrodził drogę. 
- Umówimy się?- spytał z nadzieją.
- Nie – odmówiła dobitnie.
- Dlaczego?
- Bo nie umawiam się na randki.
- Dlaczego?
- Nie szukam nikogo na stałe. 
- Dałem się nabrać, że coś zaiskrzyło – wyznał z żalem.
- To urocze, że istnieją jeszcze romantycy. Ale spójrzmy prawdzie w oczy, chciałam Cię tylko zerżnąć i było bardzo miło, ale czas na mnie. Więc z łaski swojej wypuść mnie – zażądała.
Odsunął się od drzwi i spojrzał na nią wrogo.
- Jeszcze za mną zatęsknisz - powiedział i zabrzmiało to jak groźba.
- Nie licz na to - odgryzła się i wymaszerowała z mieszkania. 
Szybkim krokiem przemierzyła całą drogę do domu. Czuła się nieswojo. W domu nikogo nie było, wszyscy pracowali. Weszła pod prysznic i dokładnie wyszorowała się gąbką. Przebrała się w piżamę i zrobiła sobie kanapki i herbatę. Szybko to skonsumowała i położyła się do łóżka.
Przez chwile wierciła się i zastanawiała się kim tak naprawdę był Harry. Jego ostatnie słowa zabrzmiały jak groźba, i to nie amatora, a osoby która często zachowuje się w ten sposób. Gdzie zniknął miły dżentelmen? I co taki bogaty gość robił w klubie, w którym pracowała? Na pewno były inne bardziej eksluzywne, pasujące do ludzi jego pokroju? Czy mógł być z mafii? To by tłumaczyło groźbę. 
Nie, to prostu rozżalony facet, poszukujący drugiej połówki- wmawiała sobie, choć nie bardzo w to wierzyła. 
Niespokojna, w końcu zasnęła.









Rozdział 2

Hope przez cały tydzień kątem oka rejestrowała obecność Stona, który przychodził razem z Benem. Wiedziała, że to nie przypadek, że od kiedy się poznali, pojawiał się codziennie. Ignorowała jego istnienie i skupiała się na ćwiczeniach. Jednak zastanawiała się na co on liczy. Lubił być w jej pobliżu, to mu wystarczało? Nie składał już żadnych propozycji, gdy natykali się na siebie w drzwiach. Uśmiechał się przyjaźnie, wymieniali parę nic nieznaczących słów i to wszystko. Jego zachowanie przypominało oswajanie się ze zwierzyną, którą chce się zdobyć. Intrygowała ją ta strategia w jego wykonaniu. Jednak po mimo tego nie poświęcała mu zbyt wielkiej uwagi w swoich myślach, choć już zdążył się w nich zalęgnąć i nie chciał sobie iść. 
W piątek jej dwie najbliższe koleżanki Andżelika i Katy, których w żadnym wypadku nie nazwałaby przyjaciółkami, namówiłby ją na pójście na imprezę do ich wspólnego kolegi Chada. Hope była zwykle odporna na ich namowy, ale czasem może z ciekawości albo z satysfakcji, że dziewczyny zabierają ją, bo czują się przy niej bezpieczniej, szła z nimi. Tak było i tym razem. 
Gdy stała przed lustrem, zastanawiając się w co się ubrać, przy czym wybór tyczył się T-shirtów, jej brat Pou objął ją w pasie i uśmiechnął do niej w lustrze. Pogładziła go po ciemnoblondwłosach, które zaczęły opadać mu na oczy.
- Co tam kamracie?- spytała. 
- Nic, skończyłem czytać książkę.
- O Robin Hodzie?
Pokiwał głową.
- I jak podobała Ci się?
- Pewnie. Przygody, walki, sprawiedliwość, bohaterstwo.
Zaczął skakać po łóżku i zachęcił ją gestem, by do niego dołączyła. 
- Zawali się pode mną. 
Przestał na chwilę skakać, spojrzał na nią krytycznie. 
- Nie ma pod czym. Skóra, mięśnie i kości. 
- Dzięki, że dodałeś mięśnie, to mi pochlebia. Przesuń się. 
Ucieszony zaczął znów skakać. Dołaczyła do niego i skakali tak śmiejąc się do rozpuku. W końcu opadli z sił i legli na łóżko. Hope sięgnęła po brata, podwinęła mu koszulkę i zaczęła go gilgotać. Zaczął się wić, śmiać i błagać o litość. Dała mu spokój dopiero, gdy prawie spadł z łóżka. 
- Przez Ciebie, muszę jeszcze raz się umyć przed wyjściem – zaśmiała się, czując jeszcze ciepło na twarzy. 
- Musisz wychodzić?- spytał ze smutną minką. 
Obróciła się bokiem i oparła na łokciu.
- Już się zgodziłam. Poza tym – podniosła się i przygładziła włosy i bluzkę – może nie będzie tak źle. 
- Nie lubisz takich imprez, więc będzie źle. Ale możemy urządzić własną. Popiszesz się nowymi krokami, których się nauczyłaś albo które wymyśliłaś. Proszę, proszę, proszę - zaczął skomleć. 
- Nie mogę. Umówiłam się z dziewczynami i nie chcę ich zawieść. 
Parsknął, założył ręce na piersi i udał obrażonego. 
- To wygląda raczej zabawnie niż groźnie - zauważyła i wyjęła z szafy dwie bluzki. – Która?- spytała go. 
Przez chwile nie opowiadał, na co ona ostentacyjnie przewróciła oczami. 
- No - przynagliła go. – Pomóż mi.
Pufnął i teatralnie zwrócił oczy ku sufitowi jakby chciał powiedzieć „Co ja z nią mam?!”, ale przydreptał pod lustro. 
- Są podobne. 
- Mają inne kolory – podkreśliła. 
- To ta błękitna - wskazał na tę, którą trzymała w prawej ręce.
- Dzięki, a teraz się odwróć.
- Bo nigdy nie widziałem Cię w staniku – rzucił z przekąsem, ale posłusznie się odwrócił. 
- Jesteś już duży, nie powinieneś oglądać takich widoków. 
- Proszę Cię! – wyrzucił ręce w górę. – Śpimy w tym samym łóżku.
- To co innego.
- Dla niektórych nie. 
- A dla nas tak – upierała się. – Dobra, muszę już iść, więc trzymaj się.
Z ociąganiem poszedł za nią, by za nią zamknąć.
- Baw się dobrze!- rzucił na odchodnym, a ona pomachała mu ręką na pożegnanie. 
Było chłodniej niż zakładała, więc ucieszyła się, że jednak założyła kurtkę. Do domu Chada doszła w niecałe dziesięć minut. Część gości stała na zewnątrz czekając zapewne na innych. Wśród nich były Andżelika i Katy. Gdy ją dostrzegły zamachały energicznie.
- Myślałam, że zrezygnujesz - przyznała się Katy, gdy już się odnalazły. 
- Obiecałam Wam - przypomniała Hope. 
- No wiem, ale jednak – mętnie tłumaczyła Katy.
- Możemy już wchodzić - zarządziła Andżelika i ruszyły do domu. 
Hope kiedyś już tu była na innej imprezie. Dziewczyny rozejrzały się za alkoholem, a w tym czasie Hope znalazła miejsce na kanapie. Po chwili dosiadł się do niej chłopak ze zwisającym kolczykiem w nosie i zaczesanych na prawo włosach, a za nim dołaczyły kolejne osoby. 
Chłopak z kolczykiem zlustrował ją wzrokiem i przedstawił się jako Jake. 
- Wyglądasz jak hipsterka – zagaił rozmowę. 
- Czemu? – spytała, choć odpowiedź mogła być tylko jedna.
- Chodzi o ubranie. Jesteś zakryta od stóp po głowę, mało tu takich dziewczyn. 
- Preferuję inny styl.
- I to jest cool - kiwał głową i uśmiechał się do niej. – Hipsterzy powinni trzymać się razem. 
- A ty niby czemu nim jesteś?
- Jak to, a kolczyki?- rzucił lekko urażony.
Rozejrzała się już po pełnej sali gości i zwróciła się do Jake.
- Naliczyłam przynajmiej dziesięć osób, które też je mają. 
Popatrzył na nią jak na wariatkę. A mogła przyznać mu rację, miałaby święty spokój. 
- To już nie jest cool – powiedział, kręcąc głową. – Nie możesz sobie przywłaszczać tego tytułu.
- Nie przywłaszczyłam go – patrzyła na niego z niedowierzaniem. – Sam mnie tak określiłeś. I masz rację wyróżniam się i jest mi z tym dobrze. Naprawdę będziesz mi to miał za złe?
Kiwnął głową i miała wrażenie jakby rozmawiała z dzieckiem. 
- Takie to dla Ciebie ważne, żeby zaliczali Cię do grona hipsterów? 
- To mnie definiuje, sprawia że jestem wyjątkowym - powiedział jakby w to szczerze wierzył.
- Jeśli to jedyna rzecz, która sprawia, że się tak czujesz to chyba najwyższy czas, żeby ktoś Ci powiedział, że jesteś wyjątkowy, bo jesteś sobą. To wystarczy – próbowała go przekonać. 
Chłopak zaczął wiercić się na kanapie. Najwyraźniej jej słowa wprawiły go w zakłopotanie. 
- Może...- zaczął się podnosić, ale nakazała mu ruchem ręki, by usiadł i zrobił to zdezorientowany.
- Pójdę już, zwolnię kanapę dla kogoś innego – zdecydowała i się oddaliła. 
Fatalny początek wieczoru, ale czego się spodziewała? Normalnej rozmowy, w której nie musiałaby mówić tego, co ktoś chce usłyszeć? 
Pytanie retoryczne. Przemknęła między ludźmi do stolika, na którym stało picie, tam też spotkała swoje koleżanki. 
- Jest tu coś bezalkoholowego?- spytała Katy. 
Ta wydęła policzki, rozejrzała się pobieżnie po stole i rozłożyła bezradnie ręce.
- Nie ma, ale może znajdziesz coś w kuchni. Ale potem do nas wróć. Zaczynamy grać w „prawdę czy wyzwanie”, a ty nigdy jeszcze nie brałaś w tym udziału. 
Zostawiła dziewczyny i szybko znalazła kuchnie, w której aktualnie nikogo nie było. Z lampki nad kuchenką sączyło się trochę światła, ale większość pomieszczenia tonęła w półmroku. Otworzyła lodówkę i szybko znalazła sok pomarańczowy, wyjęła go i usłyszała jak ktoś wchodzi do kuchni. Odwróciła się. To Chad. 
- Przepraszam, szukałam czegoś do picia – zaczęła się tłumaczyć, bo było jej niezręcznie grzebać w czyjejś lodówce.
- Nie przepraszaj. Mi casa es tu casa – uśmiechnął się. 
Wyglądał na zrelaksownego jak zwykle zresztą. Miał na sobie podkoszulek, który ładnie opinał jego szerokie barki i dżinsy. W ręce trzymał butelkę piwa. Podszedł do jednej z szafek i wyjął z niej kubek. Podał Hope. Uśmiechnęła się z wdzięcznością i nalała sobie soku. Upiła łyk i spojrzała na Chada.
- Co tu robisz? Nie powinnieneś być w salonie z resztą, w końcu to twoja impreza - spytała zaciekawiona, bo naprawdę nie spodziewała się tu jego obecności.
Przysiadł na stole i wskazał jej ręką, żeby do niego dołączyła. 
- Tak między nami - spojrzał na nią znacząco – to się wstawiłem.
- Już? Dopiero zaczęła się impreza.
- Zaczeliśmy pić z chłopakami wcześniej.
- I dlatego siedzisz ze mną w kuchni?- nie rozumiała. 
- Poniekąd. 
- A tak naprawdę?
- Grają w tą głupią grę, której nienawidzę. 
- „Prawda czy wyzwanie.”
- Dokładnie. Powinnienem zrobić regulamin, który zabrania tej gry w moim domu. 
- Zrób to  – zachęciła go.
- Coś ty. Oni ją uwielbiają. Darmowe pocałunki, wstydliwe sekrety, karmią się tym. 
- Masz coś do ukrycia – zrozumiała. 
Pokiwał głową. 
- A kto nie ma?- zadał pytanie retoryczne. – A ty dlaczego tu nadal jesteś?
- To znaczy?
- Przyszłaś po sok, bo nie pijesz alkoholu, ale mogłabyś wrócić do reszty?- zauważył.
Spojrzała na szklankę i powoli nią obracała, tak że płyn przechylał się na jedną lub drugą stronę.
- Nie lubię tej gry tak samo jak ty – odpowiedziała w końcu.
- I?- ciągnął ją za język.
- Może lubię twoje towarzystwo - zasugerowała.
- Nie ściemniaj – machnął ręką. – Nieważne. Możemy razem przeczekać tę głupią grę. 
- Okej - zgodziła się. 
Milczeli przez jakiś czas, pijąc. 
- Wiesz, że nikomu nie powiedziałem, że wtedy płakałem?- przerwał ciszę.
Hope dobrze pamiętała tamto wydarzenie. Niełatwo zapomnieć płaczącego Chada, jednego z popularniejszych chłopaków w szkole. To był przypadek, że go zobaczyła. Rozmawiała po lekcji z nauczycielką i przyszła do szatni, gdy wszyscy już się przebrali i poszli do domu. Ale nie Chad. Siedział na ławce i płakał. Mieli po dwanaście lat. Podeszła do niego i dała mu chusteczkę. Pociagnął nosem i spojrzął na nią przez łzy. 
- Obiecaj, że nikomu nie powiesz? – poprosił wtedy, chlipiąc.
- Obiecuję.
I tak zrobiła. Nikt nigdy się nie dowiedział, że Chad płakał tego dnia w szkole. 
- Domyślam się – odpowiedziała. – Czemu do tego wracasz?
- Bo nigdy Ci nie podziękowałem – spojrzał na nią. - Więc robię to teraz. Dziękuję.
- Nie ma za co. 
- Właśnie, że jest. Gdybyś komuś powiedziała straciłbym szacunek wśród kolegów, a to było dla mnie bardzo ważne. Dlaczego mi pomogłaś?
Upiła łyk ze szklanki, patrząc przed siebie. 
- Bo chciałam i bo to było słuszne - wyjaśniła.
- Nie byłem dla Ciebie zbyt miły – przypomniał. 
- I dlatego miałbym Cię pogrążyć? Bo mi dokuczałeś, podstawiałeś nogę i ciągałeś za włosy. 
- To niezły powód. 
- Byliśmy dziećmi, poza tym nie jestem mściwa. Zasługiwałeś na to, żeby spełnić daną Ci obietnicę.
- To było cztery lata temu – zamyślił się. – Wiesz czemu wtedy płakałem?
- Nie musisz mi mówić – zastrzegła.
- Ale chcę. Utrzymałaś mój sekret w tajemnicy, więc możesz się dowiedzieć –  mówił z pasją. – No więc, dzień wcześniej ojciec zbił mnie tak mocno, że ledwo chodziłem. Gdy obudziłem się rano dalej mnie bolało, ale było już znacznie lepiej. Na lekcji wuefu kolega mnie przez przypadek kopnął i zwinąłem się z bólu. Zacisnąłem zęby i ćwiczyłem dalej. Ale gdy szatnia opustoszała rozryczałem się, bo mnie bolało i nie chciałem wracać do domu. 
Hope słuchała tego z rosnącym współczuciem. 
- Dlaczego Cię bił?- spytała.
- Pracował w mafii, to stresująca robota. Odreagowywał na mnie i straszym bracie. Brat wyjechał, zostałem sam, więc oberwało mi się mocniej. Nie trwało to długo, bo stary coś spartaczył i go sprzątnęli. 
- Wybaczyłeś mu?
- Nie myślę o tym. 
Nie pytała więcej, bo i tak sporo powiedział, choć nie musiał. Wypiła to co pozostało w szklance i wsadziła ją do zlewu. Opłukała ją i postawiła na suszarkę. 
Odwróciła się do pogrążonego w myślach Chada. W jego zielonych kocich oczach gościł ból i smutek. Podeszła i położyła mu rękę na ramieniu, chcąc dodać mu otuchy. Po tamtym wydarzeniu, gdy Chad płakał, stał się dla niej miły. Czasem zdarzało im się porozmawiać, ale nigdy tak szczerze jak dziś. To pewnie przez alkohol. 
- Wracamy do pozostałych?- spytała z entuzjazem, który chciała przelać na niego. 
- Chyba nie jestem gotowy - pokręcił głową, ale dalej wydawał się nieobecny. 
- No chodź. To twoja impreza. Nie masz żadnych planów? 
- Miałem poflirtować z Katy, to twoja koleżanka, co nie?- zwrócił się do niej, już bardziej świadomy. 
- Tak, bardzo Cię lubi.
- Wiem, widzę to. 
- Co zamierzasz?
- Serio pytasz?- teraz wrócił już na dobre.
- Cokolwiek to jest nie skrzywdź jej, okej?
- Niektóre rzeczy bolą – podkreślił słowo „rzeczy” i uśmiechnął się szelmowsko. 
Przewróciła oczami. 
- Będziecie parą?- pytała dalej.
- Jesteś swatką? Bo nie jestem gotowy na twój zestaw pytań, a chyba jeszcze ich dużo. 
- To nie tobie będzie się potem wypłakiwać w rękaw -  wytłumaczyła. 
- Będę miły i delikatny, okej?
- Okej – kiwnęła głową bez przekonania. – To idź do niej.
Ruszył do drzwi i obrócił się, gdy już w nich stał.
- Dzięki.
Uśmiechnęła się w odpowiedzi, a on poszedł w głąb domu. Chwilę później ruszyła jego śladami. W salonie panował tłok. Na kanapie i sofie siedziało dwa razy tyle ludzi, ile przewidział producent. Niektórzy podpierali ściani, inni tańczyli do głośnej muzyki, która wypełniała cały pokój. Trzeba było też uważać, żeby nie nadepnąć na ktogoś siedzącego na podłodze. Hope wyszukała Katy. Siedzała na kolanach Chada na sofie. Gdy ich spojrzenia się spotkały, pomachała energicznie do Hope, cała promieniejąc. Potem znowu poświęciła całą swoją uwagę Chadowi, który wyglądał na całkowicie zaabsorbowanego jej osobą. Hope uśmiechnęła się sama do siebie. 
Nie znalazła wśród gości Andżeliki, może była w innym pokoju albo na górze z jakimś chłopakiem. 
Hope usiadła pod jedną ze ścian, gdzie było trochę puściej i zamknęła oczy. Wsłuchała się w muzykę i głosy. Śmiechy, chichoty, wrzaski, piski, podniesione głosy, szepty. A wszystko to spowite elekroniczną muzyką dobiegającą z wieży stojącej na komodzie. Otworzyła oczy i przypatrywała się każdemu z osobna. W salonie mieściło się z czteredzieści góra pięćdziesiąt osób, każdy z jakąś historią. Ze swoimi problemami i marzeniami. Czy dla tych ludzi i dla niej jej rodzice chcieli czegoś lepszego? Dlatego założyli Ruch Oporu? Za to zginęli? Czym to coś było? Bezpieczeństwem, uwolnieniem od mafii, brakiem sprzedających pod szkołą dilerów? 
Podniosła się i odpędziła natrętne myśli. Piętrzące się pytania bez odpowiedzi. Poszukała łazienki, ale szybko przekonała się, że ktoś korzysta z niej do celów innych niż te najbardziej podstawowe. Zaczęła walić w drzwi, ale Ci po drugiej stronie albo jej nie usłyszeli albo z powodzeniem ignorowali. Poirytowana pobiegła na górę i trafiła na łazienkę. Drzwi były otwarte. Zadowolona weszła do środka, ale nie była sama. Jakiś chłopak sikał, lekko się chwiejąc. Wkurzyła się i pouczyła go, żeby na przyszłość zamykał się na klucz. Nie wyglądał na przekonanego. Gdy wyszedł wreszcie odetchnęła z ulgą. Parę minut później zbiegła ze schodów, zastanawiając się co dalej robić, gdy zauważyła chłopaka, który wyglądał znajomo. Nie mogła uwierzyć, że on tu był. Nie była pewna czy chce się z nim witać czy woli traktować go jak powietrze. W tym czasie on zdążył ją zauważyć i wysłać jej szeroki uśmiech. Przerwał rozmowę z blondynem i podszedł do niej. 
- Stone – stwierdziła. 
- Sky, hej – przywitał się.
- Co tu robisz?- spytała raczej wrogo. 
- Zostałem zaproszony. 
- Znasz Chada?- niedowierzała.
- Mój kumpel Brook zna – wskazał na chłopaka, z którym przed chwilą rozmawiał. – A ty?
- To kolega z klasy. 
- Czyli też masz szesnaście lat?- raczej stwierdził niż spytał. 
- Tak. A ty?
- Dwadzieścia, Brook dziewiętnaście. 
Tak myślała, choć w jego oczach było coś, za co równie dobrze mogłaby mu dać kilka lat więcej. 
- Jak się bawisz?- spytał. 
- Świetnie. 
- Nie wyglądasz.
- Cóż, musisz wierzyć mi na słowo. Ale będę się powoli zbierać. 
- Czyli jednak się nudzisz?
- Nie mam tu już nic ciekawego do roboty - podała wymijającą odpowiedź.
- Możesz zostać ze mną. 
- Marzę o tym - zakpiła.
Uznała konwersację za skończoną i zmierzała do wyjścia, ale niefortunnie drogę zagrodził jej Zack. Uśmiechnął się na jej widok z wyższością i wymalowaną na twarzy złośliwością. 
- Mała Sky, witaj – odezwał się pierwszy i Hope miała wrażenie jakby nagle cała uwaga zgromadzonych w salonie skupiła się na ich dwójce. 
- Cześć, Zack- odpowiedziała z wymuszoną serdecznością. 
- Nie wiedziałem, że chadzasz na imprezy. To nie w twoim stylu. 
Zmusiła się, by nie przewrócić oczami. 
- Zdarza się. 
- Zatańczysz?- spytał od niechcenia. 
- Sorry, ale nie tańczę z tobą. 
- Niby czemu?- przybliżył się do niej z groźbą w oczach.
- Dobrze wiesz dlaczego – odparła chłodno. 
- Bo pobiłem twojego małego braciszka?! - zaśmiał się i zwrócił z triumfem w stronę imprezowiczów. – Dalej boli.
- Nie mam ochoty z tobą gadać, więc z łaski swojej zejdź mi z drogi – rzuciła, plując mu tymi słowami w twarz. 
Poczuła jak w pokoju temperatura maleje, wszyscy z przestrachem patrzyli na scenę rozgrywającą się przed ich oczami. Mało kto miał tyle odwagi, by przeciwstawić się Zackowi Sajczento, siostrzeńcowi Tonego Sajczento - lokalnego mafioso. Ten kto to zrobił, musiał liczyć się z konsekwencjami, ale Hope nie wydawała się tym przejęta. Nie miała zamiaru płaszczyć się przed nikim pokroju Zacka, przed nikim kto skrzywdził jej brata, a on to zrobił. Dlatego Grace zapoznała ich z Benem, który zaczął ich trenować. By ta sytuacja nigdy się nie powtórzyła. Była przekonana, że w równej walce, jeden na jeden, dałaby bez trudu rady Zackowi. Nie wydawał się zbyt przygotowany do konfrontacji. 
- A co jak nie zejdę, hę?- rzucił jej wyzwanie i widać było, że gotuje się w nim od gniewu. 
- To Cię ominę.
Chciała wyminąć go z prawej, ale złapał ją za ramiona i mocno szarpnął. Zachwiała się, ale nie straciła równowagi. Zacisnęła usta i spojrzała wrogo na Zacka. 
- O co Ci chodzi? 
- Przeproś. 
- Za co?
- Za to, co twój brat zrobił mojemu.
- Nie przeproszę za coś czego nie żałuję. 
- To było trzy lata temu, mogłabyś poczuć w końcu skruchę. 
Parsknęła. No pięknie. Jeszcze tego brakowała. 
- Daj mi odejść, to oboje zakończymy ten wieczór miło. Co ty na to?- zaproponowała. 
- Przeproś.
- Nie zamierzam- odszczeknęła i ponownie spróbowała go obejść. 
Znów złapał ją za ramiona i mocno odepchnął. Upadła na pupę i szybko się podniosła. Ruszyła na niego z pięściami, ale ktoś zagrodził jej drogę. To był Stone. Miał poważną minę i na sto procent nie bał się Zacka. Ale co on do diabła zamierzał zrobić?!
- Co powiecie na rozejm? – rzucił pomysł. – Szczerze to nikt nie ma ochoty na wasze przepychanki. To miała być impreza a nie pokaz siły. 
- Chcesz się bawić w mediatora, Stone - Zack popatrzył na niego rozbawiony.  – Nie radzę. Nie jesteś taki nietykalny jak Ci się wydaje. 
- A ty taki bezkarny, jak myślisz – uśmiechnął się swobodnie, acz z wyczuwalną groźbą. – Jeśli jej stanie się krzywda, będziesz miał kłopoty. 
- Niby jakie? Skopiesz mi dupsko? Nie żartuj sobie, nie odważysz się ze mną zadrzeć.
- A ty odważysz się zadrzeć z Benem Crickiem?- spytał wyraźnie z siebie zadowolony jakby ta sytuacja sprawiała mi frajdę.
- Nic do Ciebie nie mam, tylko do niej - wskazał na Hope. 
- No właśnie – potwierdził jakby to wszystko tłumaczyło, bo w istocie tak było.
- Co ty kręcisz?
- Sky to córka Bena – wyjaśnił Stone. – Dalej będziesz ją popychał?- rzucił zaczepnie. 
Do Zacka powoli docierały nowe informacje. Z jednej strony powiązanie Sky i Bena, nie było tajemnicą, a z drugiej nie afiszowali się z tym nadto. Dlatego niczym dziwnym nie była niewiedza Zacka oraz innych osób z otoczenia. 
- Niech to szlag - zaklął Zack, wyraźnie niezadowolony. 
Hope w czasie gdy Zack trawił niusy wyminęła go i opuściła dom Chada. 
- Kiedyś stracisz ten swój immunitet i twoja dziewczyna też, wtedy się policzymy – zagroził Zack Stonowi, ale on go już nie słuchał. 
Wybiegł z domu, chcąc dogonić Sky. Stała za ogrodzeniem. 
- Wszystko okej?- spytał kontrolnie. 
Zwróciła na niego nieprzychylne spojrzenie. 
- Nie potrzebowałam pomocy- odezwała się.
- Myślę, że jednak potrzebowałaś. 
- Bo znasz moje potrzeby, proszę Cię - wyrzuciła ręce w górę. -Pojawiasz się znikąd i zgrywasz bohatera? To twój sposób na podryw. Znudziło Ci się gapienie z oddali?- wyrzucała z siebie nabuzowana.
- Nie ma za co – zaśmiał się. 
- Co Cię tak bawi?- zaatakowała go.
- Nie umiesz przyznać, że Ci pomogłem. 
- Przecież mówię, że ...
- Ta jasne - wszedł jej w słowo. – A ja jestem złotą rybką. Pobiłabyś się z nim, co? Wiem, że umiesz. Pewnie byś go pokonała. Ale co potem, jego koledzy by oddali dwa razy mocniej. Masz ochotę na kolejne pobicie?- mówił rzeczowym poważnym tonem, przy okazji wyciągnął z kieszeni papierosy i odpali jednego.
Spoglądał na nią i dostrzegł, że zranił ją do żywego. Poczuł się głupio.
- Przepraszam, nie powinnienem - zaczął się tłumaczyć, ale przerwała mu.
- Skąd wiesz?- spytała twardo.
- Od Bena, powiedział mi. 
Nie czekała na dalsze wyjaśnienia. Ruszyła przed siebie. 
- Poczekaj, pracujemy razem, muszę wiedzieć takie rzeczy. Inaczej by mi nie powiedział. To aż tak źle, że to wiem? Nikomu nie wygadam, obiecuję -  dotrzymywał jej kroku i próbował ratować sytuację.
Nie reagowała na jego słowa, aż w końcu gwałtowanie się obróciła i popchnęła go. Cudem, nie upadł na ziemię. Spojrzał na nią skołowany. Miała zaczerwienione policzki, z oczu ciekły jej łzy, na twarzy rysował się grymas bólu i złości.
- Ty podły dupku. Dowiedziałeś się i zacząłeś mną interesować. Rycerz na białym koniu. Ukoję twoją zranioną duszę, ulituję się nad twoim biednym losem – mam to gdzieś. Nie potrzebuję twojej fałszywej życzliwości i pomocy. Jesteś taki jak wszyscy. 
- Wszyscy stali jak słupy, kiedy rozmawiałaś. Wszyscy nie ruszyli palcem, gdy Cię popchnął. Wszyscy nic by nie zrobili, gdyby doszło do bijatyki – wyliczał poirytowany. – Dalej jestem wszyscy?
- Nie mam ochoty z tobą gadać – krzyknęła.
- Super. Nikt Ci nie każe tego robić. Ale mogę Cię chociaż odprowadzić. 
Nie protestowała, więc całą drogę przebyli razem w ciszy. Na pożegnanie wymienili się zdawkowym „hej” i rozeszyli, każdy w swoją stronę. 
Hope weszła do domu i zatrzasnęła za sobą mocno drzwi. W kuchni napatoczyła się na Bena. Siedział w drucianych okularach na nosie i coś czytał. Hałas, który spowodowała, odciągnął go od tego co robił wcześniej.  
- Coś się stało?- spytał z troską.
- Życie się stało - odburknęła i zamierzała pójść prosto do pokoju, ale zdała sobie sprawę, że jest głodna. 
Skręciła do lodówki, wyjęła z niej masło i wędlinę, sięgnęła do chlebaka po bułkę. Gdy przygotowywała sobie kanapkę, Ben uważnie ją obserwował. 
- Pokłóciłaś się na imprezie, ktoś Ci zrobił krzywdę?- spytał w końcu.
- Co cię to obchodzi?
- Martwię się. 
- I nie dotrzymujesz obietnic – warknęła, po czym wsadziła bułkę do ust. 
- Co?- spytał nie rozumiejąc.
- Powiedziałeś swojemu kumplowi Stone’owi o moim pobiciu, wielkie dzięki – wyrzuciła na jednym oddechu.
Ben przymknął oczy i miał przy tym wyraz człowieka umęczonego życiem. Potem zwrócił się do niej. 
- Masz rację powiedziałem mu w końcu. Ta sytuacja była niebezpieczna dla nas obojga, dlatego uznałem, że musi wiedzieć. To kwestie zawodowe. 
- A co z obietnicami one już się nie liczą?- rzuciła ostro. 
- Oczywiście, że się liczą, kochanie.
- Nie mów na mnie kochanie – prawie krzyknęła. 
- Okej- poddał się. – Nie będę tak na Ciebie mówić i przepraszam. 
Hope nie odzywała się przez chwile pałaszując resztę kanapki. 
- Skąd się dowiedziałaś, że on wie? Widzieliście się? – zaczął wypytywać. 
- Był na imprezie u Chada. Przywitaliśmy się, a potem kiedy miałam już wychodzić natknęłam się na Zacka. Stoczyłam z nim utarczkę słowną siejąc tym samym postrach. Kiedy przeszło do rękoczynów wmieszał się Stone. Czuł się w konfrontacji z Zackiem jak ryba w wodzie i uświadomił mu, że jestem twoją córką, więc lepiej żeby mnie więcej nie popychał – opowiedziała z wysoko uniesionymi brwiami i zadziwiająco spokojnym głosem.
- Nieźle - skwitował Ben. – Chyba zadziałało, bo jesteś cała i zdrowa – bardziej stwierdził niż spytał. 
- Dzięki, nic mi nie jest. 
- Nie doszłaś do momentu, w którym Stone ujawnia, że wie o twoim pobiciu.
- Wyszłam z domu, a on pobiegł za mną. 
- A ty zaczęłaś mu wyrzucać, że nie potrzebowałaś jego pomocy- dopowiedział z niedowierzaniem w oczach, po jej minie zrozumiał, że ma rację. 
Potarł czoło i cicho się zaśmiał. 
- Chyba nie muszę Ci mówić, że zrobiłaś głupio. 
- Poradziłabym sobie sama – odpowiedziała z zacięciem. 
- Pewnie. Pobilibyście się, wygrałabyś, a potem miałbym kolejne problemy na głowie. 
Nie odpowiedziała.
- Od kiedy unosisz się tak dumą i jesteś taka nierozsądna? – spytał chyba bardziej siebie niż ją. 
- Pewnie broń jego, swojego złotego chłopca – mówiła z goryczą. 
- On nie jest złotym chłopcem, Jezu, będziesz zazdrosna?- pokręcił głową z niedowierzaniem. – On nie jest moim dzieckiem, choć może pełnię w jego życiu rolę ojca. Nawet nie wiesz ile złych dezycji podjął w życiu, nic o nim nie wiesz. Nie powinien wspominać w takim momencie, że wie o pobiciu, ale się uniósł jak rzuciłaś się z pretensjami. Oboje powinniście przeprosić. 
Skończył mówić i wrócił do tego co czytał. Hope poszła do łazienki, umyła się i ubrała w piżamę, myśląc nad tym co powiedział. Miał rację, że przesadziła. W gruncie rzeczy Stone jej pomógł i to wcale nie umniejszając jej wartości u oczach Zacka. Może przy następnym starciu wspomni o Stonie jako o jej wybawicielu, ale nie będzie patrzył na nią jak na kogoś słabszego. Może właśnie reakcja 
Stona sprawi, że Zack zacznie się jej bać. To byłoby miłe uczucie. Może strach, że straciła w oczach Zacka, spowodował że tak potraktowała Stona. A może duma, o której wspominał Ben. Hope radziła sobie dotąd sama. Staczała werbalne bitwy bez pomocy innych, bez wsparcia. Może o to chodziło, to była odpowiedź. Zdecydowała, że przeprosi Stona za swoje zachowanie. Było w nim coś co lubiła, może ta pewność siebie albo siła widoczna w jego oczach i ruchach. Sama nie wiedziała, ale czuła że są siebie warci, jakolwiek dziwnie by to nie brzmiało. 
Zajrzała do kuchni - Ben dalej czytał. Podeszła do niego po cichu i oparła się na jego plecach. Nie przestraszył się. Odłożył książkę i położył swoją rękę na jej. 
- Co tam?- spytał. 
- Myślałam nad tym, co mi powiedziałeś. Masz rację, przeproszę go. 
- To miło - uśmiechnął się kącikiem ust, zadowolony, że jego monolog coś dał.
- Ben - zaczęła. 
- Hmm?- spytał.
- Będziesz miał przeze mnie problemy? 
Zaśmiał się.
- Nie. Zresztą nie martw się większość problemów tworzę sam, taki już jestem. 
- Szanują Cię wśród tych mafiosów?- ni to stwierdziła ni spytała. 
- Stwarzają takie pozory. Ale ta sytuacja z tobą pokazała mi, że nie jest tak dobrze jakbym chciał. 
- Dlaczego zacząłeś to robić? Handlować bronią?- zaczęła wypytywać, naprawdę tego ciekawa. 
- Znam się na broni. Nigdy Ci nie mówiłem, ale byłem kiedyś żołnierzem.
- Naprawdę?- niedowierzała, chociaż to wiele tłumaczyło.
- Uhm. Nie lubię się tym chwalić. Rzeczy, które robiłem – przerwał by nabrać głęboko powietrza – były złe i się ich wstydzę. Ale gdy udało mi się stamtąd odejść...
- Jak?- weszła mu w słowo.
- Chyba nie za wiele wiesz, o tym jak funkcjonuje wojsko w naszym kraju?
- A jak funkcjonuje?- odpowiedziała pytaniem.
- To historia na inny moment. W każdym razie pomogli mi dobrzy ludzie, nie tylko mi. Zajmowali się wyciąganiem z Miasteczek Wojskowych żołnierzy, którzy nie chcieli już pełnić służby albo nie zgadzali się z tym co robią. 
- Jacy to byli ludzie?
- To tajemnica. 
Hope się zasępiła. Coś jej podpowiadało, że musi poznać odpowiedź na to pytanie. Spróbuje innym razem. 
- Ale wracając do mojej pracy. Znam się na broni, umiem z niej korzystać i nadaję się do pertraktowania z ludźmi. W zamian mam niezłe zarobki i relacje z niebepiecznymi ludźmi. 
- Uważasz, że warto?
- Nie nadaję się do niczego innego. 
Hope się zamyśliła. Czemu wcześniej nie pytała o takie rzeczy Bena? Może temu, że wcześniej nie byli w tak dobrych stosunkach co teraz. Co prawda, zdarzało jej się gniewać na niego za pobicie, ale paradoksalnie to wydarzenie zbliżyło ich do siebie. Hope weszła nie chcący do świata, w którym funkcjonował Ben i może to było kluczem. Chciała go jeszcze spytać, czemu Stone z nim pracuje, ale uznała, że spyta Stona osobiście.

Stone siedział na dachu swojej kamienicy i wypalał powoli papierosa, gdy dołączył się do niego Brook. Usiadł obok niego i zaczął nawijać jak najęty.
- Gdybyś tylko widział jak wszyscy wstrzymywali oddech jak wasza trójka dyskutowała. Ta dziewczyna to naprawdę córka Bena?
- Tak, ale adoptowana - wyjaśnił Stone.
- Nieźle, nie wiedziałem, że ma rodzinkę. 
- Nie chwali się tym przesadnie i dobrze. Chociaż tym razem lepiej by było gdyby Zack wiedział. 
- Nie spodziewał się takiego ataku z twojej strony. Nieźle sobie poradziłeś.
- Dzięki.
Brook wziął papierosa od Stona i zaciągnął się porządnie.
- Kim jest ta dziewczyna dla Ciebie?
- Dla mnie? – Stone spojrzał na Brooka badawczo. 
- No tak, broniłeś jej, a potem wybiegłeś za nią z domu i nie wróciłeś. 
- To był obowiązek, robiłem to dla Bena.
- Ta jasne - Brook szturchnął go przyjaźnie w ramię, na co Stone wybuchł śmiechem. – A ja jestem święty. 
- To na pewno nie. 
- Gadaj - zarządał kumpel.
- Co mam Ci powiedzieć? Ma na imię Sky, jest blondynką - zaczął Stone wciąż wymigując się od konkretnej odpowiedzi, ale Brook szybko mu przerwał. 
- Nie chrzań. Podoba Ci się?
- Tak - odparł bez wahania Stone i wziął od Brooka swojego papierosa. 
- Ile ją znasz?
- Od poniedziałku. 
- Jak Ci idzie?
- Średnio, nie umiem zaskarbiać sobie uwagi dziewczyn - stwierdził.
- Chyba nie masz wprawy, co? Kiedy ostatnio starałeś się o jakąś dziewczynę?
Stone udawał, że głęboko się nad tym zastanawia, choć odpowiedź była bardzo prosta. 
- Będzie jakieś dwa lata z hakiem i nie nazwałbym tego staraniem. To dużo bardziej skomplikowane. 
- Czyli podsumowując jesteś beznadziejnym adoratorem- skwitował Brook.
- Wypraszam sobie. Z własnej woli nie zalecam się do dziewczyn, ale jestem dla nich przyjacielski.
- To czemu zarywasz do tej? – nie rozumiał.
- Przeczucie. 
- Myślałem, że kobiety się nim kierują. 
- No tak, bo mężczyźni wolą przyrodzeniem – zażartował.
- Tą wersję słyszałem. A co ci mówi twoje?- dał mu kuksańca w bok, a co Stone oddał mu tym samym, śmiejąc się.
- Myślę głową i sercem – wrócił do poważnego tonu Stone.
- To co ona ma takiego?
- Powiem Ci jak ją poznam- podniósł się i rzucił papierosa na ziemię i przydeptał go butem.
- Chcę to mieć na bieżąco – zarządał Brook. 
- To pooglądaj telenowelę – poklepał go po ramieniu Stone. 
Wybuchli śmiechem i wrócili do domu. 



Rozdział 3

Bramkarz przywitał Stona po imieniu i po uiszczeniu zapłaty, pozwolił wejść do klubu. Przeszli przez drzwi i oczom Hope ukazał się wąski i wysoki, oraz okryty półmrokiem korytarz,który otwierał się na wielką salę o takim samym oświetleniu i wypełniony ludźmi. Po lewej stronie zbici w jedną masę tańczyli klubowicze, a stojący na podwyższeniu DJ-ej zapuszczał stosowną muzykę. Po prawej było miejsce z kanapami i niskimi stolikami, gdzie w większości siedzieli mężczyźni, czasem też z dziewczynami na kolanach, głośno rozmawiając, popijając trunki i wypalając tytoń. Na wprost znajdował się bar, za którym kręciły się barmanki obsługując klientelę. Tam też się skierowali. 
Hope czuła się nieswojo w nowym miejscu, nie sądziła, że kiedykolwiek zawita w progi tego przybytku. Do tej pory chodziła tylko na domówki i to jej wystarczało. Ale kiedy rano po pogodzeniu się ze Stonem zaproponował wspólny wypad do klubu, nie miała nic przeciwko. Nie była pewna czy zaprosił ją na randkę, bo wyraźnie tego nie sprecyzował, a ona nie miała żadnego doświadczenia w sprawach damsko-męskich, by to stwierdzić. Zapobiegawczo nie wystroiła się za mocno, tylko dla odmiany włożyła dawno nienoszoną koronkową bluzkę. Była to wielka odmiana jak na to, że jej zwykłym lookiem był –T-shirt bez dekoldu i dopasowane dżinsy. Nie miała w swojej szafie żadnej spódnicy i tylko jedną sukienkę. Nie wiedziała czym powodowany był jej styl, ale jedyną sensowną odpowiedzią była taka, że zakryta od stóp po głowę w stroju nie krępującym ruchów najłatwiej mogła się bronić i uciekać. Zastanawiała się czy kiedyś zmieni imige i czy będzie czuła się bezpiecznie i dobrze w innym ubraniu. 
Usiadła na wysokim stołku przy barze i popatrzyła na butelki z alkoholem rozciągnięte na całej powierzchni ściany, do wyboru, do koloru. Miała nadzieję, że znajdzie się coś bezalkoholowego do picia. Nie chciała pić alkoholu, poza tym gdyby nawet chciała widmo kary jaką dałby jej Ben, skutecznie odpędzało takie zamiary. Kiedy się dowiedział, że została zaproszona przez Stona, był bardzo niezadowolony. Powiedział, że nigdzie nie pójdzie. Zaoponowała, co przerodziło się w kłótnie. Podniosła w niej fundamentalny argument, że idzie ze Stonem, a więc z kimś kogo Ben zna i komu ufa. W złości dodała również, że nie jest jej ojcem, żeby jej czegokolwiek zabraniać. Myślała, że go tym rozwścieczy, ale albo przyzwyczaił się do tego tekstu, albo nie pokazał, że go to ubodło. Ostatecznie po litanii rad, obietnicy, że w jej ograniźmie nie znajdzie się ani kropelka wyskoprocentowego napoju i zatwierdzeniu stroju, puścił ją. 
- Czego chcesz się napić? Ja stawiam – spytał Stone. 
- Może być cola albo jakiś sok. Są tu takie rzeczy? 
- Coś się znajdzie - odparł z uśmiechem. 
Po chwili zjawiła się barmanka o rudoblond włosach i zielonych oczach. 
- Cześć Spencer- przywitał się z nią. 
- Cześć Stone, widzę że wreszcie przyprowadziłeś ze sobą niewiastę. To pamiętny dzień i nowy etap twojego życia – powiedziała podniośle i zaśmiała się pod nosem, na co Stone przewrócił oczami. 
- Poznaj Sky. Sky to Spencer. Spencer to Sky. 
Podały sobie ręce nad blatem i obdarzyły się zdawkowym uśmiechem.
- Co zamawiacie?- spytała rzeczowo. 
- Dwa razy colę. 
Spencer zmarszczyła brwi.
- Serio, nawet nie wiem czy coś takiego tu serwujemy. 
- Proszę Cię – upomniał ją. 
- Jak nie ma dowodu to nie problem, sprzedajemy nieletnim. A przy okazji nie wiedziałam, że umawiasz się z dziećmi. Ile ty masz lat?- skierowała to pytanie do Sky.
- Szesnaście – odpowiedziała Sky z niezadowoleniem przygladając się przebiegowi rozmowy. 
- Serio? Nie wierzę. Nie mogłeś znaleźć sobie kogoś zbliżonego wiekiem? Przecież ona ma prawie tyle co Allison, a nią opiekujesz się jak dzieckiem. 
Hope stwierdziła w duchu, że nie polubi tej dziewczyny. 
- Dasz mi co chcę, czy mam poprosić kogo innego?- rzucił ostro Stone. 
- Okej, okej. Ja tylko mówię w dobrej wierze. Już podaję, jeśli to rzeczywiście znajduje się w naszym inwentarzu. 
Stone prychnął, a Spencer poszła poszukać zamówienia. Stone odwrócił się do Hope ze skruchą w oczach. 
- Przepraszam za nią. Jest – westchnął – trudna i lubi wtykać nos w nie swoje sprawy. 
- To nie twoja wina.
- Taa, ale wiedziałem, że tak może zareagować. Nie lubi dzielić się swoim terytorium. 
- Czekaj. Byliście kiedyś razem?- ta myśl nie przypadła jej do gustu.
- Nigdy w życiu. Mieszkamy razem jeszcze z dwójką osób Allison i Brokiem, którego widziałaś na imprezie. Rozmawiałem z nim, przed naszym spotkaniem przy schodach. 
- Pamiętam. Od dawna trwa ten układ?
- Od dwóch lat i jakoś się trzyma. 
Spencer wróciła z dwiema colami i ostentacyjnie postawiła je przed nimi. Stone zapłacił. 
- Serio bez napiwka?
- Za niemiłą obsługę - wyjaśnił jej ze spokojem Stone. 
- Jesteś niemożliwy.
- I nawzajem. 
Po tych słowach odeszła do innego klienta, a atmosfera między Stonem a Hope się oczyściła. Wzięli napój i stuknęli się butelkami. 
- Za pierwszą randkę – oznajmił Stone. 
- Czyli jednak to randka- ni to spytała, ni stwierdziła.
- Tak, nie wiedziałaś? – spytał zbity z tropu. 
- Nie sprecyzowałeś, a ja nie byłam pewna.
- Cholera przepraszam. Myślałem, że to oczywiste, że mi się podbasz i nie chcę się tylko zakumplować. Jak zgodziłaś się na coś, czego nie chcesz, to przekształcimy to w spotkanie towarzyskie. 
Hope była zaskoczona bezpośredniością i zaniepokojona oczekiwaniami, które ma wobec niej Stone. 
- A czym różni się jedno od drugiego? – spytała kontrolnie.
- W praktyce niczym, tylko intencją. Jeśli to miałaby być randka to starałbym się Ci zaimponować, a jeśli nie to bym z tego zrezygnował. 
- Czyli nie chodzi o ... ?- nie dokończyła, bo przekaz był wymowny. 
- O seks? Nie. Nie jestem tego typu facetem. 
- W takim razie to może być randka – zgodziła się z uśmiechem, na co odpowiedział tym samym. 
Przechylili butelki i wypili połowę jej zawartości. 
Przez chwile się nie odzywali, a Stone uważnie się jej przypatrywał. 
W końcu spytał. 
- Czemu wcześniej Cię nie spotkałem? 
Zaśmiała się. 
- Bo poza domówkami nie bywam na imprezach i jestem mało towarzyska. 
- Ta dzielnica nie jest taka duża, poza tym ćwiczę na tej samej hali co ty od dwóch lat. 
Kiedy dowiedziała się o tym rano, gdy zaczął ćwiczyć pod koniec jej treningu, też była zdziwiona jakim sposobem nie doszło do wcześniejszego spotkania. Najwyraźniej Ben starannie o to zadbał. 
- Nie wiem. A jakie to ma znaczenie?- spytała nie rozumiejąc do czego dąży. 
- Zwyczajnie się zastanawiam, co by było gdyby. 
- Dawno się przekonałam, że takie rozważania nie mając sensu- rzuciła lekko, choć wspomnienia, które ukazały się jej w umyśle nie należały do łatwych. 
- Wiem. Chcesz zatańczyć?- zmienił temat. 
Pokiwała głową. Dopiła cole i prowadzona przez Stona wmieszała się w tłum. Znalazł kawałek wolnej przestrzeni i tam pozostali. Elektroniczna muzyka tętniła jej uszach i poczuła jak przenika do jej ciała. Rytm, tempo, nastrój, tekst. Wchłaniała to, by przetworzyć  na kroki. Początkowo czuła się skrępowana. Dotychczas nie tańczyła w miejscach publicznych i przełamanie się wymagało się wymagało czasu. Musiała wyrzucić z głowy pytania jak ktoś oceni jej taniec, czy uzna że tańczy fatalnie czy świetnie. Stone zbliżył się do niej jakby poczuł, że potrzebuje wsparcia. 
- Coś się stało?
- Po prostu nigdy nie tańczyłam przed kimś, poza bratem. Mam barierę. Nie przejmuj się, może mi przejdzie. 
- Mogę Ci pomóc- zaoferował się.
- Niby jak?
- Mogę prosić do tańca?- wyciągnął w jej stronę dłoń. 
- To nie muzyka do tańczenie we dwoje, tylko osobno.
- Jesteś tego pewna?- spojrzał na nią z przebiegliwością. 
Zastanawiała się co knuje, ale szybko się przekonała. Wziął ją za ręce i przyciągnął do siebie. Nie poczuła się niekomfortowo. Miała wrażenie jakby przeszła przez tajemne wrota do innego świata, jego świata. Owionął ją zapach jego wody kolońskiej i brzoskwinie, oraz jeszcze jakiś inny, który nazwała jego charakterystycznym zapachem. Z tej odległości mogła podziwiać zarys jego szczęki, dostrzec małą bliznę w kąciku ust, która pewnie była pamiątką po rozcięciu sobie ust w dzieciństwie podczas upadku; szorstką skórę po goleniu ; bliznę na łuku brwiowy; spierzchnięte usta i każdy detal jego twrzy. Otrząsnęła się z zachwytu, by nie zdążył go zauważyć. Zaczęli tańczyć. Początkowo było to powolne poruszanie się w takt muzyki przetykane obrotami, zawijasami, ale z czasem oboje nabrali rozpędu i tańczyli wciąż razem choć jakby osobno. Zniknęła niewidzialna bariera w umyśle Hope i tańczyła tak jak chciała. Prezentowała kroki, które sama wymyśliła z tymi, które podpatrzyła w telewizji. Stone dotrzymywał jej tempa i starał się dopasować do jej tańca, co  szło mu świetnie. Ich kroki współgrały, jakby wcześniej stworzyli choreografię i przećwiczyli ją kilka razy. To było niesamowite. Hope nigdy wcześniej się tak nie czuła. Swobodnie, cudownie, idealnie. Hope miała wrażenie, że tworzą ze Stonem mały spektakl, bo czy nie tym był taniec. Grali w nim bratnie dusze, które rozumiały się bez słów i jakby wiedziały jaki następny krok wykona druga osoba jeszcze nim to zrobiła. Kiedy muzyka zmieniła się na bardziej zmysłową, odpowiednią do tańczenia już bardziej razem niż osobno, nie miała oporów, by znaleźć się blisko niego i tańczyć w bardziej kobiecy sposób. To przychodziło naturalnie. Stone się chyba tego nie spodziewał, tej odwagi z jej strony, ale wyraźnie się ucieszył. Teraz nie grali bratnich dusz, tylko zafascynowaną sobą parę. Mężczyznę i kobietę podziwiających swoją aparycje, seksualność. Stone położył jej dłonie na biodrach, a ona jakby nimi kierowana zakręciła nimi w takt muzyki. Przybliżyła się i zarzuciła mu ręce na szyję, a on załapał ją w tali i odchylił do tyłu. Gdy wrócili do pionu ich twarze dzieliły centymentry i Hope nie wiedziała czego się spodziewać ani czego by tak naprawdę chciała. Ale wrócili do tańca, tak zwyczajnie jakby iskry wcale nie przeskoczyły pomiędzy nimi wzmagając wzajemne zainteresowanie. 
Hope nie wiedziała ile tańczyli, ale w pewnym momencie Stone zaproponował by zrobili przerwę. Chętnie na to przystała. Czuła się wyczerpana, włożyła masę wysiłku w taniec. Więc niech ktoś tylko powie, że taniec to nie sport. Oboje wyglądali na zmachanych, kilka kropelek potu błyszczało na czole Stona i Hope była pewne, że na jej również. Poszła do łazienki, żeby obmyć twarz. W środku były trzy kabiny i każda była zajęta i wypełniona jękami. Hope ochlapała wodą twarz i wyszła. Znalazła Stone z Brookiem. 
- Hej. Przedstwię Was sobie oficjalnie. Sky Brook. Brook Sky – zapoznał ich Stone.
- Miło poznać. Dzięki tobie wygrałem zakład- odezwał się Brook.
- Jaki zakład? – spytała marszcząc nos.
- Nie miałem z tym nic wspólnego, słowo. Dowiedziałem się przed chwilą – zatrzegł Stone. 
- Wyjaśnię Ci, to nic zbereźnego. Zwykły zakład o to czy Stone przyprowadzi kiedyś jakąś dziewczynę do klubu.  
- Czyli naprawdę wszyscy się tym tak przejmują – niedowierzała Hope. 
- Wszyscy poza głównym zainteresowanym. Nikt nie mógł dojść dlaczego ten facet, mój kumpel pragnę nadmienić, ma dobry kontakt z płcią piękną i płeć piękna go pociąga, a ani razu żadną się nie zainteresował. Niby nic złego, ale człowiek się zastanawia. Ktoś obstawiał, że żyjeszcz w celibacie i tak będzie do końca twoich dni. Na szczęście on też przegrał zakład. 
Stone słysząc to nie czuł się zapewne komfortowo, ale nie okazał tego.
- Okej, koniec tego – uciął, zanim Brook powiedziałby coś jeszcze. – Wracaj do swoich ziomków, serdecznie ich ode mnie pozdrów i przekonaj ich, że moje życie nie jest ich sprawą. 
- No, weź. Nie obruszaj się tak. To zwykła troska. 
- Ja jestem spokojny patrząc na okoliczności – prychnął.
Rzeczywiście nie złościł się, opanował emocje albo po prostu wcale tak bardzo się tym nie przejął. Hope nie wiedziała, która wersja jest prawdziwa. 
- Okej, już idę. Ale wiesz dobrze, że będą się tobą interesować nawet po moich prośbach. Są jak przekupki na targu, ględzą o wszystkim i o wszystkich, jakby własnego życia nie mieli. 
Brook odwrócił się, by wrócić do swoich koleżków, ale Hope go zatrzymała. 
- Zaczekaj. Może pójdziemy z tobą? – zasugerowała. 
- Co?- spytał Stone wybałuszając na nią oczy. 
- Mogłabym poznać twoich znajomych. 
- Myślę, że to jeszcze za wcześnie – przekonywał Stone. 
- A ja nie. Mam ochotę kogoś poznać. 
Sama się sobie dziwiła. Poznawanie ludzi wiązało się z ryzykiem narażenia ich na niebiezpieczeństwo i możliwością, że ktoś odkryje kim tak naprawde jest. Jednak z drugiej strony poznając jakieś grono mogła dowiedzieć się wielu przydatnych infromacji, a może nawet zapewnić sobie jakieś wsparcie, gdyby kiedyś żołnierze po nią przyszli. Ciężko było przesądzić rezultat tej decyzji. 
- Zostaw nas na chwile – poprosił Brooka Stone. 
Chłopak oddalił się i usiadł na jednej z kanap przyłączając się do gromady mężczyzn. 
- Naprawdę chcesz ich poznać?- zwrócił się do niej Stone. 
- Tak. 
- Mówiłaś, że jesteś niezbyt towarzyska.
- Bo to prawda, ale może najwyższy czas to zmienić. 
- Chcesz mi zaimponować?
- Co, nie - roześmiała się.- Nie muszę tego robić, i tak Ci się podobam- odparła z wyższością.
- Racja- skwitował.- Możemy tam pójść, ale może zrobić się niezręcznie. Zacznął Cię wypytywać o różne rzeczy, będą wścibscy, ordynarni...
- Przestań- przerwała mu wywód. – Nie musisz mnie tak ochraniać. Może i nigdy nie byłam w takiej sytuacji, ale sobie poradzę. To ty nie chcesz tam iść. 
- Masz rację – zgodził się bez wykrętów. 
- No widzisz. Dobrze, że to ustaliliśmy. Jak tak bardzo nie chcesz to tam nie pójdziemy, sama przecież się tam nie zjawię. Ale wytłumacz mi dlaczego.
Namyślał się prze chwile, choć była przekonana, że wcale nie musi. Odpowiedź na to pytanie miał już dawno.
- Słucham- ponagliła go. 
- Nie muszę Ci tego mówić- stwierdził i patrzył na nią z irytacją. 
- Racja nie musisz. Możesz zaproponować coś innego, tańczenie, spacer, a ja mogę się zgodzić. 
- To czemu chcę Ci wyjaśnić. 
- To aż takie dziwne?
- Tak, zaważając, że nie lubię mówić o sobie i motywach, którymi się kieruje. 
- To, rzeczywiści zastanawiające – uśmiechnęła się figlarnie.
- To według Ciebie zabawne?- obruszył się. 
- Trochę.
- Niby czemu.
- Bo odpowiedź jest bardzo prosta.
- Oświeć mnie.
- Albo chcesz mi zaimponować, ale odrzucam tę opcje na wstępie. Ty też dobrze wiesz, że nie musisz się starać, żebym była tobą zainteresowana. 
- Cieszę się, że to przyznałaś. Wiem na czym stoję - uśmiechnął się szeroko.
- Myślałam, że to oczywiste- rzuciła rozbawiona i szczęśliwa, że wyrzuciła te słowa z siebie. 
- Rozumiem aluzję. 
- Kontynując, prawdziwy powód to ten, że chcesz mi powiedzieć, bo chcesz, żebym Cię poznała. A jako, że chyba jestem pierwszą osobą, przed którą chcesz się otworzyć, czujesz się nieswojo i dziwnie. 
- Może masz rację - przyznał niechętnie. 
- Czyli na pewno ją mam- rzuciła zadowolona. – To co idziemy czy proponujesz coś innego?
- Dalej ci nie wyjaśniłem. 
- Racja, ale nie musisz.
- Droczysz się ze mną?
- Odciążam Cię, bo naprawdę nie musisz. Jak to zrobisz to będę Ci dłużna. 
- Wcale nie.
- A jednak. Ty powiesz coś osobistego, to wypadałoby zrobić to samo. 
- Nie wierzę, że o tym dyskutujemy. 
- To dobry początek – zauważyła. 
- Niby tak. Okej, to chodźmy do chłopaków. 
- Jesteś pewien?- powątpiewała. 
- Większości rzeczy jestem pewien w tym tej. Chodźmy. 
Pojawienie się tej dwójki wywołało spore zainteresowanie wszystkich zgromadzonych przy stoliku. Hope się przedstawiła i każdy z nich zrobił to samo. Było ich sześciu albo nawet ośmiu plus Brook. Czuła na sobie ich spojrzenia, które nie tylko badały jej ciało, ale też starały się przeniknąć do jej wnętrza, co oczywiście nie było możliwe. 
- A więc kolega Stone wreszcie znalazł sobie towrzyszkę niedoli. Ludzie jest jeszcze nadzieja dla tego świata. 
To zdanie rozbawiło mężczyzn i zaczęli głośno rechotać. Natomiast dla Hope to zdanie brzmiało jak proroctwo i to na dodatek ściśle z nią związane. Odpędziła te myśli i serdecznie się uśmiechnęła. 
- To od kiedy jesteście parą?- spytał jeden z nich, bodajże Danny. 
- W zasadzie to jeszcze nie jesteśmy, ledwo się poznaliśmy. 
- Hoho, ale widać że Was do siebie ciągnie. 
Stone wyglądał jakby właśnie usłyszał jakieś brednie, ale nikt tego nie zauważył poza nią. 
Nachyliła się do niego. 
- Może kłamie?- spytała. 
- Co?
- Nie zgadzasz się z nim. 
- A o to Ci chodzi. Myślisz tak, bo się skrzywiłem- zaśmiał się. 
- Tylko się z tobą droczę.
- Najwyraźniej to polubiłaś.
- Najwyraźniej ty też. 
Uśmiechnęli się do siebie, na co zareagował Danny.
- A nie mówiłem. Jak siły grawitacji. 
- Bzdury – żachnął się inny, Robin i zaciągnął się porządnie papierosem. – Nigdy nic nie jest pewne. 
- Mógłbyś już skończyć z tym pesymizmem. Żona rzuciła Cię już pięć lat temu, zapomnij człowieku i rusz dalej – poradził Danny. 
- Nie wiesz o czym gadasz. Nic nie przeżyłeś, nic nie rozumiesz. 
- Rozumiem tyle, że se nie radzisz z prawdą. 
- Wal się ty, dupku – warknął na Dannego Robin. 
- Chopaki - zaapelował Matt- uspokójcie się. Zachowujecie się jak dzieci. 
- Masz małe bachory, to wydaje Ci się teraz jakby wszyscy nimi byli – burknął Robin.
- Bo to prawda. Zmieniła mi się perspektywa. 
- No widzisz, to się nie odzywaj. Wielki Pan Wychowaca. 
Trzech mężczyzn sprzeczało się ze sobą, a reszta słuchała wolno wypalając papierosy i popijąc alkohol. 
Stone też wyciągnął papierosa i go odpalił. Uśmiechnął się błogo wypuszczając smużkę dymu. 
- Nie wiedziałam, że palisz- szepnęła. 
- Wielu rzeczy o mnie nie wiesz- wygłosił prawdę objawioną. 
- To mogłam zauważyć, ale masz mocną wodę kolońską. 
- Podoba Ci się?
- Tak- odparła bezwiednie,a potem się zmitygowała do czego się właśnie przyznała. 
Stone uśmiechnął się jak dziecko.
- Dlaczego mnie pytasz o takie rzeczy- żachnęła się. – Zmieniłbyś dla mnie perfumy, gdyby mi nie odpowiadały?
- Tak, przecież jakoś musisz ze mną wytrzymać. Jak będę Cię odstraszać zapachem to daleko nie zajedziemy.
- Jaki pragmatyczny.
- Raczej praktyczny.
- Ja nie będę się ubierać inaczej, żeby Ci się bardziej podobać. 
- Nic takiego nie sugeruję. 
- Ale składasz deklaracje.
- Owszem, ale one dotyczą wyłącznie mnie. 
- I to niby nie działa w obie strony. 
- Masz chyba mylne pojęcie o związkach.
- Jeszcze żaden nie powstał.
- Ale właśnie raczkuje.
- Ja bym powiedziała, że jest na etapie kilkudniowego zarodka. 
- To duża rozbieżność.
- Zauważyłam.
Zaśmiała się i oparła na kanapie, gdzie Stone trzymał rękę. Miała ochotę się do niego przytulić. Zapach tytoniu jej nie przeszkadzał, był przyjemny, choć szkodliwy. Zrobiła się lekko senna, więc przysunęła się do Stona i oparła na jego ramieniu. Poczuła mocniej zapach używki. Stone objął ją lekko ręką, a ona złapała jego dłoń. Dotknęła szorstkich i lekko zgrubionych opuszków i pogładziła materiał rękawiczki, którą nosił. Chciała spytać czemu je nosi, ale miała wrażenie, że to nie pora i miejsce na to. Ten wieczór mógłby zakończyć się pięknie, ale tak się nie stało. 
Niespodziewanie obok ich stolika pojawił się nie kto inny, tylko Zack Sajczento we własnej osobie. Hope od razu przybrała bojową pozycje.
- No, no, staczasz się – skierował te słowa do Hope, czym reszta kanapowiczów była wyraźnie zdziwiona, poza Stonem i Brookiem. 
- To nie adekwatne sformułowanie- odparła. 
- Myślę, że wręcz przeciwnie. 
- O co Ci znowu chodzi?- rzuciła poirytowana. 
- Zobaczyłem Cię i nie mogłem się powstrzymać, by nie zamienić z tobą paru słów.
- Już zamieniłeś. A teraz spadaj i nie psuj atmosfery. 
Poczuła, że Stone lekko się spina i szykuje się, żeby ją wesprzeć w razie potrzeby. Natomiast Brook, Danny, Robin i reszta udawali, że obok nich nie toczy się rozmowa. Na krótko spojrzała na ich miny. Wyglądali jakby chcieli być gdzieś inczej. Naprawdę tak się bali Zacka?
- Powiedziałbym, że ty psujesz atmosferę. Nie pasujesz do tego miejsca. 
- Ja nie powoduje, że ludzie nagle milknął. 
- Oni po prostu są rozsądni, nie to co ty. Wiedzą z kim nie zadzierać. 
- Zluzuj stary – włączył się do rozmowy Stone. – Jest sobotni wieczór...
- Znowu ta sama śpiewka – uciął mu Zack. – Mam cię serdecznie dosyć. Zawsze mi się stawiasz, choć dobrze wiesz jak dużo masz do stracenia. Jesteś zwyczajnie głupi. 
- A ty zwyczajnie udajesz, że jesteś kimś ważnym, a nie jesteś. 
- Licz się ze słowami - warknął Zack.
- Liczę się, to ty mnie sprowokowałeś. 
- Słyszałem, że umiesz być opanowany w każdej sytuacji. 
- Nie widzisz jestem oazą spokoju –  uśmiechnął się z wyższością Stone i zaciągnął papierosem. 
Hope nie wiedziała w jakim kierunku zmierza ta rozmowa, ale chciałaby żeby się zakończyła. 
- Jak chcesz mi coś ważnego przekazać to to zrób i rozstańmy się w pokoju, okej?- zaproponowała Hope ugodowym tonem. 
- Mam Ci wiele rzeczy do powiedzenia, choć pogadamy w cztery oczy. 
- Nie ma mowy.
- Tak się mnie boisz?- zadrwił.
- To raczej ty powinnieneś – rzuciła na odczepkę. 
- Okej, chciałem Ci tylko powiedzieć, że twój immunitet może niedługo wygasnąć, a wtedy moi koledzy chętnie się z tobą spotkają. 
Zrozumiała aluzję i momentalnie  zaczęła szybciej oddychać i zbladła. Próbowała zapanować nad emocjami, ale nie dało rady. Zack widząc jej reakcje uśmiechnął się z triumfem. 
- Co nie polubiłaś ich? 
Nie dała rady nic wykrztusić. Zalała ją fala obrazów. Dobijanie się do drzwi, próba ucieczki, wyłamanie drzwi, czterech mężczyzn czerpiących satysfakcję z jej strachu, cios za ciosem, krew ściekająca na posadzkę, przeraźliwy ból. 
Poczuła ręka Stona zaciskającą się na jej talii i wróciła do teraźniejszości. Wszystko było na miejscu, włącznie z Zackiem. Zapałała do niego ogromną nienawiścią, chciała zrobić mu to, co zrobiono jej. Zamiast tego syknęł w jego stronę. 
- Wynocha.
- Słyszałeś spadaj- poparł ją Stone. 
- Już idę, chciałem tylko, żebyś była świadoma, co Cie czeka. Może dzięki temu nabierzesz trochę ogłady. 
Wzięła pierwszą lepszą szklankę i wylała jej zawartość na Zacka. Był zaskoczony. Patrzył to na swoją mokrą koszulkę to na Hope. 
- Doigrałaś się- rzucił wściekle. – Powiem im, żeby wchodzili w Ciebie od przodu, kiedy będą Cię gwałcili, tak żebyś musiała patrzeć im w oczy.
Na te słowa Stone się podniósł i przywalił Zackowi z całej siły w twarz i na tym nie poprzestał. Poprawił ciosem w nos, tak że krew zaczęła ciec obwicie na ubranie i podłogę .Tamten nawet nie próbował się bronić, kompletnie wytrącony z równowagi. 
Wokół zdarzenia zebrał się tłum. Niektórzy odciągnęli Zacka na bok i łypali z nienawiścią na Stona. To musieli być jego mafijni przyjaciele, bo jeden z nich stanął przed Stonem z groźną miną. 
- Coś ty zrobił, Stone. Tony da Ci teraz tak w dupę, że się nie pozbierasz. 
- Tony nic mi nie zrobi, jeśli chce mieć wciąż broń- odpowiedział chłodno Stone, wypalając papierosa, który podczas starcia cały czas trzymał w palcach. 
- Nie bądź taki pewien. Obierwiesz ty i ta mała. 
- Pierdol się, Winsko. 
Z tymi słowami ruszył do wyjścia, dając Hope do zrozumienia, żeby zrobiła to samo. 
Gdy wyszli na zewnątrz owionęło ich chłodne powietrze. Hope czuła się koszmarnie i miała poczucie winy. 
- Przepraszam- rzuciła w stronę Stona. 
Szli przed siebie, Hope dokładnie nie wiedziała gdzie, ale on wydawał się pewien. 
- Nie masz za co. Ten gnój dostał na co zasłużył. 
- Będziesz miał przez to kłopoty i Ben, i ja. 
Hope raptownie się zatrzymała, bo jej oczom ukazała się wizja, przed którą przez ostatnie miesiące mózg ją ochraniał, budząc ją w odpowiednim momencie. Widziała siebie przyciśniętą do ściany tym razem tyłem. Błagała o litość, ale jej nie słuchali. Była sparaliżowana, gdy zsuwali jej spodnie i gdy wdzierali się w nią. Bolało, tak bardzo. Musiała patrzeć im w oczy i z każdą chwilą coraz bardziej się rozpadała. Czuła jak staje się martwa w środku, pusta. To kim była nagle zniknęło. Zabrali jej tożsamość, poczucie, że jest człowiekiem, że jest coś warta. Przestała w pewnym momencie czuć ręce na piersiach i przyśpieszony oddech, i ból, ruch. Nic nie istniało. Niekończąca się fala obojętności i pustki. 
Poczuła, że ktoś potrząsa jej ramionami i wyszła z przerażającej wizji. Stone patrzył na nią z przestrachem. 
- Co się stało?- prawie krzyknął. 
Nie odpowiedziała, bo poczuła straszne mdłości. Zdążyła się odwrócić i zwymiotowowała. Nie mogła przestać jakby wszystko co miała w żołądku musiało zniknąć, by pozbyła się obrazów z głowy. 
Odepchnęła Stona, który podtrzymywał jej włosy i odeszła parę kroków, cała drżąca. Chciał się zbliżyć, ale dała mu znak by się zatrzymał. 
- Nie podchodź- zażądała.
- Cała drżysz. Chcę Ci pomóc. 
- Idź sobie.
- Nie zostawię Cię w takim stanie. 
- Zjeżdżaj!- wrzasnęła, ale nie zrobił żadnego kroku. 
Odwróciła się do niego plecami i się rozpłakała. 
- Dlaczego nie chcesz, żebym Ci pomógł?- spytał z bólem.
- Nie chcę, żebyś mnie taką widział. 
- Czyli jaką?
- Załamaną. 
Nie czekał aż zmieni zdanie, tylko podszedł i ją objął, a ona nie zaprotestowała. Rozryczała się jeszcze bardziej. Rzadko wypłakiwała się komuś w rękaw. Nie lubiła płakać, odbierała to jako przejaw słabości, na którą nie może sobie pozwolić. Jeśli ktoś był przy tym jak płakała, czuła potem wstyd, dlatego robiła to w samotności. 
Wchłaniała zapach jego bluzki i napawała się jego ciepłem próbując się uspokoić, co nie było takie łatwe. Łzy nie chciały przestać płynąć, a rozrywające pierś uczucie jej nie opuszczało. 
Gdy poczuła się lepiej odsunęła się od Stona tylko trochę i zadarła głowę, by spojrzeć mu w oczy. Dostrzegła w nich troskę i współczucie, oraz ból jakby dokładnie rozumiał co przeżywa. 
- Dziękuję- wyszeptała. 
- Nie musisz. 
- Nie zostawiłeś mnie. 
- Bo wiedziałem, że tego nie chcesz. 
Pokiwała głową, zgadzając się z nim. 
- Chcesz wiedzieć co się stało?- spytała drżącym głosem.
- Nie musisz mi nic...
- Widziałam jak mnie gwałcili w mojej wyobraźni – przerwała mu, zanim straciłaby odwagę, by wypowiedzieć to na głos. 
Zamikł. W jego oczach odmalował się kolejno ból, złość i znów ból. Przyciągnął ją ponownie do siebie. Stali tak nie odzywając się. Zastanawiała się co myśli o niej i jak tak szybko się do siebie zbliżyli. Może powinni zwolnić. Znał ją tak krótko, a wiedział więcej ważnych rzeczy niż ktokolwiek inny. Nie przyznałaby się Benowi do tego co zobaczyła umysłem, ani Grace. Nigdy też nie zdradziła, że w koszmarach poza biciem jest coś jeszcze, coś co się nie wydarzyło, ale mogło. Nie czuła się źle z tym, że Stone wie. W niewytłumaczalny sposób wiedziała, że to właściwe, że on jest dla niej. 
W milczeniu ruszyli w stronę jej domu trzymając się za ręce. 
- Nie pozwolę by to Cię spotkało- przerwał ciszę. 
- Możesz nie mieć na to wpływu- odparła ze smutkiem. 
- Zadbam razem z Benem, żeby stosunki z mafią się nie pogorszyły, to Cię ochroni.
- Dobrze wiesz, że ostatnim razem wszystko było dobrze między nimi, a i tak stała mi się krzywda. Zack zrobi wszystko, żeby mnie zniszczyć. 
- Nie rozumiem, dlaczego mu na tym aż tak zależy. 
- Mój brat pobił jego brata, bo ten powiedział coś na mój temat. Nie dowiedziałam się co, ale musiało być obraźliwe, bo rozwścieczyło Johnnego. W zamian za to Zack ze swoimi chłopakami pobił mojego brata. To było trzy lata temu. 
- Mięczak, bić słabszych – niedowierzał Stone. – To wtedy Ben zaczął Cię trenować i zeszli się z Grace? 
- Tak, właśnie wtedy. 
- I Zack dalej żywi urazę.
- Ja też. Przez trzy lata zbyt często nie dochodziło do konfrontacji, ale jak już były to on zawsze liczył, że skurczę się w sobię i go przeproszę. Ale tak się nie stało. Nie podoba mu się, że się go nie boję. Dawno nie starliśmy się dwa razy pod rząd, to tym bardziej go rozwścieczyło. Dowiedział się skądś, co mi się stało i ma nade mną władze. Od teraz zawsze będzie do tego nawiązywać. Chyba, że stanie się to co obiecuje, wtedy nie będzie musiał. Wygra. 
- To nie będzie miało miejsca.
- Nie możesz być pewny. 
- Obiecuję...
- Nie możesz mi nic obiecać- przerwała. -Obiecałam bratu, że będzie bezpieczny, a go pobili. Obiecałam sobie, że nie będę narażać innych, a narażam Ciebie. Tyle są warte obietnice – pokręciła głową ze smutkiem. 
- Ja też nie spełniłem wielu obietnic, ale to nie powstrzyma mnie przed składaniem kolejnych. Bo wierzę, że którejś dotrzymam. 
- Optymistyczne podejście. 
- Nic innego mi nie pozostało – odparł z bólem i spuścił wzrok. 
Zastanawiała się co kryje się w jego duszy. Ile bólu zdołał już znieść i ile go jeszcze zniesie. Jakie brzemie dźwiga na swoich barkach. Czy gdyby połączyli siły, podzielili się ze sobą tym co mają, czy byłoby prościej przetrwać kolejny dzień, myśleć o przyszłości i uporać się z przeszłością. 
- Mi też niewiele zostało – dodała, żeby zrozumiał, że są w tej samej sytuacji.
- Z naszych niewiele może zrodzić się wiele – zasugerował. 
- Myślisz, że to wykonalne?
- Tak. Dawno niczego nie byłem tak pewien. 
- Więc przekonajmy się – zgodziła się, siląc się na uśmiech.
Podjęła najlepszą czy najgorszą decyzję w swoim życiu? Czas pokaże, ale teraz chciała wierzyć, że będzie dobrze po mimo wszystko. Zbudują razem świat na nowo i świat musi się z tym pogodzić. 


 
Rozdział 4

Spencer siedziała na łóżku otoczona porozrzucanymi kolorowymi zdjęciami, pochłonięta malowaniem paznokci prawej ręki, gdy ktoś zapukał do drzwi jej pokoju. Odkrzyknęła proszę i do środka wszedł Brook. 
- Hej - rzucił na przywitanie i rozwalił się na łóżku obok niej.
- Uważaj na zdjęcia- ofuknęła go. 
Podniósł się i wygrzebał z pod siebie kolorowe fotografie, po czym wrócił do poprzedniej pozycji. 
- Jak chciałeś się położyć to trzeba było zostać we własnym łóżku- rzuciła poirytowana.
- Ale tam byłbym samotny- zrobił minkę smutnego stworzonka. -„Wolę jedno życie z tobą niż samotność przez wszystkie ery tego świata” – wyrecytował, udając poruszenie. 
Wybuchła śmiechem, do którego jej zawtórował. 
- Okej. A tak serio?- spytała.
- Spędziłem osiem godzin w warsztacie z jednym mężczyzną i stęskniłem się za towarzystwem kobiety.
- Trzeba było popatrzeć na ścianę, na pewno wisiała na niej jakaś uśmiechnięta półnaga niewiasta. 
- Nie zwracam na nie uwagi.
- Jesteś aż tak aseksualny czy aż tak przywoity?
- Po prostu są brzydkie.
- Hoho, jaki wybredny. 
- Mam dobry gust. A propos to muszę Ci się czymś pochwalić - zrobił dumną minę.  
- Dawaj.
- Poznałem kogoś. 
- Nie wierzę!- udawała zaaferowaną. – Toż to cud. 
- Właśnie. Ma na imię Tatiana. 
- Kojarzy mi się z „Tytanikiem”. Też na koniec zatonie?
- Haha, aleś ty zabawna. To czarująca młoda kobieta. 
- Już brzmi źle - wydęła usta. 
- Co? Niby dlaczego?
- Określenia czarująca używasz, gdy ciężko Ci opisać kogoś inaczej. Nie jest ani wystarczająco seksowna i olśniewająca, by to zaznaczyć, ani za nudna. Po prostu przeciętna. 
- Wiesz, nie jestem pewny, czy przypadkiem nie stosujesz nadinterpretacji. 
- Aż takim jesteś specem z języka?- uniosła powątpiewajco brew do góry. – To okaż mi na to papierek. 
- Czarująca. 
- Nie obrażaj mnie. Jestem ponad przeciętną- wycelowała w niego oskarżycielsko palcem. 
- Czemu zawsze, gdy opowiadam Ci o nowo poznanej dziewczynie ty stajesz się głównym tematem. 
- Bo te laski nie dosięgają mi do pięt, więc nie są warte uwagi. 
Założył ręce za głowę i przymknął oczy.
- I tak nic z tego nie będzie - westchnął. 
- Mówisz o Tytaniku?
- Tak, o Tatianie, miło, że zapamiętałaś- rzucił kąśliwie. 
- Jak zadowalasz się byle czym to może wyjść.
Zaśmiał się. 
- No co? Może nie mam racji- żachnęła się. 
- Nie, skądże. Ty miałabyś nie mieć racji - mówił z sarkazmem. – Toż jesteś alfą i omegą. 
- Bez przesady - machnęła ręką z udawaną skromnością. – Poza tym masz szansę udowodnić mi, że mylę się co do tej dziewoi. 
-  Kto tak mówi? 
 - Ja, mało czytasz, to i słownictwo ubogie. 
- Chyba na żadnym polu Ci nie dorównam. 
- W zasadzie nie, ale ja nie naprawię samochodu, więc jak widzisz jest coś co robisz lepiej. 
- Wow, dzięki. Poczułem się doceniony. 
- I o to chodziło. Ładne?- wyciągnęła w jego stronę dłonie. 
- Wyzywające - ocenił po krótkiej obserwacji. 
- Gdybym chciała być wyzywająca to chodziłabym po ulicy w bieliźnie. 
- Miałabyś wtedy tłum adoratorów. Ale mówiłem w kontekście pracy, że wyzywające. 
- Człowieku, w mojej pracy właśnie o to chodzi, by ładnie wyglądać, miło się uśmiechać, a wzamian dostajesz wysokie napiwki i kuszące propozycje. 
- Dużo tych propozycji?- spytał niby od niechcenia.
- Godnych uwagi niewiele, ale wystarczająco by nakarmić swoje wygórowane ego. 
- Ono jest olbrzmie nie wygórowane. Jak stąd do stolicy.
- Przesadzasz.
- No chyba nie. 
- No chyba tak. 
- No chyba nie.
- Koniec! Ja znam się lepiej, więc się już przymknij. 
- Polemizował bym.
- Ach tak, to kiedy mam urodziny?
- Siedemnastego sierpnia- wyrecytował z uśmiechem.
- Ulubiony kolor?
- Czerwony.
- To tylko podstawowe nic nie znaczące informacje- osądziła i podniosła się z łóżka. – Nic ważnego o mnie nie mówią. 
- Ale liczy się, że je zapamiętałem, prawda? 
- Liczy się dla kogo?
Stała przy niegdyś białej komodzie, z której teraz łuszczyła się farba. Tani praktyczny grat. Szperała w niej w poszukiwaniu bluzki i spódnicy do pracy. Nie miała za bardzo w czym przebierać, ale to nie przekreślało dylematu wyboru. Wyciągnęła czerwoną bluzkę na ramączkach z dekoldem w serek i błękitną bluzkę z łagodnym łódkowatym dekoldem, którego w żadnym razie nie możnaby nazwać kuszącym. Przestała słuchać Brooka, który trajkotał coś o wadze zapamiętywanych rzeczy i o innych tym podobnych bzdetach.
Najchętniej pozbyłaby się go z pokoju, ale rówczocześnie czuła się przy nim bezpieczniej niż w swoim towarzystwie. Odłożyła skromną bluzkę do szuflady i zdjęła z siebie wcześniejszy T-shirt. 
- Hoho, nie widziałem, że taki zaszczyt mnie kopnął- zagwizdał z podziwem Brook, przerywając swoje wywody. 
- Nie ekscytuj się tak. 
Wciagnęła materiał na siebie i odwróciła się w jego stronę. 
- Zobaczyłeś tylko plecy, to nic wielkiego- zauważyła. 
- I tatuaż- dodał. 
- Którego kawałek widzisz za każdym razem, gdy mam krótszą bluzkę- zbagatelizowała jego zachwyt. 
- Po prostu lubię na niego patrzeć i zastanawiać się co znaczy- wytłumczył, choć dobrze to wiedziała. 
Podciągnęła koszulkę do góry i pogłaskała bok pokryty czarnym atramentem. Osiem dużych drukowanych liter, które na zawsze odmieniły jej życie. 

C A R O L I N E

Zakryła słowo, by nie zacząć zatracać się w przeszłości. Nie było na to teraz czasu. Zresztą nigdy nie było, przynajmniej to starała się sobie wmówić. Żyj teraźniejszością, zostaw przeszłość, spoglądaj w przyszłość. Ten slogan byłby coś wart, gdyby uporała się z poprzednimi latami i mogła ruszyć dalej. Ale tak nie było. Ciągle żyła tamtymi wydarzeniami, które odmieniły jej życie, które sprawiły, że nie wiedziała jak odnaleźć się we własnej skórze. Kim była? Ile była warta? Nawet nie dla innych, ale dla siebie samej. Nie chciała tych pytań, uciekała przed nimi, ale one wciąż ją dopadały. Więc biegła coraz szybciej i szybciej, i bała się mocniej i mocniej, i tak w kółko. Od trzech lat. Gonienie własnego ogona. Chaos jako wybawienie, porządek jako udręka. Umiejcowienie siebie we wrzechświecie było jak umieszczenie punktu w układzie współrzędnych, który nie miał
osi, jednostek, oznaczeń, a do zadania zabierał się uczniak, który ledwo co poznał tabliczkę mnożenia. 
- Okej, tyłka Ci nie pokażę- oznajmiła znienacka.
Brook spojrzał na nią zdezorientowany.
- No idź sobie, chcę założyć spódnicę- pomachała w stronę drzwi jakby odganiała muchę. 
- A, jasne. Już idę. A nie można było prościej?- zasugerował. 
- A nie mogłeś się domyślić?- odbiła piłeczkę.
- Wiesz, mężczyznom ciężko to wychodzi- podrapał się po głowie.
- Ale jak widzicie nagą kobietę przed sobą to nie macie z tym problemu.
- To niezawodna logika – uniósł palec w górę jakby wygłaszał arcyważną myśl.
- Raczej myślenie kutasem.
- No wiesz co.
- Co? Nie podoba Ci się słowo? A może nie podoba Ci się w moich ustach?- dodała ze śmiechem. 
Jak się spodziewała Brook zaczerwienił się jak chłopczyk, który usłyszał słowo pierś. Wyglądał w takim stanie przeuroczo.
- To nie jest zabawne - rzucił ze złością, bo nie przestała się śmiać. 
- Ależ jest. Zachowujesz się jak pięciolatek. 
- Nie każdy po prostu tak się wyraża jak ty. Niektórych peszy takie- wykonał nieokreślony ruch ręką. 
- Kutasy Cię peszą, a to ciekawe. A jak radzisz sobie z własnym ? Przechodzicie jakąś terapię małżeńską?
- Ty podła kreaturo -zacisnął usta w wąską kreskę, a w jego oczach czaiła się zemsta. 
Doskoczył do niej, przewrócił na łóżko i zaczął łaskotać po brzuchu. Zaczęła wić się pod jego dotykiem w bezskutecznej próbie wyzwolenia. Przy czym nie mogła pohamować śmiechu. 
- Błagam!- krzyczała.- Już nie będę żartować z twojego penisa, tylko przestań. To gilgocze!
Na jego usta wypłynął triumfalny uśmiech. Jeszcze trochę się pokajała i odpuścił. Stoczył się z niej, ale pozostał na łóżku.
- Wiesz- podjęła na nowo temat. – Byłoby łatwiej gdybyś został rozprawiczony.
- Co?
- No tak, oswoiłbyś się z materią i nie krępował tak. Chłopie masz już dziewiętnaście lat, czas najwyższy wziąć się do roboty.
- Nie mów mi jak mam żyć - odburknął, wstał i zaczął wygładzać pomięte ubranie.
- Może ten Tytanik?- nie ustępowała.
- Spencer, błagam. Ja nie prześwietlam twojego życia intymnego.
- Bo moje życie seksualne jest bujne i byś się jeszcze pogubił. Ale twoje to pustynia i uważam, że powinien zakwitnąć na nim jakiś kaktus.
- Czy ty właśnie pod postanią kaktusa przedstawiłaś mój pierwszy raz?- patrzył na nią z niedowierzaniem. - Wiesz, lepiej już się przebierz, bo jeszcze nie zdążysz do pracy. 
Podszedł do drzwi i je otworzył, nim wyszedł, obrócił się jeszcze z miną, którą ciężko było odczytać. 
- Jak się z kimś prześpię, to będziesz ostatnią osobą, której to powiem. 
Po tych słowach zatrzasnął z hukiem drzwi. 
No i się obraził, stwierdziła w duchu, ale szybko zajęła się ważniejszymi sprawami. Niedługo później stała już za blatem baru, obsługując klientelę wszelkiej maści. Z czasen klub zapełniał się coraz bardziej, wigilia piątkowego wieczoru. Spoglądała co jakiś czas na
swoją koleżankę, która była dziś w wyjątkowo dobrym humorze zważając na to, że niektóre wydarzenia na pewno do tego nie  nastrajały. Chłopak ją rzucił, gdy dowiedzial się, że jest w ciąży. Zniosła to z godnością. Szczęście wywołane noszem pod sercem dziecka, przesłaniało jej wszelkie przykrości. Zdążyła już raz zjawić się u ginekologa, który uspokoił ją, że dziecko rozwija się prawidłowo. Nakazał o siebie dbać i zwracać uwagę na wszystkie niepokojące zachowania jej ciała. Spencer również wzięła to sobie do serca i starała się poświęcać więcej uwagi na poczynania współpracownicy oraz to jak inni ją traktują. 
- Spencer - ktoś krzyknął za jej plecami, gdy robiła drinka. 
Odwróciła się, nierozpoznając głosu. Gdy zobaczyła kto był nadawcą tej wypowiedzi, pożałowała swojej decyzji sprzed jakiś dwóch tygodni. Jej oczom ukazał się nie kto inny tylko Harry. Ubrany w elegancką szarą marynarkę i błękitną koszulę rozpiętą przy
kołnierzyku, wyróżniał się na tle innych bywalców tego miejsca. Pociągła przystojna twarz z gładko ogolonym zarostem nie wydawała się jej tak ponętna jak ostatnio. Jego rozciągnięte w uśmiechu wargi nie zachęcały do ich zasmakowania, a figlarne błyski w oku nie zwiastowałay przyjemnej rozmowy, ostrzegały przed czymś, co jej się nie spodoba. Ciarki przeszły jej po plecach. Zdołała wykrzesać z siebie namiastkę uśmiechu. Z udawaną radością przywitała się i spytała co mu podać. Poprosił o whisky, ale po otrzymaniu zamówienia i zapłacie z solidnym napiwkiem, nie odszedł. Siedział przy barze i czuła, że obserwuje każdy jej ruch. Jego leniwe spojrzenie skupiało się na piersiach, pośladkach i nogach, i nie musiała wdzierać mu się do głowy, by wiedzieć jakie fantazje się w niej tworzą. Przełknęła ślinę i starała się zachowywać normalnie, bez nerwowych reakcji. Nie mogła dać po sobie poznać, że się go boi. Dopiero po trzeciej szkaneczce trunku rozpoczął z nią rozmowę. Nawiązał do tego, że nie chciał się
jej naprzykrzać jak Freddi, ale nie może o niej zapomnieć. Nie mógł doczekać się, gdy znów ją zobaczy. Miał ton głosu lekko napitego, ale równocześniej w jego intonacji było coś roszczeniowego, taka niewypowiedziana groźba. 
- Nie chcę usłyszeć od Ciebie, żebym się pierdolił. Proszę umów się ze mną - kontynuował. 
- Nie zmianiłam zdania od ostatniego razu. Odpowiedź brzmi: nie. A teraz nie przeszkadzaj mi więcej, jestem w pracy. 
- Jasne, rozumiem – odparł zrezygnowany. – Pozwól chociaż, żebym postawił Ci drinka. Proszę. Na zakończenie tego małego incydentu. 
Nie ufała mu, nie znała jego intencji, ale chciała, żeby się w końcu odczepił. Więc się zgodziła, czego miała żałować do końca życia. Potem próbowała sobie przypomnieć kiedy to się stało, w którym momencie dosypał jej czegoś do napoju. Chyba wtedy, gdy nachylał się w jej stronę, nad szklanką, ale nie miała pewności. Choć co to za
różnica, stało się. Wypiła drinka, a on wpatrywał się w nią z uśmiechem, który zaczął się jawić w tamtym momencie jako uśmiech szaleńca. I taki właśnie był. Szybko po wypiciu poczuła się dziwnie. Zawroty głowy, problemy z koncentracją, głosy dochodzące jakby zza mgły. Nadia coś do niej mówiła, ale nie rozumiała co. Wszystko zlało się w wielką kolorową plamę, a potem ciemność. 
Po przebudzeniu poczuła kilka rzeczy na raz- to, że jest naga, to, że ktoś nad nią sapie, i to, że ktoś, a właściwie coś rusza się jej w miednicy. Te pare faktów uświadomiło jej w jakim jest położeniu i kim się właśnie stała. Została odurzona, zgwałcona i to nadal się działo. Nie docierało to do niej, może dla tego tak późno zaczęła się bronić. Próbowała zwalić oprawcę z siebie, ale bezskutecznie. Jej nadgarstki zostały przygwoźdżone do łóżka jego dłońmi, a on nie przestawał się nią zadowalać. Wiszącą nad nią twarz Harrego, pokrywały kropelki potu, mięśnie twarzy były spięte z wysiłku. Ile to już trwało? Z jej
gardła dobył się wrzask, ryk. Uderzył ją w twarz, ale nie przestała krzyczeć. Ponowił uderzenie, ale nie dawała za wygraną. Może gdzieś niedaleko są ludzie, usłyszą, pomogą. Byle nie przestać walczyć, trzeba robić cokolwiek. Nie można dać się stłamsić. To jak się teraz zachowa będzie rzutować na całe jej życie. 
Opuścił jej ciało i znikł z pokoju. Przez chwile leżała w bezruchu, otępiała. Otrząsnęła się i podniosła do pozycji siedzącej. Podciągnęła nogi pod brodę i ciasno oplotła je ramionami. Kiwała się lekko w przód i w tył bez celu. Jej mózg nie chciał tego przetwarzać. Dopiero gdy zjawił się w slipach, wybudziła się z transu. Automatycznie odsunęła się w głąb łóżka. Popatrzył na nią władczo z satysfakcją, jakby zrobiła dokładnie to czego sobie zażyczył. 
- Nie uciekniesz przede mną maleńka. 
- Nie jestem żadna maleńka!- chciała odwarknąć, ale wydobył się z niej tylko słaby pomruk. 
- Oh, jesteś. I wiesz kim teraz jesteś, moją dziewczyną. 
- Nie jestem żadną twoją dziewczyną, ty pierdolony gwałcicielu - tym razem głos się jej posłuchał i warknęła, prawie krzyczała. 
- Jaki gwałcicielu, Spencer, kochanie, to tylko taka zabawa. Dobrze się bawimy to wszystko. 
- Gwałt nie jest zabawą!- odszczeknęła. 
- Oczywiście, że nie. To obrzydliwe przestępstwo, coś niewybaczalnego. Ale tu to nie miał miejsca, na szczęście. Przecież się zgodziłaś, nie pamiętasz?- spytał niewinnie, z troską. 
- Ja...- zawahała się. 
Miała pustkę w głowie. Nie wiedziała jak znalazła się w tym pomieszczeniu, czemu Nadia jej nie zatrzymała, co się do cholery działo od momentu, kiedy zemdlała. I co jeśli on ma racje. Co jeśli pod wpływem narkotyku dała mu pozwlenie na to, żeby się z nią przespał. Nie, niemożliwe. On kłamie, chce nią zmanipulować, przekonać, że jej nie skrzywdził. 
- Nie zgodziłam się!
- Dokładnie pamiętam twoje słowa „Bierz mnie całą”. Inaczej bym Cię nie tknął nawet małym paluszkiem. Przecież wiesz, że jestem dżentelmenem.
- Zamknij się. Jesteś manipulatorem. Na nic się nie zgodziłam. Wypiłam tylko tego cholernego drinka. 
- Alkohol uderzył Ci do głowy i dlatego nic nie pamiętasz. Byłaś taka pocieszna i chętna. Szkoda, że Cię nie nagrałem. 
- Nie prawda, kłamiesz! Ty podły skurwielu, zgwałciłeś mnie i zbiłeś, a teraz próbujesz udawać, że jest inaczej!
Poderwała się z łóżka, przełykając wstyd, który poczuła, gdy musiała ponownie pokazać mu się naga. Zgarnęła z podłogi swoje ubrania i gorączkowo je zakładała. Uciec, uciec- mówił jej mózg. Podniosła torębkę i ruszyła do drzwi. Zagrodził jej przejście. 
- Co tak szybko, ślicznotko. Jeszcze nie skończyliśmy. Nie pozwolę wyjść Ci z tym fałszywym przekonaniem z pokoju. 
Pchnął ją do przodu i ruszył na nią. Posuwała się do tyłu, aż natrafiła na ścianę. Przysunął się i oparł ręce obok jej głowy. Spojrzał na nią jak nauczyciel na niesfornego ucznia.
- Zobaczyłem Cię kiedyś w gazetce - zaczął, a jej zmroziło krew w żyłach. - Taka piękna, doskonała. Zapragnąłem Cię mieć. I patrz, spotkałem Cię pare lat później w barze i jak rycerz na białaym koniu pomogłem Ci przegonić namolnego adoratora. Objawiłem się jako ktoś z klasą, zaimponowało Ci to i reszta poszła z górki. Spędziliśmy razem cudowną noc, a potem mnie bezceremonialnie odepchnęłaś. Złamałaś mi serce. Mnie się nie rani, to ja ranię innych. A teraz może Cię zgwałciłem może nie i tak tego chciałaś, podświadomie. Nie będę za nic przepraszał. Nic mi nie zrobisz. 
- Zgnijesz w piekle. Mam znajomości, skopią Ci dupę, że zapłaczesz do mamy.
- Nie odważą się, jestem członkiem mafii kochanie, nie wiedziałaś?- spytał z lekkim zdziwieniem.
Rozdziawiła buzię i szybko ją zamknęła. Gorzej być nie mogło. Jaka głupia idiotka. Miała ochotę palnąć sobie w łeb. Będzie musiała przełknąć złość, żal i wstyd, i udawać, że dzisiejszy wieczór nie miał miejsca. Nie powie Brookowi, Stonowi, nikomu. Chociaż czy naprawdę zamierzała się im przyznać do tego, co się stało? To niestotne, najgorsze było to, że nie miała wyboru. Była bezsilna, bezbronna wobec wszędobylskich macek mafii. Musi zniknąć z tego pokoju jak najszybciej i nigdy więcej nie pić drinków od Harrego. Wypuści ją, jeśli uwierzy, że nikomu nie będzie rozpowiadać o gwałcie i jeśli nie będzie twierdziła, że to rzeczywiście stało się wbrew jej woli.
- Już przetrawiłaś informację jak widzę. Pozwól, że powiem co się teraz stanie. Otóż...
- Nikomu nie powiem- przerwała mu. – Nikt się nie dowie o naszej małej zabawie, nikt nawet nie domyśli się, że wydarzyło się tu coś, co nie powinno, bo nie wydarzyło, prawda?- dodała słodziutkim głosikiem. 
Nie był zadowolony, że nadawała ton tej rozmowie, wyraźnie chciał od niej czegoś jeszcze. Ale czego?
- Bardzo się cieszę, że doszliśmy do porozumienia w tej kwestii. Zuch dziewczynka. Jednak nim opuścisz moje lokum musimy poruszyć jeszcze jeden temat. 
Zrobił pauzę, by zaczerpnąć tchu i nadać rangi kolejnym słowom.
- Twój koleżka Brook ma poważne kłopoty z mafią. Jak wiesz sprzedaje narkotyki, jednym słowiem jest dilerem.
Ukryła zdziwienie i powoli przetwarzała nowe informacje. Po tylu kazaniach Stona o tym, że mafia jest niebezpieczna, że pracując dla niej stajesz się jej niewolnikiem i wykorzystuje przeciwko tobie szantaż, tak po prostu do niej dołączył? Dlaczego? Nie wystarczała mu praca w warsztacie samochodowym. Miał przecież przywoitą pensję, lubił to. Jeśli miał kłopoty mógł powiedzieć, pomogliby mu. Dlaczego milczał? Dobrze znała odowiedź na to pytanie. Choć mieszkali razem nie dzielili się przeważnie swoimi problemami. Nie wiedzieli skąd pochodzą, chyba że ktoś wtrącił to mimochodem. Ich przeszłość była tajemnicą. Byli zbieraniną ludzi, którzy skrzętnie ukrywali to co chcieli i wybierali co powiedzą. Ona też nikomu nie powiedziała o groźbie Harrego, która właśnie się zrealizowała. 
- Podpadł jednemu członkowi mafii i nie wygląda to dobrze. Nie jesteśmy znani z miłosierdzia, a zwłaszcza ten konkretny gość po nie nie sięga. Mógłbym rozluźnić atmosferę, wstawić się za chłopakiem- zaproponował. 
Powiedz wreszcie czego chcesz- popędzała go w myślach, chociaż z drugiej strony obawiała się tego, co usłyszy. 
- Gdybyś została moją dziewczyną, problemy Brooka by zniknęły, może nawet lepiej by mu się powodziło. Wszedłby w struktury mafii, więcej zarabiał, miałby ładniejsze panny, kto wie? Na pewno nie musiałby martwić się już o swój tyłek. 
- A co jeśli odmówię?- spytała, choć odpowiedź mogła być tylko jedna. 
- Zabiją go- odparł krótko i odsunął się od niej, złudne poczucie, że daje jej przestrzeń na decyzje.
Nie wierzyła w to wszystko, co działo się przed jej oczami. Jednego dnia miała zostać zgwałcona, poniżona i miała zgodzić się na układ, który miał ocalić jej przyjaciela. Układ, który sprawi, że de facto stanie się niewolnicą z własnego wyboru. Nie chciała być dziewczyną Harrego, nie chciała się z nim pieprzyć, gadać, to wszystko było odrażające. Nie chciała wejść do tego piekła. Nie, błąd, nie chciała, nie miała wyboru. Życie Brooka kontra jej godność. A co Ci pozostało z tej godności- przypomniał umysł. Gówno- odpowiedziała sobie. 
- Ile to będzie trwać?- spytała. – Ile mam być twoją dziewczyną? I co to obejmuje?
I czemu, kurwa, muszę z tobą pertraktować jakbym była prostytutką – dodała w myślach.
- Urabianie takiego człowieka sporo trwa. Minimum miesiąc musiałabyś być moja.
- Nie wolałbyś pieniędzy? Co Ci przyjdzie z posiadania mnie przez taki krótki czas?- podpuszczała go.
- Widzisz spełnię swoje marzenie. Zawsze chciałem mieć kogoś takiego jak ty u swego boku. 
- A skąd mam wiedzieć, że nie kłamiesz? 
Wzruszył ramionami. Świetnie, wpakuj się w gównaniny układ w imię czegoś co nie istnieje. Brawo. Mogła spytać Brooka, ale mógł nic o tym nie wiedzieć. Harry miał lepszy dostęp do informacji. Musiała to jakoś sprawdzić. Nie ufała mu i podawanie się na tacy, było ostatnim czego pragnęła. Ale jeśli miał rację, to jutro rano może zobaczyć trupa na wycieraczce i nie daruje sobie, że temu nie zapobiegła. 
- Tydzień – oznajmiła. – Wyrób się w tydzień i tydzień będę twoja. 
- Od kiedy to ty stawiasz warunki?- zaśmiał się. – Robię Ci łaskę, że oferuję pomoc. 
Przełknęła ślinę. Czuła, że się poci ze strachu. A może to wina narkotyku. Zamiast na swoich myślach zaczęła skupiać się na jego twarzy. Była tak blisko, widziała każdą zmarszczkę, niedoskonałość, włos. Jego oddech, wolniejszy niż, gdy... Wydarzenia sprzed piętnastu minut odtworzyły się w jej głowie. Przestań, przestań! Pauza! Nogi zaczęły jej drżeć. Poczuła, że ubranie przywiera jej do skóry. A do tego Harry przyglądał się jej uważnie i mylnie interpretował jej zachowanie, co po chwili okazał. Położył jej rękę na tali i przejechał nią w górę po piersi. 
- Nie!- krzyknęła mu w twarz. 
Zachowywał się jakby był głuchy. Wczepił się w jej usta i błądził rękoma po klatce piersiowej. Miała wrażenie, że straciła czucie. Jego dotyk był tak odrażający, że umysł porzucił kontakt z ciałem. Znała ten stan. Wpatrywać się w przestrzeń i udawać, że jest się gdzie indziej. Nie być, nie istnieć, przeczekać, aż się skończy. „ Nie ruszaj się” nakazywała sobie wtedy. Ale teraz nie była pod niczyimi rozkazami, sama była dorosła i nikt nie będzie jej kazać jak ma postępować. Koniec robienia wbrew swojej woli. 
Gdy dotarła wreszcie do bezpiecznego domu, osunęła się po drzwiach wejściowych i łkała. Po mimo deklaracji nie zrobiła nic, by mu przeszkodzić. Tępo oglądała biały sufit, gdy on robił co chciał. Gdy skończył powiedział, że nawet jeśli nie przystanie na ten układ, będzie jego. Jeśli nie kto inny, to on zabije Brooka, byle tylko mieć ją w posiadaniu. Nie odezwała się. Ze spuszczonym wzrokiem, wysłuchiwała jego słów. Zaczęło się, kolejne władanie, czas by do tego przywyknąć. Teraz chociaż poświęcała się dla kogoś, kogo kochała, wtedy dla cudzej ambicji. 
Z trudem podniosła się z podłogi. Wyczerpana powlekła się do łóżka i padła na nie bez sił. Czuła jego zapach, obecność w sobie. Powinna wyszorować się do krwi, ale nawet wtedy nie czułaby się wolna. Dlaczego zawsze spotykały ją nieszczęścia. Dlaczego z taką ławością ludzie traktowali ją jak swoją zabawkę, środek do własnego celu. Ach, to piękno, nad którym wszyscy się rozpływali, które się przeciw niej sprzymierzyło. Czy gdyby była brzydsza, widołaby szczęśliwe życie? A może nie w tym leżał problem. Nie rozumiała tego świata, nie rozumiała ludzi, miłości, siebie, wszystkiego. Jedna wielka beznadziejna zagadka. 
Brook, jego imię zamajaczyło na obrzerzach rozpaczy. Zależało mu na niej. Kochał ją. Nigdy nie wymówił tych słów na głos, ale to wiedziała. Na poczatku ich znajomości starał się jej zaimponować, umówić się, ale szybko ucięła jego zapędy. Powiedziała przyjaciele i nic więcej, a on się zgodził. Jednak to nie zmieniło jego uczuć. Był iskierką nadziei w tym całym poplątaniu. Znosił jej humory, obcesowość, rozwiązłość. Poczuła przypływ siły i ruszyła do jego pokoju. Cicho zamknęła za sobą drzwi i zakradła się do łóżka. Brook spał na wznak z rękami powyginanymi z dziwne pozycje. Uniosła kołdrę i wślizgnęła się pod nią. 
- Spencer?- wymówił ledwo przytomny, otwierając prawe oko. 
- Tak to ja. 
- Coś się stało?
- Nie mogę spać. Miałam zły sen. Mogę zostać?
- Pewnie. 
Przesunął się bardziej na lewo, dając jej więcej miejsca. 
- Chcesz o tym pogadać?
- Nie. Chcę się przytulić. 
Przysunął się i objęła go. Było jej ciepło i bezpiecznie. Nic więcej teraz nie potrzebowała.

\end{document}